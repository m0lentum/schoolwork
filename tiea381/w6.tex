%! TEX program = xelatex

\documentclass{article}
\usepackage[a4paper, margin=3cm]{geometry}
\setlength{\parindent}{0pt}
\setlength{\parskip}{1em}
\usepackage{fontspec}
\setmainfont{Lato}

\usepackage{amsmath,amssymb,amsthm}
% \usepackage{hyperref}
% \usepackage{graphicx}
% \usepackage{pgfplots}
% \pgfplotsset{compat=1.16}

\usepackage{verbatim}
\usepackage{listings}
\usepackage{xcolor}
\definecolor{purple}{RGB}{135,20,85}
\definecolor{gray}{RGB}{100,100,100}
\lstset{
  basicstyle=\ttfamily,
  keywordstyle=\color{purple},
  commentstyle=\color{gray},
}

% matrices with customizable stretch
% as per https://tex.stackexchange.com/questions/14071/how-can-i-increase-the-line-spacing-in-a-matrix
\makeatletter
\renewcommand*\env@matrix[1][\arraystretch]{%
  \edef\arraystretch{#1}%
  \hskip -\arraycolsep
  \let\@ifnextchar\new@ifnextchar
  \array{*\c@MaxMatrixCols c}}
\makeatother

%
% begin document
%

\title{TIEA381 Demo 6}
\author{Mikael Myyrä}
\date{\number\day.\number\month.\number\year}

\begin{document}
\maketitle

\section*{1.}

Eksplisiittinen Euler-askel antaa
\[
  \tilde{y}_{n+1} = y_n + hf_n.
\]
Käyttämällä tätä alkuarvauksena kiintopisteiteraatioon implisiittisessä
Euler-askelessa saadaan
\[
  y_{n+1} = y_n + hf(\tilde{y}_{n+1}).
\]
Nyt $f(y) = -y^2$, $h = \frac{1}{2}$, $y(0) = 1$, ja halutaan ratkaisu välillä $[0,2]$.
Matlab-koodi:

\lstinputlisting[language=Matlab]{w6_1.m}

tulostaa

\lstinputlisting{|"octave w6_1.m"}


\section*{2.}

Tähän voidaan soveltaa samaa ratkaisumenetelmää kuin 1.-tehtävässä
eri $f$:n, $y(0)$:n ja $h$:n arvoilla. $f$ ja $y(0)$ on annettu tehtävässä,
ja valitsen $h = 0.01$. Matlab-koodi:

\lstinputlisting[language=Matlab]{w6_2.m}

tulostaa

\lstinputlisting{|"octave w6_2.m"}


\end{document}

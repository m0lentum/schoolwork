%! TEX program = xelatex

\documentclass{article}
\usepackage[a4paper, margin=3cm]{geometry}
\setlength{\parindent}{0pt}
\setlength{\parskip}{1em}
\usepackage{fontspec}
\setmainfont{Lato}

\usepackage{amsmath,amssymb,amsthm}
% \usepackage{hyperref}
\usepackage{graphicx}
% \usepackage{pgfplots}
% \pgfplotsset{compat=1.16}

\usepackage{verbatim}
\usepackage{listings}
\usepackage{xcolor}
\definecolor{purple}{RGB}{135,20,85}
\definecolor{gray}{RGB}{100,100,100}
\lstset{
  basicstyle=\ttfamily,
  keywordstyle=\color{purple},
  commentstyle=\color{gray},
}

% matrices with customizable stretch
% as per https://tex.stackexchange.com/questions/14071/how-can-i-increase-the-line-spacing-in-a-matrix
\makeatletter
\renewcommand*\env@matrix[1][\arraystretch]{%
  \edef\arraystretch{#1}%
  \hskip -\arraycolsep
  \let\@ifnextchar\new@ifnextchar
  \array{*\c@MaxMatrixCols c}}
\makeatother

% usually I want images 350pt wide and centered
\newcommand{\img}[1]{
  \begin{center}
    \includegraphics[width=350pt]{#1}
  \end{center}
}

%
% begin document
%

\title{TIEA381 Demo 6}
\author{Mikael Myyrä}
\date{\number\day.\number\month.\number\year}

\begin{document}
\maketitle

\section*{1.}

Eksplisiittinen Euler-askel antaa
\[
  \tilde{y}_{n+1} = y_n + hf_n.
\]
Käyttämällä tätä alkuarvauksena kiintopisteiteraatioon implisiittisessä
Euler-askelessa saadaan
\[
  y_{n+1} = y_n + hf(\tilde{y}_{n+1}).
\]
Nyt $f(y) = -y^2$, $h = \frac{1}{2}$, $y(0) = 1$, ja halutaan ratkaisu välillä $[0,2]$.
Matlab-koodi:

\lstinputlisting[language=Matlab]{w6_1.m}

tulostaa

\lstinputlisting{|"octave w6_1.m"}


\section*{2.}

Tähän voidaan soveltaa samaa ratkaisumenetelmää kuin 1.-tehtävässä
eri $f$:n, $y(0)$:n ja $h$:n arvoilla. $f$ ja $y(0)$ on annettu tehtävässä,
ja valitsen $h = 0.01$. Matlab-koodi:

\lstinputlisting[language=Matlab]{w6_2.m}

tulostaa

\lstinputlisting{|"octave w6_2.m"}


\section*{3.}

Muokataan ensin yhtälö muuttujanvaihdolla ensimmäisen kertaluvun yhtälöryhmäksi,
jotta voidaan soveltaa RK4-menentelmää. Merkitään $\theta_0 = \theta$,
$\theta_1 = \theta'$. Saadaan
\[
  \begin{cases}
    \frac{d\theta_0}{dt} = \theta_1 \\
    \frac{d\theta_1}{dt} = -\kappa \theta_1 - \sin \theta_0 \\
  \end{cases}
\]
Ratkaistaan tämä RK4:llä välillä $t \in [0,10]$, kun $\theta_0(0) = 0.5$,
$\theta_1(0) = 0$ ja $\kappa = 0.1$. Matlab-koodi:

\lstinputlisting[language=Matlab]{w6_3.m}

piirtää

\img{w6_3.jpg}


\section*{4.}

Richardsonin ekstrapolointimenetelmä (kerran iteroituna) on
\[
  D_R(x,h) = \frac{4}{3}D_0(x,\frac{h}{2}) - \frac{1}{3}D_0(x,h)
\]
missä $D_0(x,h)$ on keskeisdifferenssiapproksimaatio askelpituudella $h$.
Tämän pidemmälle ei voida annetuilla tiedoilla iteroida, koska
tarvittaisiin enemmän laskentapisteitä kuin on käytettävissä.

Nyt $f(x) = x^x$, $h = 0.2$ ja etsitään likiarvoa $\frac{d}{dx}f(\frac{1}{2})$:lle.
\begin{align*}
  D_R(\frac{1}{2}, 0.2) &= \frac{4}{3}\Big(\frac{f(0.6) - f(0.4)}{0.2}\Big)
  - \frac{1}{3}\Big(\frac{f(0.7) - f(0.3)}{0.4}\Big) \\
                        &\approx 0.2173
\end{align*}


\section*{5.}

Tehdään muuttujanvaihto $y_0 = y$, $y_1 = y'$, $y_2 = y''$, $y_3 = y^{(3)}$.
Saadaan yhtälöt
\[
  \begin{cases}
    \frac{dy_0}{dt} = y_1, \quad 0 < t < 1 \\
    \frac{dy_1}{dt} = y_2 \\
    \frac{dy_2}{dt} = y_3 \\
    \frac{dy_3}{dt} = -1 \\
  \end{cases}
\]
ja reunaehdot $y_0(0) = y_1(0) = y_2(1) = y_3(1) = 0$.


\section*{6.}

$A$:n itseisarvoltaan suurimman ominaisarvon reaaliosa on n. -20.0118
ja pienimmän -0.0441. Näiden välinen ero on kokoluokkaan nähden suuri,
joten näyttää siltä, että tätä yhtälöryhmää voisi kuvailla kankeaksi.

Tarkastelin yhtälöryhmää Octaven työkaluilla. Kun $t \rightarrow \infty$,
niin yhtälöt lähestyvät nollaa lähes lineaarisesti,
mutta lähellä $t=0$:aa tapahtuu jyrkkä mutka.
\verb#ode45#-funktiolla välillä $[0,5]$ sain seuraavan kuvaajan:

\img{w6_6_1.jpg}

\verb#ode23s# sen sijaan näyttää tältä:

\img{w6_6_2.jpg}

Askelpituudessa on selkeä ero alun jyrkän mutkan jälkeen.

Koska kankeudelle ei ole lukuarvoista määritelmää, niin se, miten kankea tämä
yhtälöryhmä on, riippuu vertailukohdasta, jota nyt ei ole.
Ominaisarvojen ja Octaven menetelmien tehokkuuden perusteella sanoisin kuitenkin,
että kyse on yleisesti ottaen kankeasta tehtävästä.

\end{document}

%! TEX program = xelatex

\documentclass{article}
\usepackage[a4paper, margin=3cm]{geometry}
\setlength{\parindent}{0pt}
\setlength{\parskip}{1em}
\usepackage{fontspec}
\setmainfont{Lato}

\usepackage{amsmath,amssymb,amsthm}
\usepackage{algorithmicx}
\usepackage{algpseudocode}
% \usepackage{hyperref}
% \usepackage{graphicx}
% \usepackage{pgfplots}
% \pgfplotsset{compat=1.16}

\usepackage{verbatim}
\usepackage{listings}
\usepackage{xcolor}
\definecolor{purple}{RGB}{135,20,85}
\definecolor{gray}{RGB}{100,100,100}
\lstset{
  basicstyle=\ttfamily,
  keywordstyle=\color{purple},
  commentstyle=\color{gray},
}

% matrices with customizable stretch
% as per https://tex.stackexchange.com/questions/14071/how-can-i-increase-the-line-spacing-in-a-matrix
\makeatletter
\renewcommand*\env@matrix[1][\arraystretch]{%
  \edef\arraystretch{#1}%
  \hskip -\arraycolsep
  \let\@ifnextchar\new@ifnextchar
  \array{*\c@MaxMatrixCols c}}
\makeatother

% usually I want images 350pt wide and centered
\newcommand{\img}[1]{
  \begin{center}
    \includegraphics[width=350pt]{#1}
  \end{center}
}

%
% begin document
%

\title{TIEA381 luentotentti}
\author{Mikael Myyrä}
\date{\number\day.\number\month.\number\year}

\begin{document}
\maketitle

\section*{1.}

Newtonin menetelmää varten yhtälö tarvitaan muodossa $f(x) = 0$.
Nyt $f(x) = 2^x - x^2$.  Lasketaan kahdeksan merkitsevän numeron tarkkuudella,
joten merkitään $\varepsilon = 10^{-8}$.
Olkoon $x_0 \in \mathbb{R}$ jokin alkuarvaus.

Algoritmi:
\begin{algorithmic}
  \State $x\gets x_0$
  \Loop
    \State $x_{next} \gets x - \frac{f(x)}{\frac{d}{dx}f(x)}$
    \If{$|x_{next} - x| \leq \varepsilon$}
      
      \Return $x_{next}$
    \EndIf
    \State $x \gets x_{next}$
  \EndLoop
\end{algorithmic}


\section*{2.}

Eulerin menetelmää varten tarvitaan tehtävä ensimmäisen kertaluvun yhtälöryhmän muodossa.
Tehdään muuttujanvaihto $y_0 = y$, $y_1 = y'$.
Saadaan tehtävä
\[
  \begin{cases}
    y_0' = y_1 \\
    y_1' = -y_1^2 - y_0^2 \\
    y_0(0) = 2 \\
    y_1(0) = 3 \\
  \end{cases}
\]
Nyt etsitään ratkaisua välillä $[0, 1]$ ja askelpituus $h = \frac{1}{4}$,
joten saadaan viisi laskentapistettä.

Merkitään $y_0$:n ja $y_1$:n arvoja laskentapisteissä $x_i$, $i = 0,1,2,3,4$
vektoreilla $\mathbf{y}_i = \begin{bmatrix}
  y_0(x_i) \\ y_1(x_i)
\end{bmatrix}$.
Yhtälö\-ryhmän oikean puolen funktio vektorimuodossa on
\[
  f(\mathbf{y}) = \begin{bmatrix}
    y_1 \\
    -y_1 - y_0^2
  \end{bmatrix}.
\]
ja alkuehto
\[
  \mathbf{y}_0 = \begin{bmatrix} 2 \\ 3 \end{bmatrix}.
\]
Algoritmi:
\begin{algorithmic}
  \State $\mathbf{y} \gets \mathbf{y}_0$
  \For{$i = 0,1,2,3$}
  \State $\tilde{\mathbf{y}}_{n+1} \gets \mathbf{y}_{n} + h\mathbf{f}(\mathbf{y}_n)$
  \Comment{Arvaus eksplisiittisellä Eulerilla}
  \State $\mathbf{y}_{n+1} \gets \mathbf{y}_n + h\mathbf{f}(\tilde{\mathbf{y}}_{n+1})$
  \Comment{Kiintopistekorjaus implisiittisellä Eulerilla}
  \EndFor
\end{algorithmic}
Nyt alkuperäisen toisen kertaluvun tehtävän tuntemattoman funktion arvot ovat
vektoreiden $\mathbf{y}_i$ ensimmäiset alkiot.


\section*{3.}

\subsection*{(a)}

\begin{align*}
  \int_0^3 f(x)\,dx &\approx \frac{1}{2}*\frac{1}{2}(f(0) + 2f(\frac{1}{2}) + f(1))
    + \frac{1}{2}*1(f(1) + 2f(2) + f(3)) \\
                    &= \frac{1}{4}(1 + 2*3 + 2) + \frac{1}{2}(2 + 2*1 + 0) \\
                    &= \frac{17}{4} \\
\end{align*}

\subsection*{(b)}

\begin{align*}
  \int_0^3 f(x)\,dx &\approx \frac{1}{3}*\frac{1}{2}(f(0) + 4f(\frac{1}{2}) + f(1))
  + \frac{1}{3}*1(f(1) + 4f(2) + f(3)) \\
                    &= \frac{1}{6}(1 + 4*3 + 2) + \frac{1}{3}(2 + 4*1 + 0) \\
                    &= \frac{9}{2} \\
\end{align*}


\section*{4.}

Silmukassa ratkaistava yhtälö saadaan muotoon
\[
  (\mathbf{A} + \kappa \mathbf{I})\mathbf{x}^{(k+1)} = \mathbf{b} + \kappa\mathbf{x}^{(k)}.
\]
Tässä matriisi $\tilde{\mathbf{A}}$ = $(\mathbf{A} + \kappa \mathbf{I})$ on
kahden symmetrisen positiividefiniitin matriisin summa, joten se on myös
symmetrinen ja positiividefiniitti tridiagonaalimatriisi.

$\tilde{\mathbf{A}}$ on tridiagonaalimatriisina hyvin harva, joten sitä ei kannata
tallentaa neliömatriisina. Nauhamatriisi voidaan tallentaa joukkona vektoreita,
jotka kuvaavat nollasta poikkeavien diagonaalien arvoja. Koska $\mathbf{A}$ on
lisäksi symmetrinen ja positiividefiniitti, niin sille on olemassa Choleskyn hajotelma
$\tilde{\mathbf{A}} = \mathbf{LL}^T$, missä $\mathbf{L}$ on alakolmiomatriisi,
jolla on sama nauhan leveys kuin $\tilde{\mathbf{A}}$:lla.
$\tilde{\mathbf{A}}$ ei muutu laskennan aikana,
joten hajotelma voidaan laskea kerran ennen tehtävän ratkaisemista.
Siis tarvitsee tallentaa vain
matriisin $\mathbf{L}$ kaksi nollasta poikkeavaa diagonaalia vektoreina.

Näillä valmisteluilla silmukan yhtälö voidaan ratkaista siten,
että ensin ratkaistaan yhtälö
\[
  \mathbf{L}\mathbf{y} = \mathbf{b} + \kappa \mathbf{x}^{(k)}
\]
etenevillä sijoituksilla, ja saadun $\mathbf{y}$:n avulla ratkaistaan
\[
  \mathbf{L}^T\mathbf{x}^{(k+1)} = \mathbf{y}
\]
takenevilla sijoituksilla. En ehdi aikarajan puitteissa kirjoittaa
auki näiden sijoituksien kaavoja, mutta niissä on 
vektorimuotoisen tallennustavan ansiosta vain kaksi kertolaskua per matriisin rivi.

\end{document}

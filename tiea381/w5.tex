%! TEX program = xelatex

\documentclass{article}
\usepackage[a4paper, margin=3cm]{geometry}
\setlength{\parindent}{0pt}
\setlength{\parskip}{1em}
\usepackage{fontspec}
\setmainfont{Lato}

\usepackage{amsmath,amssymb,amsthm}
% \usepackage{hyperref}
% \usepackage{graphicx}
% \usepackage{pgfplots}
% \pgfplotsset{compat=1.16}

\usepackage{verbatim}
\usepackage{listings}
\usepackage{xcolor}
\definecolor{purple}{RGB}{135,20,85}
\definecolor{gray}{RGB}{100,100,100}
\lstset{
  basicstyle=\ttfamily,
  keywordstyle=\color{purple},
  commentstyle=\color{gray},
}

% matrices with customizable stretch
% as per https://tex.stackexchange.com/questions/14071/how-can-i-increase-the-line-spacing-in-a-matrix
\makeatletter
\renewcommand*\env@matrix[1][\arraystretch]{%
  \edef\arraystretch{#1}%
  \hskip -\arraycolsep
  \let\@ifnextchar\new@ifnextchar
  \array{*\c@MaxMatrixCols c}}
\makeatother

%
% begin document
%

\title{TIEA381 Demo 5}
\author{Mikael Myyrä}
\date{\number\day.\number\month.\number\year}

\begin{document}
\maketitle

\section*{1.}

Neljä laskentapistettä riittää kolmannen asteen tarkkuuteen (lause luentodiasta 190).
Ratkaistaan $\mathbf{A}$ yhtälöryhmästä
\begin{align*}
  \sum_{i=1}^k A_ix_i^m &= \int_0^1 x^m\,dx, \quad m = 0,1,2,3 \\
  \iff \begin{bmatrix}[1.25]
    1 & 1 & 1 & 1 \\
    \frac{1}{8} & \frac{1}{4} & \frac{1}{2} & 1 \\
    \frac{1}{64} & \frac{1}{16} & \frac{1}{4} & 1 \\
    \frac{1}{512} & \frac{1}{64} & \frac{1}{8} & 1 \\
  \end{bmatrix}
  \begin{bmatrix}[1.25]
    A_1 \\ A_2 \\ A_3 \\ A_4
  \end{bmatrix}
                        &=
  \begin{bmatrix}[1.25]
    1 \\ \frac{1}{2} \\ \frac{1}{3} \\ \frac{1}{4}
  \end{bmatrix}
\end{align*}
Octaven tulkilla ratkaistuna tämä antaa integrointikaavalle likiarvon
\[
  \int_0^1 f(x)dx
    \approx 0.5079f(\frac{1}{8})
    - 0.4444f(\frac{1}{4})
    + 0.7778f(\frac{1}{2})
    + 0.1587f(1).
\]


\section*{2.}

Kvadraattinen avoin Newtonin ja Cotesin kaava on
\[
  \int_a^b f(x)\,dx \approx \frac{4h}{3}(2f_0 - f_1 + 2f_2)
\]
missä $h = \frac{1}{4}(b - a)$.
Nyt $h = \frac{1}{2}$, $f(x) = \frac{\sin x}{x}$
ja laskentapisteet $x_0 = \frac{1}{2}$, $x_1 = 1$ ja $x_2 = \frac{3}{2}$.
\[
  \int_0^2 f(x)\,dx \approx \frac{2}{3}(2f(\frac{1}{2}) - f(1) + 2f(\frac{3}{2}))
    \approx 1.6041
\]


\section*{3.}

Nyt $h = \frac{1}{2}$, $f(x) = \cos\sqrt{1+10x^2}$ ja laskentapisteet
$x_0 = 0$, $x_1 = \frac{1}{2}$, $x_2 = 1$, $x_3 = \frac{3}{2}$ ja $x_4 = 2$.
Puolisuunnikassäännöllä osaväleittäin saadaan
\[
  \int_0^2 f(x) \approx \frac{h}{2}(f_0 + 2f_1 + 2f_2 + 2f_3 + f_4)
  \approx -1.1014.
\]


\section*{4.}

Samat $h$, $f$ ja $x_i$ kuin edellisessä.
Simpsonin säännöllä
\[
  \int_0^2 f(x) \approx \frac{h}{3}(f_0 + 4f_1 + 2f_2 + 4f_3 + f_4)
  \approx -1.2088.
\]


\section*{5.}

Rombergin ensimmäinen ekstrapolantti on muotoa
\[
  T_{k+1,1} = \frac{4}{3}T_{k+1,0} - \frac{1}{3}T_{k,0}
\]
missä $T_{k+1,0}$ ja $T_{k,0}$ ovat puolisuunnikasapproksimaatioita
osavälin pituuksilla $h_k = \frac{b-a}{2^k}$.
Tarkas\-tellaan tapausta $k=0$, jolloin
\[
  T_{0,0} = \frac{b-a}{2}(f(a) + f(b)) + \alpha_1h^2 + \alpha_2h^4 + \dots,
\]
\[
  T_{1,0} = \frac{b-a}{4}(f(a) + 2f(a+h_1) + f(b))
  + \frac{1}{4}\alpha_1h^2 + \frac{1}{16}\alpha_2h^4 + \dots
\]
ja
\begin{align*}
  T_{1,1} &= \frac{4}{3}T_{1,0} - \frac{1}{3}T_{0,0} \\
          &= \frac{b-a}{6}(f(a) + f(b)) + \frac{2(b-a)}{3}f(a+h_1) + O(h^4) \\
          &= \frac{h_1}{3}(f(a) + 4f(a+h_1) + f(b)) + O(h^4) \\
\end{align*}
mikä on Simpsonin säännön lauseke.


\section*{6.}

Funktion 
\[
  f(x) = \int_0^x e^{\sin t}\,dt - 7
\]
derivaatta on
\[
  f'(x) = e^{\sin x}.
\]
Ratkaistaan $f(x) = 0$ Newtonin menetelmällä.
Matlab-koodi:

\lstinputlisting[language=Matlab]{w5_6.m}

tulostaa

\lstinputlisting{|"octave w5_6.m"}


\section*{7.}

Tehdään muuttujanvaihto $y_0 = y$, $y_1 = y'$, $y_2 = y''$, $y_3 = y'''$.
Tästä saadaan yhtälöryhmä
\[
  \begin{cases}
    \frac{dy_0}{dt} = y_1, \\
    \frac{dy_1}{dt} = y_2, \\
    \frac{dy_2}{dt} + (\sin y_2 + y_1)^2 = \cos t. \\
  \end{cases}
\]
ja alkuarvot $y_0(0) = 3$, $y_1(0) = 2$ ja $y_2(0) = 1$.

\end{document}

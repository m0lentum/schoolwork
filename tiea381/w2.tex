%! TEX program = xelatex

\documentclass{article}
\usepackage[a4paper, margin=3cm]{geometry}
\setlength{\parindent}{0pt}
\setlength{\parskip}{1em}
\usepackage{fontspec}
\setmainfont{Lato}

\usepackage{amsmath,amssymb,amsthm}
%\usepackage{hyperref}
\usepackage{verbatim}
\usepackage{graphicx}
%\usepackage{pgfplots}
%\pgfplotsset{compat=1.16}

\title{TIEA381 Demo 2}
\author{Mikael Myyrä}
\date{\number\day.\number\month.\number\year}

\begin{document}
\maketitle

\section*{1.}

\begin{align*}
  f(x) &= x^3 - 1000 = 0 \\
  f'(x) &= 3x^2 \\
\end{align*}

Lasketaan muutamia $x_n$:n arvoja Newtonin menetelmällä
ja tarkastellaan niiden virhettä.

\begin{align*}
  x_0 &= 2 \\
  x_1 &= x_0 - \frac{f(x_0)}{f'(x_0)} = 2 - \frac{-992}{12} = \frac{254}{3} \\
  x_2 &= x_1 - \frac{f(x_1)}{f'(x_1)} \approx 56.491 \\
  x_3 &\approx 37.765 \\
  x_4 &\approx 25.410 \\
  x_5 &\approx 17.457 \\
  x_6 &\approx 12.732 \\
  x_7 &\approx 10.544 \\
  x_8 &\approx 10.028 \\
  x_0 &\approx 10.000076 \\
\end{align*}

Tarkka ratkaisu on $x^* = 10$. Edellä laskettujen $x_n$:n virheet ovat

\begin{align*}
  e_0 &= |10 - 2| = 8 \\
  e_1 &= |10 - \frac{254}{3}| = \frac{164}{3} \\
  e_2 &\approx |10 - 56.491| = 46.491 \\
  e_3 &\approx 27.765 \\
  e_4 &\approx 15.410 \\
  e_5 &\approx 7.457 \\
  e_6 &\approx 2.732 \\
  e_7 &\approx 0.544 \\
  e_8 &\approx 0.028 \\
  e_9 &\approx 0.000076 \\
\end{align*}

Arvioidaan konvergenssiä kolmen viimeisen virheen perusteella.

\[
  \begin{cases}
    e_8 \approx Ce_7^p \\
    e_9 \approx Ce_8^p \\
  \end{cases}
\]
\[
  \begin{cases}
    0.028 \approx C*0.544^p \\
    0.000076 \approx C*0.028^p \\
  \end{cases}
\]
Näistä saadaan
\begin{align*}
  \frac{0.028^{(p+1)}}{0.544^p} &\approx 0.000076 \\
  \Big(\frac{0.028}{0.544}\Big)^p &\approx \frac{0.000076}{0.028} \\
  0.051471^p &\approx 0.027143 \\
  p &\approx 1.216 \\
\end{align*}

Ratkaistaan tästä $C$:

\begin{align*}
  0.028 &\approx C*0.544^{1.216} \\
  C &\approx 0.058704 \\
\end{align*}

Tällä tarkkuudella näyttää siis, että konvergenssi olisi superlineaarista
melko pienellä virhevakiolla.

\section*{2.}

Rakennetaan yhtälöryhmä matriisimuodossa ja käytetään Matlabin kenoviivaoperaattoria.

Matlab-koodi:

\verbatiminput{w2_2.m}

tulostaa

\verbatiminput{|"octave w2_2.m"}


\section*{3.}

\begin{align*}
  \mathbf{S} &= \mathbf{D} - \mathbf{CA}^{-1} \mathbf{B} \in \mathbb{R}^{m \times m} \\
  \mathbf{y} = \mathbf{Sx} &= (\mathbf{D} - \mathbf{CA}^{-1} \mathbf{B})\mathbf{x} \\
                       &= \mathbf{Dx} - \mathbf{CA}^{-1}\mathbf{Bx} \\
\end{align*}
Tästä vasemmanpuoleinen termi osataan laskea.
Oikeanpuoleinen saadaan ratkaisemalla $\mathbf{z}$ yhtälös\-tä
$\mathbf{Az} = \mathbf{Bx}$ ja laskemalla tulo $\mathbf{Cz}$.


\section*{4.}

\[
  \mathbf{A} = \begin{bmatrix}
    4 & 0 & -1 \\
    0 & 4 & -1 \\
    -1 & -1 & 4 \\
  \end{bmatrix}
\]
$\mathbf{L}$:n diagonaalin alapuoliset alkiot:
\[
  l_{ij} = \frac{1}{l_{jj}}\Big(a_{ij} - \sum_{k=1}^{j-1}l_{ik}l_{jk}\Big),
  \quad i = 1,\dots,n \quad j = 1,\dots, i-1, \quad i \neq j
\]
ja diagonaalialkiot:
\[
  l_{ii} = \Big(a_{ii} - \sum_{k=1}^{i-1}l_{ik}^2\Big)^{\frac{1}{2}},
  \quad i = 1,\dots,n
\]
\begin{align*}
  &l_{11} = 4^{\frac{1}{2}} = 2 \\
  &l_{21} = \frac{1}{2}(0) = 0, \quad l_{22} = (4 - 0^2)^{\frac{1}{2}} = 2 \\
  &l_{31} = \frac{1}{2}(-1) = -\frac{1}{2}, \quad l_{32} = \frac{1}{2}(-1 - 0) = -\frac{1}{2},
  \quad l_{33} = (4 - (\frac{1}{4} + \frac{1}{4}))^{\frac{1}{2}} = \sqrt{\frac{7}{2}} \\
\end{align*}

Siis

\[
  \mathbf{A} = \mathbf{LL}^T =
  \begin{bmatrix}
    2 & 0 & 0 \\
    0 & 2 & 0 \\
    -\frac{1}{2} & -\frac{1}{2} & \sqrt{\frac{7}{2}} \\
  \end{bmatrix}
  \begin{bmatrix}
    2 & 0 & -\frac{1}{2} \\
    0 & 2 & -\frac{1}{2} \\
    0 & 0 & \sqrt{\frac{7}{2}} \\
  \end{bmatrix}.
\]


\section*{5.}

Täytyy osoittaa, että $\mathbf{x}^T (\mathbf{A} + \alpha\mathbf{B})\mathbf{x} > 0$ kaikille
$\mathbf{x} \in \mathbb{R}^n, \mathbf{x} \neq \mathbf{0}$, kun $\alpha \geq 0$,
ja lisäksi että on tapauksia $\alpha < 0$, joissa tämä ei pidä paikkaansa.
\begin{align*}
  &\mathbf{x}^T(\mathbf{A} + \alpha\mathbf{B})\mathbf{x} \\
  &= \mathbf{x}^T(\mathbf{Ax} + \alpha\mathbf{Bx}) \\
  &= \mathbf{x}^T\mathbf{Ax} + \alpha\mathbf{x}^T\mathbf{Bx} \\
\end{align*}
Koska oletettiin, että $\mathbf{A}$ ja $\mathbf{B}$ ovat positiividefiniittejä,
niin $\mathbf{x}^T\mathbf{Ax} > 0$ ja $\mathbf{x}^T\mathbf{Bx} > 0$.
Kun lisäksi $\alpha \geq 0$, niin koko lauseke on $> 0$.
Tämä todistaa ensimmäisen väitteen.

Jos $\alpha < 0$, niin $\alpha\mathbf{x}^T\mathbf{Bx} < 0$. Tällöin jos
$|\alpha\mathbf{x}^T\mathbf{Bx}| > |\mathbf{x}^T\mathbf{Ax}|$, niin koko lauseke on $< 0$.
Tämä todistaa toisen väitteen.


\section*{6.}

Bidiagonaaliset yhtälöryhmät ovat erikoistapauksia ala- ja yläkolmiomatriiseista,
joihin voi soveltaa etenevää ja takenevaa sijoitusta.
Tässä tapauksessa kaavat ovat $\mathbf{Ly} = \mathbf{b}$:lle (alakolmiomatriisi,
päädiagonaalilla 1 ja sivudiagonaalilla $\beta_i$, ete\-nevä sijoitus)
\begin{align*}
  y_1 &= b_1, \\
  y_i &= b_i - y_{i-1}\beta_{i}, \quad i = 2,\dots,n,
\end{align*}
ja $\mathbf{Ux} = \mathbf{y}$:lle (yläkolmiomatriisi, päädiagonaalilla
$\omega_i$ ja sivudiagonaalilla $\alpha_i$, takeneva sijoitus)
\begin{align*}
  x_n &= \frac{y_n}{\omega_n}, \\
  x_i &= \frac{y_i - x_{i+1}\alpha_{i}}{\omega_{i}}, \quad i = n-1,n-2,\dots,1. \\
\end{align*}


\section*{7-8.}

Jos $\mathbf{LUx} = \mathbf{b} \iff \mathbf{Ux} = \mathbf{L}^{-1}\mathbf{b}$, niin
$x$ saadaan ratkaisemalla ensin $\mathbf{y}$ yhtälöstä $\mathbf{Ly} = \mathbf{b}$
ja sitten $\mathbf{x}$ yhtälöstä $\mathbf{Ux} = \mathbf{y}$.

Matlab-koodi:

\verbatiminput{w2_solve_tridiag.m}

ja ratkaisijan testaus:

\verbatiminput{w2_7.m}

tulostaa

\verbatiminput{|"octave w2_7.m"}


\end{document}

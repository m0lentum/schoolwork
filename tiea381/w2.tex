%! TEX program = xelatex

\documentclass{article}
\usepackage[a4paper, margin=3cm]{geometry}
\setlength{\parindent}{0pt}
\setlength{\parskip}{1em}
\usepackage{fontspec}
\setmainfont{Lato}

\usepackage{amsmath,amssymb,amsthm}
%\usepackage{hyperref}
\usepackage{verbatim}
\usepackage{graphicx}
%\usepackage{pgfplots}
%\pgfplotsset{compat=1.16}

\title{TIEA381 Demo 2}
\author{Mikael Myyrä}
\date{\number\day.\number\month.\number\year}

\begin{document}
\maketitle

\section*{1.}

\begin{align*}
  f(x) &= x^3 - 1000 = 0 \\
  f'(x) &= 3x^2 \\
\end{align*}

Lasketaan muutamia $x_n$:n arvoja Newtonin menetelmällä
ja tarkastellaan niiden virhettä.

\begin{align*}
  x_0 &= 2 \\
  x_1 &= x_0 - \frac{f(x_0)}{f'(x_0)} = 2 - \frac{-992}{12} = \frac{254}{3} \\
  x_2 &= x_1 - \frac{f(x_1)}{f'(x_1)} \approx 56.491 \\
  x_3 &\approx 37.765 \\
  x_4 &\approx 25.410 \\
  x_5 &\approx 17.457 \\
  x_6 &\approx 12.732 \\
  x_7 &\approx 10.544 \\
  x_8 &\approx 10.028 \\
  x_0 &\approx 10.000076 \\
\end{align*}

Tarkka ratkaisu on $x^* = 10$. Edellä laskettujen $x_n$:n virheet ovat

\begin{align*}
  e_0 &= |10 - 2| = 8 \\
  e_1 &= |10 - \frac{254}{3}| = \frac{164}{3} \\
  e_2 &\approx |10 - 56.491| = 46.491 \\
  e_3 &\approx 27.765 \\
  e_4 &\approx 15.410 \\
  e_5 &\approx 7.457 \\
  e_6 &\approx 2.732 \\
  e_7 &\approx 0.544 \\
  e_8 &\approx 0.028 \\
  e_9 &\approx 0.000076 \\
\end{align*}

Arvioidaan konvergenssiä kolmen viimeisen virheen perusteella.

\[
  \begin{cases}
    e_8 \approx Ce_7^p \\
    e_9 \approx Ce_8^p \\
  \end{cases}
\]
\[
  \begin{cases}
    0.028 \approx C*0.544^p \\
    0.000076 \approx C*0.028^p \\
  \end{cases}
\]
Näistä saadaan
\begin{align*}
  \frac{0.028^{(p+1)}}{0.544^p} &\approx 0.000076 \\
  \Big(\frac{0.028}{0.544}\Big)^p &\approx \frac{0.000076}{0.028} \\
  0.051471^p &\approx 0.027143 \\
  p &\approx 1.216 \\
\end{align*}

Ratkaistaan tästä $C$:

\begin{align*}
  0.028 &\approx C*0.544^{1.216} \\
  C &\approx 0.058704 \\
\end{align*}

Tällä tarkkuudella näyttää siis, että konvergenssi olisi superlineaarista
melko pienellä virhevakiolla.

\section*{2.}

Rakennetaan yhtälöryhmä matriisimuodossa ja käytetään Matlabin kenoviivaoperaattoria.

Matlab-koodi:

\verbatiminput{w2_2.m}

tulostaa

\verbatiminput{|"octave w2_2.m"}

\end{document}

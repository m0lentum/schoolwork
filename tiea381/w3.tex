%! TEX program = xelatex

\documentclass{article}
\usepackage[a4paper, margin=3cm]{geometry}
\setlength{\parindent}{0pt}
\setlength{\parskip}{1em}
\usepackage{fontspec}
\setmainfont{Lato}

\usepackage{amsmath,amssymb,amsthm}
%\usepackage{hyperref}
%\usepackage{graphicx}
%\usepackage{pgfplots}
%\pgfplotsset{compat=1.16}

\usepackage{verbatim}
\usepackage{listings}
\usepackage{xcolor}
\definecolor{purple}{RGB}{135,20,85}
\definecolor{gray}{RGB}{100,100,100}
\lstset{
  keywordstyle=\color{purple},
  commentstyle=\color{gray},
}

\title{TIEA381 Demo 3}
\author{Mikael Myyrä}
\date{\number\day.\number\month.\number\year}

\begin{document}
\maketitle

\section*{1.}

Koska matriisi-vektori-kertolaskussa on vain yhteen- ja kertolaskua,
niin laskujärjestyksellä ei ole vaikutusta lopputulokseen. Voidaan siis aina
laskea siinä järjestyksessä, missä alkiot on tallennettu.
Nolla-alkioiden vaikutus on nolla, joten niille ei tarvitse tehdä mitään.
Algoritmi pseudo-matlabilla:
\begin{lstlisting}[language=Matlab]
  y = zeros(row_count,1);
  for k = 1 : nz
    y(row(k)) += val(k) * x(col(k));
  end
  y
\end{lstlisting}


\section*{2.}

Lisätään kaikki nollasta poikkeavat alkiot ja niiden koordinaatit taulukkoon.
Pseudo-matlab:
\begin{lstlisting}[language=Matlab]
  % vasen yläkulma erikseen, jotta voidaan käyttää `i-1` silmukassa
  val(1) = 2 + h^2; row(1) = 1; col(1) = 1;
  for i = 2 : n
    % päädiagonaali
    val(end+1) = 2 + h^2; row(end+1) = i; col(end+1) = i;
    % sivudiagonaalit
    val(end+1) = -1; row(end+1) = i-1; col(end+1) = i;
    val(end+1) = -1; row(end+1) = i; col(end+1) = i-1;
  end
  % ylä- ja alarivien "ylimääräiset" -1:t
  val(end+1) = -1; row(end+1) = 1; col(end+1) = n-1;
  val(end+1) = -1; row(end+1) = n; col(end+1) = 2;
\end{lstlisting}


\end{document}

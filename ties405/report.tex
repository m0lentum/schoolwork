%! TEX program = xelatex

\documentclass{article}
\usepackage[a4paper, margin=3cm]{geometry}
\setlength{\parindent}{0pt}
\setlength{\parskip}{1em}
\usepackage{fontspec}
\setmainfont{Lato}

\usepackage{amsmath,amssymb,amsthm}
% \usepackage{algorithmicx} % pseudocode
% \usepackage{algpseudocode} % pseudocode
\usepackage{hyperref} % links
% \usepackage{graphicx} % images
% \usepackage{pgfplots} % plots
% \pgfplotsset{compat=1.16} % plots

% verbatim code
% \usepackage{listings}
% \usepackage{xcolor}
% \definecolor{purple}{RGB}{135,20,85}
% \definecolor{gray}{RGB}{100,100,100}
% \lstset{
%   basicstyle=\ttfamily,
%   keywordstyle=\color{purple},
%   commentstyle=\color{gray},
% }

% matrices with customizable stretch
% as per https://tex.stackexchange.com/questions/14071/how-can-i-increase-the-line-spacing-in-a-matrix
\makeatletter
\renewcommand*\env@matrix[1][\arraystretch]{%
  \edef\arraystretch{#1}%
  \hskip -\arraycolsep
  \let\@ifnextchar\new@ifnextchar
  \array{*\c@MaxMatrixCols c}}
\makeatother

% usually I want images 350pt wide and centered
\newcommand{\img}[1]{
  \begin{center}
    \includegraphics[width=350pt]{#1}
  \end{center}
}

%
% begin document
%

% \usepackage[authordate,noibid,backend=biber]{biblatex-chicago}
% \addbibresource{}


\title{TIES405 Sovellusprojektin korvausraportti}
\author{
Mikael Myyrä \\
s. 16.07.1996\\
mikael.b.myyra@jyu.fi\\
}
\date{\number\day.\number\month.\number\year}

\begin{document}
\maketitle

\section{Yleiset tiedot}

Raportti käsittelee Jyväskylän yliopiston Digipalveluissa toteutettua
\url{https://opinto-opas.jyu.fi} -palvelua.
Projekti alkoi maaliskuussa 2020 ja päättyi vuoden 2020 lopussa. (TODO tarkista git-historiasta milloin loppui ''kokoaikainen'' tekeminen)
Maaliskuusta elokuuhun olin projektissa mukana kokopäiväisesti (n. 36 h/vk) ja
loppuvuoden osa-aikaisesti n. 24 h/vk. Tätä projektia ei ole käytetty minkään
aiemman opintojakson suorittamiseen.

Teknisesti opinto-opas on GatsbyJS-alustalla toteutettu sivusto, joka
generoidaan Sisu-opintotieto\-järjestelmästä saatavan opetussuunnitelmadatan
perusteella. Sivustolla on muutamia interaktiivisia elementtejä,
merkittävimpänä haku, mutta valtaosa ohjelmasta muokkaa dataa Sisun
tieto\-mallista staattisiksi sisältöelementeiksi. Lisäksi sivustolla näytetään
Avoimen yliopiston käsin tuottamaa sisältöä.

\section{Projektin hallinta ja läpivienti}

\subsection{Organisaatio}

Projektia toteuttamassa oli itseni lisäksi kaksi muuta kehittäjää — arkkitehti,
joka vastasi teknologiavalinnoista ja osittain teknisestä toteutuksesta,
ja suunnittelija, joka suunnitteli ja toteutti osan sivuston ulkoasusta.
Lisäksi kehitystiimin vetäjä organisoi osaltaan toimintaa ja suunnittelua.

Tilaajana toimivat kaksi muuta Jyväskylän yliopiston yksikköä,
Koulutuspalvelut ja Avoin yliopisto. Heidän edustajistaan mukana oli kaksi
projektipäällikköä, jotka olivat päävastuussa projektin läpiviennistä, ja
muutamia suunnittelijoita, jotka tekivät vaatimusmäärittelyä yhteistyössä
projektipäälliköiden ja kehittäjien kanssa.

\subsection{Omat tehtäväni}

Itse vastasin projektissa pääosasta ohjelman teknistä toteutusta. Ohjelmoin mm.
sisällön generoinnin Sisun tietomallista ja hakutoiminnon käyttöliittymineen.
Lisäksi suunnittelin itse osan toteuttamistani käyttöliittymistä.

Näiden teknisten tehtävien lisäksi olin mukana suunnittelussa yhdessä tilaajan
edustajien kanssa. Kommentoin suunnitelmia erityisesti tekniikan ja tietomallin
näkökulmasta; onko haluttu tieto saa\-ta\-vil\-la ja mi\-ten. Esittelin myös
toteuttamiani ominaisuuksia suoraan tilaajille ja kävin heidän kans\-saan
palaute\-kes\-kus\-teluja.

\subsection{Prosessi}

TODO tutustu SAFEen

Suunnitteluviikot, tiimipalaverit, tsekkaukset asiakkaan kanssa

\subsection{Viestintä}

Projektiin kuuluva päivittäinen viestintä tapahtui pääasiassa
Flowdock-pikaviestipalvelun kautta. Kehittäjillä ja tilaajilla oli yhteinen
kanava, jolla keskusteltiin niin teknisistä yksityiskohdista kuin vaatimuksista
ja toteutetuista ominaisuuksistakin. Lisäksi asynkronista viestintää tapahtui
Jira-tikettien ja -kommenttien muodossa, joiden avulla organisoitiin
suunnitelmia ja seurattiin niiden toteutumista.

Näiden jatkuvasti avoimien kanavien lisäksi ajoittain pidettiin palavereja
reaaliaikaista, tavoitteellista keskustelua varten. Kehitystiimin kesken
etenemistä seurattiin säännöllisesti joka toinen päivä, ja tilaajan kanssa
keskusteltiin suunnitelmista aina suunnitteluviikkojen aikana ja muulloin
epäsäännöllisesti tarvittaessa. Näissä palavereissa ei käytetty mitään tiettyä
protokollaa, mutta yleensä projektipäälliköt toimivat jonkinlaisina
puheenjohtajina.

\section{Opittua}


\end{document}

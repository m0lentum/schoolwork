%! TEX program = xelatex

\documentclass{article}
\usepackage[a4paper, margin=3cm]{geometry}
\setlength{\parindent}{0pt}
\setlength{\parskip}{1em}
\usepackage{fontspec}
\setmainfont{Lato}

\usepackage{amsmath,amssymb,amsthm}
% \usepackage{algorithmicx} % pseudocode
% \usepackage{algpseudocode} % pseudocode
\usepackage{hyperref} % links
% \usepackage{graphicx} % images
% \usepackage{pgfplots} % plots
% \pgfplotsset{compat=1.16} % plots

% verbatim code
% \usepackage{listings}
% \usepackage{xcolor}
% \definecolor{purple}{RGB}{135,20,85}
% \definecolor{gray}{RGB}{100,100,100}
% \lstset{
%   basicstyle=\ttfamily,
%   keywordstyle=\color{purple},
%   commentstyle=\color{gray},
% }

% matrices with customizable stretch
% as per https://tex.stackexchange.com/questions/14071/how-can-i-increase-the-line-spacing-in-a-matrix
\makeatletter
\renewcommand*\env@matrix[1][\arraystretch]{%
  \edef\arraystretch{#1}%
  \hskip -\arraycolsep
  \let\@ifnextchar\new@ifnextchar
  \array{*\c@MaxMatrixCols c}}
\makeatother

% usually I want images 350pt wide and centered
\newcommand{\img}[1]{
  \begin{center}
    \includegraphics[width=350pt]{#1}
  \end{center}
}

%
% begin document
%

% \usepackage[authordate,noibid,backend=biber]{biblatex-chicago}
% \addbibresource{}


\title{TIES405 Sovellusprojektin korvausraportti}
\author{
Mikael Myyrä \\
s. 16.07.1996\\
mikael.b.myyra@jyu.fi\\
}
\date{\number\day.\number\month.\number\year}

\begin{document}
\maketitle

\section{Yleiset tiedot}

Raportti käsittelee Jyväskylän yliopiston Digipalveluissa toteutettua
\url{https://opinto-opas.jyu.fi} -palvelua.
Projekti alkoi maaliskuussa 2020 ja päättyi vuoden 2020 lopussa. (TODO tarkista git-historiasta milloin loppui ''kokoaikainen'' tekeminen)
Maaliskuusta elokuuhun olin projektissa mukana kokopäiväisesti (n. 36 h/vk) ja
loppuvuoden osa-aikaisesti n. 24 h/vk. Tätä projektia ei ole käytetty minkään
aiemman opintojakson suorittamiseen.

Teknisesti opinto-opas on GatsbyJS-alustalla toteutettu sivusto, joka
generoidaan Sisu-opintotieto\-järjestelmästä saatavan opetussuunnitelmadatan
perusteella. Sivustolla on muutamia interaktiivisia elementtejä,
merkittävimpänä haku, mutta valtaosa ohjelmasta muokkaa dataa Sisun
tieto\-mallista staattisiksi sisältöelementeiksi. Lisäksi sivustolla näytetään
Avoimen yliopiston käsin tuottamaa markkinointisisältöä.

\section{Projektin hallinta ja läpivienti}

\subsection{Organisaatio}

Projektia toteuttamassa oli itseni lisäksi kaksi muuta kehittäjää — arkkitehti,
joka vastasi teknologiavalinnoista ja osittain teknisestä toteutuksesta,
ja suunnittelija, joka suunnitteli ja toteutti osan sivuston ulkoasusta.
Lisäksi kehitystiimin vetäjä organisoi osaltaan toimintaa ja suunnittelua.
Minä olin ainoana projektissa mukana kokopäiväisesti; muilla tiimin jäsenillä
oli muita vastuita ja osallistuivat tähän vain osalla työajastaan.

Tilaajana toimivat kaksi muuta Jyväskylän yliopiston yksikköä,
Koulutuspalvelut ja Avoin yliopisto. Heidän edustajistaan mukana oli kaksi
projektipäällikköä, jotka olivat päävastuussa projektin läpiviennistä, ja
muutamia suunnittelijoita, jotka tekivät vaatimusmäärittelyä yhteistyössä
projektipäälliköiden ja kehittäjien kanssa.

\subsection{Omat tehtäväni}

Itse vastasin projektissa pääosasta ohjelman teknistä toteutusta. Ohjelmoin mm.
sisällön generoinnin Sisun tietomallista ja hakutoiminnon käyttöliittymineen.
Lisäksi suunnittelin itse osan toteuttamistani käyttöliittymistä.

Näiden teknisten tehtävien lisäksi olin mukana suunnittelussa yhdessä tilaajan
edustajien kanssa. Kommentoin suunnitelmia erityisesti tekniikan ja tietomallin
näkökulmasta; onko haluttu tieto saa\-ta\-vil\-la ja mi\-ten. Esittelin myös
toteuttamiani ominaisuuksia suoraan tilaajille ja kävin heidän kans\-saan
palaute\-kes\-kus\-teluja.

\subsection{Prosessi}

Prosessimallina on koko yksikön laajuisesti käytössä SAFe (Scaled Agile
Framework), jonka mu\-kai\-ses\-ti tämäkin projekti suoritettiin. Malli on
suunniteltu suurille organisaatioille, joilla on useita kehitystiimejä,
asiakkaita ja samanaikaisia projekteja.

Tiimitason toiminnassa malli näkyy scrum-tyyppisenä nopeana kehityssyklinä,
jossa suunnitelmia tehdään viikoittain ja niiden etenemistä seurataan lyhyissä
palavereissa useita kertoja viikossa. Kehityksen tuloksia julkaistaan sitä
mukaa kun ne valmistuvat, ja tilaajan kanssa tehdään tiivistä yhteis\-työtä.

Pidemmän aikavälin suunnitelmia tarkastellaan kolmen kuukauden välein
tapahtuvilla suunnittelu\-päivillä. Tällöin kehitystiimit ja ylempi johto
kokoontuvat yhdessä ns. asianomistajien kanssa sopimaan projektien välisestä
priorisoinnista ja tehtävien jakamisesta tiimien kesken. Asianomistajat
vastaavat tilaajien toiveiden tuomisesta mukaan suunnitteluun. Suunnitteluun
kuuluu työmäärien, kapasiteetin ja aikataulun arviointia ja riskien
ennakointia, ja suunnittelun tulokset kerätään yhtei\-selle Kanban-taululle,
''Program Boardille''.  Suunnittelujen välinen kehitysjakso jakautuu kahden
viikon ''iteraatioihin'', joiden vaihteessa suunnitelmien toteutumista
tarkastellaan tarkemmin ja tii\-mien vetä\-jät raportoivat tilanteen ylemmälle
johdolle.

Lisäksi malliin kuuluu käytäntöjä koko organisaation ''portfolion'' hallintaan.
Omassa arjessani tämä puoli mallista ei näy, koska se on ylemmän johdon ja
asianomistajien vastuualuetta. Tähän kuuluu mm. projektien kokoaminen
suurempiin ''epic''-kokonaisuuksiin, niiden välinen priorisointi, budjetointi
ja karkea aikataulutus vuosien mittakaavassa.

\subsection{Viestintä}

Projektiin kuuluva päivittäinen viestintä tapahtui pääasiassa
Flowdock-pikaviestipalvelun kautta. Kehittäjillä ja tilaajilla oli yhteinen
kanava, jolla keskusteltiin niin teknisistä yksityiskohdista kuin vaatimuksista
ja toteutetuista ominaisuuksistakin. Lisäksi asynkronista viestintää tapahtui
Jira-tikettien ja -kommenttien muodossa, joiden avulla organisoitiin
suunnitelmia ja seurattiin niiden toteutumista.

Näiden jatkuvasti avoimien kanavien lisäksi ajoittain pidettiin palavereja
reaaliaikaista, tavoitteellista keskustelua varten. Kehitystiimin kesken
etenemistä seurattiin säännöllisesti joka toinen päivä, ja tilaajan kanssa
keskusteltiin suunnitelmista aina suunnitteluviikkojen aikana ja muulloin
epäsäännöllisesti tarvittaessa. Näissä palavereissa ei käytetty mitään tiettyä
protokollaa, mutta yleensä projektipäälliköt toimivat jonkinlaisina
puheenjohtajina.

Loppukäyttäjien kanssa kehittäjät eivät keskustelleet suoraan, vaan heidän
palautteensa kulkeutui meille tilaajien kautta. Näistä palautteista
keskusteltiin yleensä Flowdock-kanavallamme.

\section{Opittua ja koettua}

Tämä oli ensimmäinen projekti, jossa olen ollut mukana keskustelemassa ja
suunnittelemassa suoraan asiakkaan kanssa. Pitäisinkin tästä saatua
viestintäkokemusta projektin tärkeimpänä uutena oppina. Teknisten asioiden
ilmaiseminen ymmärrettävästi, palautteen pyytäminen ja siihen reagointi ja
luetun ymmärtäminen ovat viestinnällisiä taitoja, joissa kehityin huomattavasti
projektin aikana. Selkeän viestinnän tärkeys korostui projektin suuren
mittakaavan takia, ja mieles\-tä\-ni jokainen osapuoli onnistui siinä hyvin.
Reaaliaikainen yhteys asiakkaaseen oli erittäin olennaista, koska sen ansiosta
kehittäminen pystyi jatkumaan ilman palautteen tai vaatimusten odottelusta
johtuvia keskeytyksiä.

Muilta osin mikään projektissa ei ollut minulle täysin uutta, koska olin tehnyt
samaa työtä jo pari vuotta samaa prosessimallia käyttäen. Tämä oli kuitenkin
pisin ja laajin projekti, johon olin osallistunut, joten suuren kokonaisuuden
hallinta korostui eri tavalla kuin aiemmin. Kuten em. vies\-tinnässä, myös
teknisesti tämä oli suurin siihen mennessä rakentamani kokonaisuus, joten sen
hal\-tuun\-otto vaati ajattelun ja tiedon organisoinnin taitojen kehittymistä.

SAFe-mallin erityisominaisuudet tulevat esille erityisesti silloin, kun
useampia projekteja tehdään samanaikai\-sesti ja ajankäyttöä täytyy jakaa
niiden kesken. Oma aikani oli varattu täysin tähän projektiin, mutta tämä
portfoliotason priorisointi näkyi muiden tiimiläisten osallistumisessa
projektiin. Olin usein ainoana kehittäjänä tekemässä tätä, koska tiimimme
vastuualue on laaja ja tekijöitä vähän. Tästä syystä koodin katselmointi ja
tiedon jakaminen kehittäjien kesken jäi vähäiseksi, ja kenelläkään ei ole niin
kattavaa tuntemusta projektin lähdekoodista kuin voisi olla. Tämä on ikuinen ja
tiedostettu ongelma tiimimme resursseilla, ja sen kanssa joudutaan toistaiseksi
elämään.

SAFen perustason käytännöt — suunnittelupäivät ja lyhytsyklinen ketterä kehitys
— ovat mielestäni toimineet erittäin hyvin pienemmissä projekteissa ja
osoittautuivat toimivaksi myös tässä. Kolmen kuukauden aikaväli on riittävän
lyhyt, että sen voi suunnitella melko tarkasti ja luotettavasti, ja valmiiksi
saatujen tehtävien välitön testaus ja julkaisu mahdollistavat keskeytyksettömän
etenemisen. Tenknisesti oleellista on automatiikka muutosten nopeaan julkaisuun
ja testiympäristö, jossa tilaaja pääsee kokeilemaan uusia ominaisuuksia.
Yhdessä reaaliaikaisen palauteviestinnän kanssa nämä mahdollistavat työn nopean
ja sujuvan etenemisen.

Ehkä vielä jotain mitä meni pieleen?

\end{document}

%! TEX program = xelatex

\documentclass{article}
\usepackage[a4paper, margin=3cm]{geometry}
\setlength{\parindent}{0pt}
\setlength{\parskip}{1em}
\usepackage{fontspec}
\setmainfont{Lato}

\usepackage{amsmath,amssymb,amsthm}
% \usepackage{algorithmicx} % pseudocode
% \usepackage{algpseudocode} % pseudocode
\usepackage{hyperref} % links
% \usepackage{graphicx} % images
% \usepackage{pgfplots} % plots
% \pgfplotsset{compat=1.16} % plots

% verbatim code
% \usepackage{listings}
% \usepackage{xcolor}
% \definecolor{purple}{RGB}{135,20,85}
% \definecolor{gray}{RGB}{100,100,100}
% \lstset{
%   basicstyle=\ttfamily,
%   keywordstyle=\color{purple},
%   commentstyle=\color{gray},
% }

% matrices with customizable stretch
% as per https://tex.stackexchange.com/questions/14071/how-can-i-increase-the-line-spacing-in-a-matrix
\makeatletter
\renewcommand*\env@matrix[1][\arraystretch]{%
  \edef\arraystretch{#1}%
  \hskip -\arraycolsep
  \let\@ifnextchar\new@ifnextchar
  \array{*\c@MaxMatrixCols c}}
\makeatother

% usually I want images 350pt wide and centered
\newcommand{\img}[1]{
  \begin{center}
    \includegraphics[width=350pt]{#1}
  \end{center}
}

%
% begin document
%

% \usepackage[authordate,noibid,backend=biber]{biblatex-chicago}
% \addbibresource{}


\title{TIES405 Sovellusprojektin korvausraportti}
\author{
Mikael Myyrä \\
16.07.1996\\
mikael.b.myyra@jyu.fi\\
}
\date{\number\day.\number\month.\number\year}

\begin{document}
\maketitle

\section*{Yleiset tiedot}

Raportti käsittelee Jyväskylän yliopiston Digipalveluissa toteutettua
\url{https://opinto-opas.jyu.fi} -palvelua.
Projekti alkoi maaliskuussa 2020 ja päättyi vuoden 2020 lopussa. (TODO tarkista git-historiasta milloin loppui ''kokoaikainen'' tekeminen)
Maaliskuusta elokuuhun olin projektissa mukana kokopäiväisesti (n. 36 h/vk) ja
loppuvuoden osa-aikaisesti n. 24 h/vk. Tätä projektia ei ole käytetty minkään
aiemman opintojakson suorittamiseen.

Teknisesti opinto-opas on GatsbyJS-alustalla toteutettu sivusto, joka
generoidaan Sisu-opintotieto\-järjestelmästä saatavan opetussuunnitelmadatan
perusteella. Sivustolla on muutamia interaktiivisia elementtejä,
merkittävimpänä haku, mutta valtaosa ohjelmasta muokkaa dataa Sisun
tieto\-mallista staattisiksi sisältöelementeiksi. Lisäksi sivustolla näytetään
Avoimen yliopiston käsin tuottamaa sisältöä.

\section*{Projektin hallinta ja läpivienti}

\subsection*{Organisaatio}

Minä, Asko, Miro, Riksa, yliopistopalvelut

\subsection*{Prosessi}

TODO tutustu SAFEen

Suunnitteluviikot, tiimipalaverit, tsekkaukset asiakkaan kanssa

\subsection*{Viestintä}

Suullista: teams-palaverit ja suunnitteluviikot, kirjallista: flowdock, jira

\section*{Opittua}




\end{document}

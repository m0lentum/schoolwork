%! TEX program = xelatex

\documentclass{article}
\usepackage[a4paper, margin=3cm]{geometry}
\setlength{\parindent}{0pt}
\setlength{\parskip}{1em}
\usepackage{fontspec}
\setmainfont{Lato}

\usepackage{amsmath,amssymb,amsthm}
% \usepackage{algorithmicx} % pseudocode
% \usepackage{algpseudocode} % pseudocode
\usepackage{hyperref} % links
% \usepackage{graphicx} % images
% \usepackage{pgfplots} % plots
% \pgfplotsset{compat=1.16} % plots

% verbatim code
% \usepackage{listings}
% \usepackage{xcolor}
% \definecolor{purple}{RGB}{135,20,85}
% \definecolor{gray}{RGB}{100,100,100}
% \lstset{
%   basicstyle=\ttfamily,
%   keywordstyle=\color{purple},
%   commentstyle=\color{gray},
% }

% matrices with customizable stretch
% as per https://tex.stackexchange.com/questions/14071/how-can-i-increase-the-line-spacing-in-a-matrix
\makeatletter
\renewcommand*\env@matrix[1][\arraystretch]{%
  \edef\arraystretch{#1}%
  \hskip -\arraycolsep
  \let\@ifnextchar\new@ifnextchar
  \array{*\c@MaxMatrixCols c}}
\makeatother

% usually I want images 350pt wide and centered
\newcommand{\img}[1]{
  \begin{center}
    \includegraphics[width=350pt]{#1}
  \end{center}
}

%
% begin document
%

% \usepackage[authordate,noibid,backend=biber]{biblatex-chicago}
% \addbibresource{}


\title{TIES405 Sovellusprojektin korvausraportti}
\author{
Mikael Myyrä \\
s. 16.07.1996\\
mikael.b.myyra@jyu.fi\\
}
\date{\number\day.\number\month.\number\year}

\begin{document}
\maketitle

\section{Yleiset tiedot}

Raportti käsittelee Jyväskylän yliopiston Digipalveluissa toteutettua
\url{https://opinto-opas.jyu.fi} -palvelua.  Projekti alkoi maaliskuussa 2020
ja päättyi saman vuoden joulukuussa. Tämän jälkeen sivustoa on vielä päivitetty
aika ajoin, mutta joulukuun jälkeen tekeminen ei ole enää ollut yhtäjaksoista
tai projektimuotoista. Maaliskuusta elokuuhun olin projektissa mukana
kokopäiväisesti (n. 36 h/vk) ja loppuvuoden osa-aikaisesti n. 24 h/vk. Tätä
projektia ei ole käytetty minkään aiemman opintojakson suorittamiseen.

Teknisesti opinto-opas on GatsbyJS-alustalla toteutettu sivusto, joka
generoidaan Sisu-opintotieto\-järjestelmästä saatavan opetussuunnitelmadatan
perusteella. Sivustolla on muutamia interaktiivisia elementtejä,
merkittävimpänä haku, mutta valtaosa ohjelmakoodista muokkaa dataa Sisun
tieto\-mallista staattisiksi sisältöelementeiksi. Lisäksi sivustolla näytetään
Avoimen yliopiston käsin tuottamaa markkinointisisältöä. Sisällön luomiseen
käytetään Plone-sisällönhallintajärjestelmää, jolle tehtiin opinto-opasta varten
räätälöity integraatio.

\section{Projektin hallinta ja läpivienti}

\subsection{Organisaatio}

Projektia toteuttamassa oli itseni lisäksi kaksi muuta kehittäjää — arkkitehti,
joka vastasi teknologiavalinnoista ja osittain teknisestä toteutuksesta,
ja suunnittelija, joka suunnitteli ja toteutti osan sivuston ulkoasusta.
Lisäksi kehitystiimin vetäjä organisoi osaltaan toimintaa ja suunnittelua.
Minä olin ainoana projektissa mukana kokopäiväisesti; muilla tiimin jäsenillä
oli muita vastuita ja osallistuivat tähän vain osalla työajastaan.

Tilaajana toimivat kaksi muuta Jyväskylän yliopiston yksikköä,
Koulutuspalvelut ja Avoin yliopisto. Heidän edustajistaan mukana oli kaksi
projektipäällikköä, jotka olivat päävastuussa projektin läpiviennistä, ja
muutamia suunnittelijoita, jotka tekivät vaatimusmäärittelyä yhteistyössä
projektipäälliköiden ja kehittäjien kanssa.

\subsection{Omat tehtäväni}

Itse vastasin projektissa pääosasta ohjelman teknistä toteutusta. Ohjelmoin mm.
sisällön generoinnin Sisun tietomallista ja hakutoiminnon käyttöliittymineen.
Lisäksi suunnittelin itse osan toteuttamistani käyttöliittymistä ja sivuston
rakenteista. Sain melko vapaasti päättää teknisistä yksityiskohdista, kuten
käytettävistä kirjastoista ja työkaluista. Sivuston sisältö ja muut
toiminnalliset vaatimukset sen sijaan olivat oman määräysvaltani ulkopuolella.

Em. teknisten tehtävien lisäksi olin mukana suunnittelemassa ohjelman
ominaisuuksia ja toteutusaikataulua yhdessä tilaajan edustajien kanssa.
Ominaisuuksien osalta kommentoin suunnitelmia erityisesti tekniikan ja
tietomallin näkökulmasta; onko haluttu tieto saa\-ta\-vil\-la ja mi\-ten.
Aikataulun osalta arvioin tehtävien vaatimia työmääriä. Esittelin myös
toteuttamiani ominaisuuksia suoraan tilaajille ja keskustelin niistä heidän
kanssaan.

Tuntimääräisesti suurin osa työstäni oli ohjelmointia. Keskusteluun ja
suunnitteluun osallistumista tein tarpeen mukaan, useimmiten alle tunnin
päivässä. Etenkin projektin alussa viestintää oli vähän, koska tarvittava
tietomalliin perehtyminen, teknisen alustan rakentaminen ja perustoimintojen
kehittäminen oli työläs tehtävä. Tämän jälkeen keskusteluun kului enemmän
aikaa, kun päästiin hio\-maan olemassaolevaa ja kehittämään korkeamman tason
ominaisuuksia.

\subsection{Prosessi}

Prosessimallina on koko Digipalveluiden laajuisesti käytössä SAFe (Scaled Agile
Framework), jonka mu\-kai\-ses\-ti tämäkin projekti suoritettiin. Malli on
suunniteltu suurille organisaatioille, joilla on useita kehitystiimejä,
asiakkaita ja samanaikaisia projekteja.

Tiimitason toiminnassa malli näkyy scrum-tyyppisenä nopeana kehityssyklinä,
jossa tavoitteita ase\-tetaan viikoittain ja niiden toteutumista seurataan
kehitystiimin kesken lyhyissä palavereissa useita kertoja viikossa.
Kehitystiimi ei ole aina sama kuin projektitiimi. Oman kehitystiimini
vastuualueella on useita projekteja, ja kaikista niistä puhutaan näissä
palavereissa. Projektin alussa nämä palaverit pidettiin kasvokkain kehitystiimin
toimistotiloissa, mutta kesällä tapahtuneen etätöihin siirtymisen jälkeen myös
palaverit suoritettiin verkkopuheluina.

Kehityksen tuloksia julkaistaan sitä mukaa kun ne valmistuvat, ja
tilaajan kanssa tehdään tiivistä yhteis\-työtä. Tilaaja on aina mukana
suunnittelussa sekä ominaisuuksien että aikataulujen osalta, ja julkaisut
tehdään vasta tilaajan antaman palautteen ja hyväksynnän jälkeen.

Käytännössä Digipalveluissa hyväksyntä ja testaus eivät aina toteudu täysin
mallin mukaisesti johtuen projektien suuresta määrästä suhteessa kehittäjien ja
asianomistajien määrään, mutta kaikissa ohjel\-mistoprojekteissamme on käytössä
vähintään Continuous Integration -järjestelmä automaattista yksikkö\-testausta
ja julkaisua varten. Opinto-opasprojektissa käytettiin tämän lisäksi julkista
testausympäristöä, johon alustavat muutokset julkaistiin tilaajan
arvioitavaksi. Tilaajan hyväksyttyä muutokset ne julkaistiin sivuston
tuotantoversioon.

Pidemmän aikavälin suunnitelmia tarkastellaan kolmen kuukauden välein
tapahtuvilla suunnittelu\-päivillä. Tällöin kehitystiimit kokoontuvat yhdessä
ns. asianomistajien kanssa sopimaan projektien välisestä priorisoinnista ja
tehtävien jakamisesta tiimien kesken. Scrum-termeissä asianomistaja vastaa
tuoteomistajaa. He vastaavat suunnittelussa huomioonotettavien vaatimuksien
määrittelemisestä. Päätökset konkreettisista tehtävis\-tä tehdään pitkälti
kehittäjien ja asianomistajien kes\-ken; ylempi johto ei tavallisesti ota
kantaa yksityiskohtiin.

Ylemmän johdon tärkein rooli on korkean tason tavoitteiden määrittely useiden
vuosien aikavälillä. Tällaisia tavoitteita ovat olleet esimerkiksi
opiskelijoiden arjen sujuvuuden lisääminen ja jatkuvan oppimisen tukeminen.
Digipalveluiden johtoryhmä sopii tavoitteista, ja niiden perusteella he
päättävät eri projektien välisestä prioriteetti\-jär\-jes\-tyksestä ja
kehitysresurssien kohdentamisesta. Nämä päätökset ohjaavat suunnittelupäivillä
niiden projektien valintaa, joita seuraavalla kehitysjaksolla otetaan
tehtäväksi.

Suunnitteluun kuuluu työmäärien, käytettävissä olevien henkilöresurssien,
tehtävien välisien riippuvuuksien ja riskien arviontia. Näiden perusteella
suunnitellaan karkea toteutusaikataulu seuraavien kolmen kuukauden ajalle,
johon tehtävät sijoitetaan kahden viikon tarkkuudella.  Tulokset kerätään
kaikille näkyvälle yhtei\-selle Kanban-taululle, jota kutsutaan Program
Boardiksi. Suunnittelupäivien välisenä aikana suunnitelmien toteutumista
tarkastellaan tarkemmin kahden viikon välein, jolloin kehitystii\-mien
vetä\-jät raportoivat tilanteen toisilleen ja ylemmälle johdolle.

Lisäksi malliin kuuluu käytäntöjä koko organisaation ''portfolion'' hallintaan.
Tämä osa mallista on oman vastuualueeni ulkopuolella, mutta siihen kuuluu mm.
projektien kokoaminen suurempiin ''epic''-kokonaisuuksiin, niiden välinen
priorisointi, budjetointi ja karkea aikataulutus vuosien mittakaavassa. Lisäksi
malli antaa johtajille rooleja, vies\-tintäkäytäntöjä ja ajattelumalleja.
Toteuttajien kannalta johtajista läheisimpiä ovat asianomistajat; muiden kanssa
harvoin tehdään suoraan yhteistyötä.

\subsubsection{Suunnittelu projektin aikana}

Viimeisillä suunnittelupäivillä ennen projektin aloitusta sovittiin projektiin
kuuluvien tehtävien lisäk\-si tärkeimmät projektiin osallistuvat henkilöt,
heidän roolinsa ja alustavat vaatimukset sekä julkaisu\-aikataulu ohjelmiston
ensimmäiselle versiolle. Kehittäjien tehtävistä sovittiin tässä vaiheessa minun
ja arkkitehdin vastuualueet. Lisäksi tässä vaiheessa sovittiin toimenpiteistä,
jos aikataulutavoitteeseen ei ehdittäisi, ja perustettiin projektikohtainen
Flowdock-viestintäkanava.

Projektin kuluessa suunnittelupäivillä keskityttiin vaatimuksista ja
aikatauluista sopimiseen. Erityisesti näillä suunnittelukerroilla tuotiin
esille uusia tarpeita, joita ei ollut huomioitu edellisissä suunnitelmissa.

\subsection{Viestintä}

Projektiin kuuluva päivittäinen viestintä tapahtui pääasiassa
Flowdock-pikaviestipalvelun kautta. Kehittäjillä ja tilaajilla oli yhteinen
kanava, jolla keskusteltiin niin teknisistä yksityiskohdista kuin vaatimuksista
ja toteutetuista ominaisuuksistakin. Lisäksi asynkronista viestintää tapahtui
Jira-tikettien ja -kommenttien muodossa, joiden avulla organisoitiin
suunnitelmia ja seurattiin niiden toteutumista.

Näiden jatkuvasti avoimien kanavien lisäksi ajoittain pidettiin palavereja
reaaliaikaista, tavoitteellista keskustelua varten. Kehitystiimin kesken
etenemistä seurattiin säännöllisesti joka toinen päivä, ja tilaajan kanssa
keskusteltiin suunnitelmista aina suunnitteluviikkojen aikana ja muulloin
epäsäännöllisesti tarvittaessa. Näissä palavereissa ei käytetty mitään tiettyä
protokollaa, mutta yleensä projektipäälliköt toimivat puheenjohtajan
kaltaisessa asemassa; he esittelivät kokouksen aiheet ja tavoitteet ja
ohjasivat keskustelua eteenpäin.

Loppukäyttäjien kanssa kehittäjät eivät keskustelleet suoraan, vaan heidän
palautteensa kulkeutui meille tilaajien kautta. Näistä palautteista
keskusteltiin yleensä Flowdock-kanavallamme.

Dokumenttimuotoista viestintää tehtiin projektin aikana hyvin vähän. Kirjoitin
ohjelmiston verkkosivuille standardimallisen saavutettavuusselosteen ja
lähdekoodin yhteyteen ohjeet kehitys\-ympä\-ris\-tön käyttöön. Projektin läpiviennin
osalta dokumentteja ei käytetty lainkaan. Asianomistajat kirjasivat vaatimukset
Jiraan suunnittelupäivien yhteydessä, ja tarvittaessa niitä tarkennettiin
kehi\-tyk\-sen aikana palaverien ja Flow\-dock-keskustelujen avulla.
Edistymisen raportointi tapahtui samo\-jen kanavien kautta. Myöskään projektin
päättymisestä ei laadittu erillistä raporttia.

\section{Opittua ja koettua}

Tämä oli ensimmäinen projekti, jossa olen ollut mukana keskustelemassa ja
suunnittelemassa suoraan tilaajan kanssa. Pitäisinkin tästä saatua
viestintäkokemusta projektin tärkeimpänä uutena oppina. Teknisten asioiden
ilmaiseminen ymmärrettävästi, palautteen pyytäminen ja siihen reagointi ja
luetun ymmärtäminen ovat viestinnällisiä taitoja, joissa kehityin huomattavasti
projektin aikana. Selkeän viestinnän tärkeys korostui aikaisempia kokemuksiani
suuremmassa projektiryhmässä, ja mieles\-tä\-ni jokainen osapuoli onnistui
siinä hyvin. Vaatimukset olivat riittävän hyvin määriteltyjä, että lisätietoja
ei tarvinnut kysyä jatkuvasti, mutta jos lisätiedolle oli tarvetta, sitä sai
asianomistajilta helposti. Omat selitykseni ja kysymykseni ymmärrettiin aina
nopeasti, joten uskon muotoilleeni sanottavani selkeästi ja kohdeyleisölle
sopivasti.

Yhteys tilaajaan oli Flowdockin kautta periaatteessa reaaliaikainen, mutta sitä
käytettiin maltillisesti ja enimmäkseen kehittäjien aloitteesta. Pyysin tämän
kanavan kautta asianomistajilta tarkennuksia vaatimuksiin ja hyväksyntää
valmiille päivityksille. Koin tällaisen viestikanavan hyödylliseksi, koska sen
ansiosta kehittäminen ei koskaan keskeytynyt pitkäksi aikaa epäselvien
vaatimusten tai hyväksynnän odottelun takia.

Muilta osin mikään projektissa ei ollut minulle täysin uutta, koska olin tehnyt
samaa työtä jo pari vuotta samaa prosessimallia käyttäen. Tämä oli kuitenkin
pisin ja laajin projekti, johon olin osallistunut, joten suuren kokonaisuuden
hallinta korostui eri tavalla kuin aiemmin. Kuten em. vies\-tinnässä, myös
teknisesti tämä oli suurin siihen mennessä rakentamani kokonaisuus, joten sen
hal\-tuun\-otto vaati ajattelun ja tiedon organisoinnin taitojen kehittymistä.

SAFe-mallin erityisominaisuudet tulevat esille erityisesti silloin, kun
useampia projekteja tehdään samanaikai\-sesti ja ajankäyttöä täytyy jakaa
niiden kesken. Oma aikani oli varattu täysin tähän projektiin, mutta tämä
portfoliotason priorisointi näkyi muiden tiimiläisten osallistumisessa. Olin
usein ainoana kehittäjänä tekemässä tätä, koska tiimimme vastuualue on laaja ja
tekijöitä vähän. Tästä syystä koodin katselmointi ja tiedon jakaminen
kehittäjien kesken jäi vähäiseksi, ja kenelläkään ei ole niin kattavaa
tuntemusta projektin lähdekoodista kuin voisi olla. Tiedon jakamista olisi
ehdottomasti tarpeen parantaa tulevissa projekteissa laadun ja ylläpitokyvyn
varmistamiseksi.

SAFen perustason käytännöt — suunnittelupäivät ja lyhytsyklinen ketterä kehitys
— ovat mielestäni toimineet erittäin hyvin pienemmissä projekteissa ja
osoittautuivat toimivaksi myös tässä. Kolmen kuukauden aikaväli on riittävän
lyhyt, että sen voi suunnitella melko tarkasti ja luotettavasti, ja jatkuva
toteutumisen seuraaminen auttaa pitämään suunnitelmista kiinni. Valmiik\-si
saatujen tehtävien pikainen hyväksymistestaus ja julkaisu mahdollistavat
keskeytyksettömän etenemisen, ja automaattinen yksikkötestaus estää selkeitä
virheitä pääsemästä julkaisuihin. Huomasin, että teknisesti tässä on oleellista
automatiikka muutosten nopeaan julkai\-suun ja testausympäristö, jossa tilaaja
pääsee kokeilemaan uusia ominaisuuksia. Yhdessä tiiviin viestinnän ja
aktiivisen tilaajan kanssa nämä tekevät työn etenemisestä sujuvaa.

Yksi asia, joka jäi projektissa lähes kokonaan puuttumaan, on järjestelmätestaus
loppukäyttäjien kanssa ja heiltä palautteen kerääminen. Meillä ei ollut mitään
virallista palautekanavaa, eikä käy\-tettävyystestausta tmv. järjestetty. Päätökset
tehtiin siis lähinnä tilaajien ja kehittäjien oman harkinnan mukaan. Olen
henkilökohtaisen verkostoni kautta kuullut positiivista palautetta opettajilta,
mutta otos on pieni ja sattumanvarainen, eikä sen perusteella voi tehdä
johtopäätöksiä. Samoin saavutettavuuden kehitys ja saavutettavuusselosteen
laatiminen jäivät minun tehtäväkseni selai\-men testaustyökalujen avulla.
Parempiin tuloksiin olisi varmasti päästy oikeiden käyttäjien avulla.

\section{Lähteet}

Scaled Agile, Inc. (2021). "SAFe 5 for Lean Enterprises." Viimeksi muokattu 22.7.2021
\\ \url{https://www.scaledagileframework.com/safe-for-lean-enterprises/}.

\end{document}

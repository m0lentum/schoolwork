%! TEX program = xelatex

\documentclass[11pt]{article}
\usepackage[a4paper, margin=3cm]{geometry}
\setlength{\parindent}{0pt}
\setlength{\parskip}{1em}
\usepackage{fontspec}
\setmainfont{Lato}

\usepackage{amsmath,amssymb,amsthm}
% \usepackage{algorithmicx} % pseudocode
% \usepackage{algpseudocode} % pseudocode
\usepackage{hyperref} % links
% \usepackage{graphicx} % images
% \usepackage{pgfplots} % plots
% \pgfplotsset{compat=1.16} % plots

% verbatim code
% \usepackage{listings}
% \usepackage{xcolor}
% \definecolor{purple}{RGB}{135,20,85}
% \definecolor{gray}{RGB}{100,100,100}
% \lstset{
%   basicstyle=\ttfamily,
%   keywordstyle=\color{purple},
%   commentstyle=\color{gray},
% }

% matrices with customizable stretch
% as per https://tex.stackexchange.com/questions/14071/how-can-i-increase-the-line-spacing-in-a-matrix
\makeatletter
\renewcommand*\env@matrix[1][\arraystretch]{%
  \edef\arraystretch{#1}%
  \hskip -\arraycolsep
  \let\@ifnextchar\new@ifnextchar
  \array{*\c@MaxMatrixCols c}}
\makeatother

% usually I want images 350pt wide and centered
\newcommand{\img}[1]{
  \begin{center}
    \includegraphics[width=350pt]{#1}
  \end{center}
}

%
% begin document
%

% \usepackage[authordate,noibid,backend=biber]{biblatex-chicago}
% \addbibresource{}

\begin{document}
\begin{center}
\vspace{2cm}
\LARGE
TIES405 Sovellusprojektin korvausraportti

\vspace{10pt}
\large
\number\day.\number\month.\number\year

\vspace{10pt}
Mikael Myyrä \\
s. 16.7.1996\\
\texttt{mikael.b.myyra@jyu.fi}\\
\end{center}
\vspace{10pt}

\section{Projektin aikajänne ja työmäärä}

Raportti käsittelee Jyväskylän yliopiston Digipalveluissa toteutettua
opinto-opassovellusta, joka on julkaistu osoitteessa
\url{https://opinto-opas.jyu.fi}.  Projekti alkoi maaliskuussa 2020 ja päättyi
saman vuoden joulukuussa. Tämän jälkeen sivustoa on vielä päivitetty aika
ajoin, mutta joulukuun jälkeen tekeminen ei ole enää ollut yhtäjaksoista tai
projektimuotoista. Maaliskuusta elokuuhun olin projektissa mukana
kokopäiväisesti (n. 36 h/vk) ja loppuvuoden osa-aikaisesti n. 24 h/vk. Tätä
projektia ei ole käytetty minkään aiemman opintojakson suorittamiseen.

\section{Kehitetty ohjelmisto}

Opinto-opas on GatsbyJS-alustalla toteutettu sivusto, joka
generoidaan Sisu-opintotieto\-järjestel\-mästä saatavan opetussuunnitelmadatan
perusteella. Sivustolla on muutamia interaktiivisia elementtejä
(merkittävimpänä haku), mutta valtaosa ohjelmakoodista muokkaa dataa Sisun
tieto\-mallis\-ta staattisiksi sisältöelementeiksi. Lisäksi sivustolla näytetään
Avoi\-men yliopiston käsin laatimaa mark\-kinointisisältöä. Sisällön luomiseen
käytetään Plone-sisällön\-hallintajärjestelmää, jolle tehtiin opinto-opasta varten
räätälöity integraatio.

\section{Projektiorganisaatio}

Ohjelmistoa kehittämässä oli itseni lisäksi kaksi muuta kehittäjää. Heistä arkkitehti
vastasi tärkeimmistä teknologiavalinnoista ja osittain teknisestä toteutuksesta,
ja suunnittelija suunnitteli ja toteutti osan sivuston ulkoasusta.
Lisäksi kehitystiimin vetäjä organisoi toi\-mintaa ja suunnittelua.
Minä olin ainoana projektissa mukana kokopäiväisesti. Muilla tiimin jäsenillä
oli muita vastuita, ja he osallistuivat tähän vain osalla työajastaan.

Tilaajina toimivat Jyväskylän yliopiston yksiköistä Koulutuspalvelut ja Avoin
yliopisto. Heidän edustajistaan projektiin osallistui kaksi projektipäällikköä
ja muutamia suunnittelijoita. Projektipäälli\-köt olivat päävastuussa projektin
läpiviennistä, ja suunnittelijat tekivät vaatimusmäärittelyä yhteistyös\-sä
projektipäälliköiden ja kehittäjien kanssa.

\section{Omat tehtäväni}

Itse vastasin projektissa pääosasta ohjelman teknistä toteutusta. Ohjelmoin mm.
sisällön generoinnin Sisun tietomallista ja hakutoiminnon käyttöliittymineen.
Lisäksi suunnittelin itse osan toteuttamistani käyttöliittymistä ja sivuston
rakenteista. Sain melko vapaasti päättää teknisistä yksityiskohdista, kuten
käytettävistä kirjastoista ja työkaluista. Sivuston sisältö ja muut
toiminnalliset vaatimukset sen sijaan olivat oman määräysvaltani ulko\-puolella.

Em. teknisten tehtävien lisäksi olin mukana suunnittelemassa ohjelman
ominaisuuksia ja toteutusaikataulua yhdessä tilaajien edustajien kanssa.
Ominaisuuksien osalta kommentoin suunnitelmia erityisesti tekniikan ja
tietomallin näkökulmasta. Esimerkiksi minua tar\-vittiin usein kertomaan, onko tietty tieto
saatavilla Sisussa ja millaisten liitosten kautta se löytyy. Aikataulun osalta
arvioin tehtävien vaatimia työmääriä. Esittelin myös toteuttamiani
ominaisuuksia tilaajille ja keskustelin niistä heidän kanssaan.

Tuntimääräisesti suurin osa työstäni oli ohjelmointia. Keskusteluun ja
suunnitteluun osallistumista tein tarpeen mukaan, useimmiten alle tunnin
päivässä. Etenkin projektin alussa viestintää oli vähän, koska tarvittava
tietomalliin perehtyminen, teknisen alustan rakentaminen ja perustoimintojen
kehittäminen oli työläs tehtävä. Tämän jälkeen keskusteluun kului enemmän
aikaa, kun päästiin hio\-maan olemassaolevaa ja kehittämään korkeamman tason
ominaisuuksia.

\section{Kehitysprosessi}

Prosessimallina on koko Digipalveluiden laajuisesti käytössä SAFe (Scaled Agile
Framework), jonka mu\-kai\-ses\-ti tätäkin projektia hallittiin. Malli on
suunniteltu suurille organisaatioille, joilla on useita kehitystiimejä,
asiakkaita ja samanaikaisia projekteja.

Tiimitason toiminnassa malli näkyy Scrum-tyyppisenä nopeana kehityssyklinä,
jossa tavoitteita ase\-tetaan viikoittain ja niiden toteutumista seurataan
kehitystiimin kesken lyhyissä palavereissa useita kertoja viikossa.
Kehitystiimi ei ole aina sama kuin projektitiimi. Oman kehitystiimini
vastuualueella on useita projekteja, ja kaikista niistä puhutaan tiimin
palave\-reis\-sa. Opinto-opasprojektin alussa palaverit pidettiin kasvokkain kehitystiimin
toimisto\-tiloissa, mutta kesällä tapahtuneen etätöihin siirtymisen jälkeen ne
suoritettiin verkko\-puheluina.

Kehityksen tuloksia julkistetaan sitä mukaa kun ne valmistuvat, ja
tilaajien kanssa tehdään tiivistä yhteis\-työtä. Tilaajat ovat aina mukana
suunnittelussa sekä ominaisuuksien että aika\-tau\-lujen osalta, ja julkistukset
tehdään vasta heidän antamansa palautteen ja hyväksynnän jälkeen.

Käytännössä Digipalveluissa hyväksyntä ja testaus eivät aina toteudu täysin
mallin mukai\-sesti johtuen projektien suuresta määrästä suhteessa kehittäjien ja
asianomistajien mää\-rään, mutta kaikissa ohjel\-mistoprojekteissamme on käytössä
vähintään Continuous Integration -järjestelmä automaattista yksikkö\-testausta
ja julkistusta varten. Opinto-opas\-pro\-jek\-tissa käytettiin tämän lisäksi kaikille
avointa testausympäristöä, johon alustavat muutokset vie\-tiin tilaajien
arvioitavaksi ennen julkistamista sivuston tuotantoversioon.

Pidemmän aikavälin suunnitelmia tarkastellaan kolmen kuukauden välein
järjestettävillä suunnit\-te\-lu\-päivillä. Tällöin kehitystiimit kokoontuvat
yhdessä ns. asianomistajien kanssa sopimaan tehtävien aikataulutuksesta ja
jakamisesta tiimien kesken. Scrum:n rooleista asi\-an\-omistaja vastaa
tuoteomistajaa. He määrittelevät ja valitsevat suunnittelussa huomioon\-otettavat
vaatimukset. Konkreettisista tehtävis\-tä päätetään pit\-käl\-ti
kehittäjien ja asiano\-mistajien kes\-ken, koska he tuntevat tarvittavat
yksityis\-koh\-dat muita malliin kuuluvia ryhmiä paremmin.

Ylemmän johdon tärkein rooli on korkean tason tavoitteiden määrittely useiden
vuosien aikavälillä. Tällaisia tavoitteita ovat olleet esimerkiksi
opiskelijoiden arjen sujuvuuden lisää\-minen ja jatkuvan oppimisen tukeminen.
Digipalveluiden johtoryhmä sopii tavoitteista, ja niiden perusteella he
päättävät eri projektien välisestä prioriteetti\-jär\-jes\-tyksestä sekä
kehitysresurssien kohdentamisesta. Nämä päätökset ohjaavat suunnittelupäivillä
projektien valintaa, joita seuraavalla kehitysjaksolla otetaan
tehtäväksi.

Suunnitteluun kuuluu työmäärien, käytettävissä olevien henkilöresurssien,
tehtävien väli\-sien riippuvuuksien ja riskien arviointia. Näiden perusteella
suunnitellaan karkea toteutus\-aika\-taulu seuraavien kolmen kuukauden ajalle,
johon tehtävät sijoitetaan kahden viikon tarkkuudella.  Tulokset kerätään
kaikille näkyvälle yhtei\-selle Kanban-taululle (ns. Program Board).
Suunnittelupäivien välisenä aikana suunnitelmien toteutumista
tarkastellaan tar\-kemmin kahden viikon välein, jolloin kehitystii\-mien
vetä\-jät raportoivat tilanteen suullisesti toisilleen ja ylemmälle johdolle.

Viimeisillä suunnittelupäivillä ennen opinto-opasprojektin aloitusta sovittiin projektiin
kuuluvien tehtävien lisäk\-si tärkeimmät projektiin osallistuvat henkilöt,
heidän roolinsa ja alustavat vaatimukset sekä julkistamis\-aikataulu ohjelmiston
ensimmäiselle versiolle. Kehittäjien tehtävistä sovittiin tässä vaiheessa minun
ja arkkitehdin vastuualueet. Lisäksi sovittiin toimenpiteistä,
jos aikataulutavoitteeseen ei ehdittäisi, ja perustettiin projektikohtainen
Flowdock-viestintäkanava.

Suuri osa suunnittelutyöstä ja käytäntöjen sopimisesta tehtiin projektin alussa, joten
projektin edetessä suunnittelupäivillä keskityttiin aikataulutukseen.
Tarvittaessa tällöin tuotiin esille uusia tarpeita, joita ei ollut huomioitu
alkuperäisissä suun\-nitelmissa.

Suunnittelun lisäksi malliin kuuluu käytäntöjä koko organisaation projektisalkun hallintaan.
Tämä osa mallista on oman vastuualueeni ulkopuolella, mutta siihen kuuluu mm.
projektien kokoaminen suurempiin ''epic''-kokonaisuuksiin, niiden välinen
priorisointi, budjetointi ja karkea aikataulutus vuosien mittakaavassa. Lisäksi
malli antaa johtajille rooleja, vies\-tintäkäytäntöjä ja ajattelumalleja.
Kehittäjät tekevät tavallisesti suoraa yhteistyötä vain asianomistajien kanssa.
Muiden johtajien rooli on arjessa etäisempi.

\section{Viestintä}

Projektiin kuuluva päivittäinen viestintä tapahtui pääasiassa
Flowdock-pikaviestipalvelun kautta. Kehittäjillä ja tilaajilla oli yhteinen
kanava, jolla keskusteltiin sekä teknisistä yksityiskohdista että vaatimuksista
ja toteutetuista ominaisuuksista. Lisäksi asynkronista vies\-tintää tapahtui
Jira-tikettien ja -kommenttien muodossa, joiden avulla suun\-nitelmia orga\-ni\-soitiin
ja niiden toteutumista seurattiin.

Em. jatkuvasti avoimien kanavien lisäksi ajoittain pidettiin palavereja
reaaliaikaista, tavoitteellista keskustelua varten. Kehitystiimin kesken
etenemistä seurattiin säännöllisesti joka toinen päivä, ja tilaajien kanssa
keskusteltiin suunnitelmista aina suunnitteluviikkojen ai\-ka\-na ja muulloin
epäsäännöllisesti tarvittaessa. Näissä palavereissa ei käytetty mitään tiettyä
protokollaa, mutta yleensä projektipäälliköt toimivat puheenjohtajan
kaltaisessa asemassa, sillä he esittelivät kokouksen aiheet ja tavoitteet sekä
ohjasivat keskustelua eteenpäin.

Loppukäyttäjien kanssa kehittäjät eivät keskustelleet suoraan, vaan heidän
palautteensa kulkeutui meille tilaajien edustajien kautta. Näistä palautteista
keskusteltiin yleensä Flow\-dock-kanavallamme.

Dokumenttimuotoista viestintää tehtiin projektin aikana hyvin vähän. Kirjoitin
ohjelmiston verkkosivuille standardimallisen saavutettavuusselosteen ja
lähdekoodin yhteyteen ohjeet kehitys\-ympä\-ris\-tön käyttöön. Projektin läpiviennin
osalta dokumentteja ei käytetty lain\-kaan. Asianomistajat kirjasivat vaatimukset
Jiraan suunnittelupäivien yhteydessä, ja tarvit\-taes\-sa niitä tarkennettiin
kehi\-tyk\-sen aikana palaverien ja Flow\-dock-keskustelujen avulla.
Edistymisen raportointi tapahtui samo\-jen kanavien kautta. Myöskään projektin
päättymi\-ses\-tä ei laadittu erillistä raporttia.

\section{Opittua ja koettua}

Tämä oli ensimmäinen projekti, jossa olen ollut mukana keskustelemassa ja
suunnittelemassa suoraan tilaajien kanssa. Pitäisinkin saatua viestintäkokemusta
projektin tärkeimpänä uutena oppina. Teknisten asioiden ilmaiseminen
ymmärrettävästi, palautteen pyytä\-minen ja siihen reagointi sekä luetun
ymmärtäminen ovat viestinnällisiä taitoja, joissa kehityin huomattavasti
projektin aikana. Sel\-keän viestinnän tärkeys korostui aikaisempia
kokemuksiani suuremmassa projektiryhmässä, ja mie\-les\-tä\-ni jokainen
osapuoli onnistui sii\-nä hyvin. Vaatimukset olivat riittävän tarkasti määriteltyjä,
että minun ei tarvinnut pyytää tarkennuksia jatkuvasti, mutta jos lisätiedolle
oli tarvetta, sitä sai asianomistajilta helposti. Omat selitykseni ja
kysymykseni ymmärrettiin yleensä nopeasti, joten uskon muotoilleeni sa\-not\-tavani
selkeästi ja kohdeyleisölle sopivasti.

Yhteys tilaajiin oli Flowdockin kautta periaatteessa reaaliaikainen, mutta sitä
käytettiin maltillisesti ja enimmäkseen kehittäjien aloitteesta. Sain
riittävästi työrauhaa keskittyäkseni työhön. Koin tällaisen viestikanavan
hyödylliseksi, koska sen nopeuden ansiosta kehittäminen ei koskaan keskeytynyt
pitkäksi aikaa epäselvien vaatimusten tai hyväksynnän odottelun takia.

Muilta osin mikään projektissa ei ollut minulle täysin uutta, koska olin tehnyt
samaa työtä jo pari vuotta samaa prosessimallia käyttäen. Tämä oli kuitenkin
toistaiseksi pisin ja laajin projektini, joten suuren kokonaisuuden
hallinta korostui eri tavalla kuin aiemmin. Tähän tarvittiin
ajattelun ja tiedon organisoinnin taitojen kehittymistä.

SAFe-mallin erityisominaisuudet tulevat esille erityisesti silloin, kun
useampia projekteja läpiviedään samanaikai\-sesti ja ajankäyttöä täytyy jakaa
niiden kesken. Oma aikani oli varat\-tu täysin tähän projektiin, mutta
projektisalkkutason priorisointi näkyi muiden tiimiläisten osallistumisessa. Olin
usein ainoana kehittäjänä toteuttamassa opinto-opasta, koska tii\-mim\-me vastuualue on laaja ja
tekijöitä vähän. Tästä syystä koodin katselmointi ja tiedon jakaminen
kehittäjien kesken jäi vähäiseksi, ja kenelläkään ei ole täysin kattavaa
projektin lähdekoodin tuntemusta. Tiedon jakamista olisi
ehdottomasti tarpeen parantaa tulevissa projekteissa laadun ja ylläpitokyvyn
varmistamiseksi.

SAFen käytännöt, erityisesti suunnittelupäivät ja lyhytsyklinen ketterä kehitys,
ovat mieles\-täni toi\-mineet erittäin hyvin pienemmissä projekteissa ja
osoittautuivat toimivaksi myös tässä. Kolmen kuukauden aikaväli on riittävän
lyhyt, että sen voi suunnitella melko tarkasti ja luotettavasti. Lisäksi jatkuva
toteutumisen seuraaminen auttaa pitämään suunnitelmista kiinni. Valmiik\-si
saatu\-jen omi\-naisuuksien pi\-kai\-nen hyväksymistestaus ja julkistaminen mahdollistavat
keskeytyksettömän ete\-nemisen, ja automaattinen yksikkötestaus huomauttaa selkeistä
virheistä nopeasti. Huomasin, että teknisesti tässä on oleellista
automatiikka muutosten nopeaan julkistamiseen ja testausympäristö, jossa tilaajat
pää\-sevät kokeilemaan uusia ominaisuuksia. Yhdessä tiiviin viestinnän ja
aktiivisen tilaa\-jan kanssa nämä teke\-vät työn etenemisestä sujuvaa.

Projektissa jäi lähes kokonaan puuttumaan järjestelmätestaus
loppukäyttäjien kanssa ja heidän pa\-lautteensa huomioiminen. Meillä ei ollut
virallista palautekanavaa, eikä käy\-tettä\-vyystestaus\-ta tmv. järjestetty. Päätökset
tehtiin siis lähinnä tilaajien ja kehittäjien oman harkinnan mukaan. Olen
henkilökohtaisen verkostoni kautta kuullut positiivista palautetta opettajilta,
mutta otos on pieni ja sattumanvarainen, eikä sen perusteella voi tehdä
joh\-topäätöksiä. Samoin saavutettavuuden kehitys ja saavutettavuusselosteen
laatiminen jäi\-vät minun tehtäväkseni selai\-men testaustyökalujen avulla.
Parempiin tuloksiin olisi varmasti päästy oikeiden käyttäjien avulla.

\section{Lähteet}

Scaled Agile, Inc. (2021). "SAFe 5 for Lean Enterprises." Viimeksi muokattu 22.7.2021
\\ \url{https://www.scaledagileframework.com/safe-for-lean-enterprises/}.

\end{document}

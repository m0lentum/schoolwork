%! TEX program = xelatex

\documentclass{article}
\usepackage[a4paper, margin=3cm]{geometry}
\setlength{\parindent}{0pt}
\setlength{\parskip}{1em}
\usepackage{fontspec}
\setmainfont{Lato}

\usepackage{amsmath,amssymb,amsthm}
%\usepackage{hyperref}
\usepackage{verbatim}
%\usepackage{graphicx}
%\usepackage{pgfplots}
%\pgfplotsset{compat=1.16}

\title{}
\author{Mikael Myyrä}
\date{}

\begin{document}

\section*{1.}

\begin{gather*}
  \frac{\partial u(x,t)}{\partial t} - \frac{\partial^2 u(x,t)}{\partial x^2} = 0, \quad x \in (1, 2) \\
  u(1,t) = 0, \quad u(2,t) = 0, \\
  u(x,0) = 8(x-1)(2-x) \\
\end{gather*}

\subsection*{(a)}

Tämä on lämpöyhtälö. Laplacen yhtälö ja aaltoyhtälö ovat paikkaderivaatan
suhteen saman näköisiä, mutta Laplacen yhtälössä ei ole aikaderivaattaa ja
aaltoyhtälössä on ajan toinen derivaatta.

\subsection*{(b)}

Diskretoidaan ensin paikan suhteen laskentapisteissä $u_i, i = 1,\dots,N$
keskeisdifferenssillä:
\[
  \frac{\partial u}{\partial t} - \frac{u_{i-1} - 2u_i + u_{i+1}}{h^2} = 0
\]
Matriisimuodossa
\[
  \frac{\partial u}{\partial t} + \frac{1}{h^2}
  \begin{bmatrix}
    2 & -1 \\
    -1 & 2 & -1 \\
       & \vdots & \vdots & \vdots \\
       & & -1 & 2 & -1 \\
       & & & -1 & 2 \\
  \end{bmatrix}
  \mathbf{u} = 0
\]
\[
  \frac{\partial u}{\partial t} + \mathbf{A}\mathbf{u} = 0.
\]
Crank—Nicolson-menetelmällä aikadiskretoinniksi saadaan
\[
  \frac{\mathbf{u}^{(k+1)} - \mathbf{u}^{(k)}}{\Delta t}
  + \frac{1}{2}\mathbf{A}\mathbf{u}^{(k+1)}
  + \frac{1}{2}\mathbf{A}\mathbf{u}^{(k)} = 0,
\]
josta ratkaistaan $\mathbf{u}^{(k+1)}$:
\begin{align*}
  \mathbf{u}^{(k+1)} - \mathbf{u}^{(k)}
  + \frac{1}{2}\Delta t \mathbf{A}\mathbf{u}^{(k+1)}
  &= -\frac{1}{2}\Delta t \mathbf{A}\mathbf{u}^{(k)} \\
  (\mathbf{I} + \frac{1}{2}\Delta t \mathbf{A})\mathbf{u}^{(k+1)}
  &= (\mathbf{I} - \frac{1}{2}\Delta t \mathbf{A})\mathbf{u}^{(k)} \\
  \mathbf{u}^{(k+1)} &= (\mathbf{I} + \frac{1}{2}\Delta t \mathbf{A})^{-1}
  (\mathbf{I} - \frac{1}{2}\Delta t \mathbf{A})\mathbf{u}^{(k)}
\end{align*}
Homogeeniset Dirichlet-reunaehdot toteutuvat tässä suoraan eivätkä vaadi
erityistä huomiointia.

\subsection*{(c)}

Ratkaisija Matlabilla:

\verbatiminput{exam_1_solve.m}

\subsection*{(d)}

Hilavälillä $\frac{1}{10}$ paikka-askelia on 10, aika-askel on $\frac{1}{200}$
ja askelia tarvitaan 100, jotta päästään ajan hetkeen $\frac{1}{2}$.
Syöttämällä nämä parametrit ratkaisijaan:

\begin{verbatim}
exam_1_solve(10, 1/200, 100)
\end{verbatim}

saadaan tulokseksi
\[
  \begin{bmatrix}
   0.0047718 \\
   0.0090765 \\
   0.0124927 \\
   0.0146861 \\
   0.0154419 \\
   0.0146861 \\
   0.0124927 \\
   0.0090765 \\
   0.0047718 \\
  \end{bmatrix}.
\]

\newpage
\section*{2.}



\end{document}

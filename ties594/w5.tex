%! TEX program = xelatex

\documentclass{article}
\usepackage[a4paper, margin=3cm]{geometry}
\setlength{\parindent}{0pt}
\setlength{\parskip}{1em}
\usepackage{fontspec}
\setmainfont{Lato}

\usepackage{amsmath,amssymb,amsthm}
%\usepackage{verbatim}
%\usepackage{graphicx}
%\usepackage{pgfplots}
%\pgfplotsset{compat=1.16}

\title{TIES594 Harjoitus 5}
\author{Mikael Myyrä}
\date{}

\begin{document}

\section*{1.}

Tarkastellaan ominaisarvojen avulla.
Paikkadiskretointioperaattoriksi saadaan tridiagonaalimatriisi
\[
  \mathbf{A} = \frac{a}{2h}
  \begin{bmatrix}
    0 & 1 \\
    -1 & 0 & 1 \\
        & \ddots & \ddots & \ddots \\
        & & -1 & 0 & 1 \\
        & & & -1 & 0 \\
  \end{bmatrix}.
\]
Tämän kanssa vektorimuodossa ilmaistuna tehtävä on
\[
  \frac{\mathbf{u}^{(k+1)} - \mathbf{u}^{(k)}}{\Delta t} + \mathbf{A}\mathbf{u}^{(k)} = 0,
\]
mistä saadaan
\[
  \mathbf{u}^{(k+1)} = \mathbf{u}^{(k)} - \Delta t \mathbf{A}\mathbf{u}^{(k)}
  = (\mathbf{I} - \Delta t \mathbf{A})\mathbf{u}^{(k)}
\]
ja
\[
  \mathbf{u}^{(n)} = (\mathbf{I} - \Delta t \mathbf{A})^n \mathbf{u}^{(0)}.
\]
Jotta menetelmä olisi stabiili, täytyy matriisin
$\mathbf{B} = \mathbf{I} - \Delta t \mathbf{A}$ suurimman ominaisarvon oltava
itseisarvoltaan $\leq 1$.
\[
  \mathbf{B} =
  \begin{bmatrix}
    1 & -\frac{a\Delta t}{2h} \\
    \frac{a\Delta t}{2h} & 1 & -\frac{a\Delta t}{2h} \\
                         & \ddots & \ddots & \ddots \\
                         & & \frac{a\Delta t}{2h} & 1 & -\frac{a\Delta t}{2h} \\
                         & & & \frac{a\Delta t}{2h} & 1 \\
  \end{bmatrix}
\]
ja sen ominaisarvot ovat
\begin{align*}
  \lambda_k &= 1 + 2\sqrt{\frac{a\Delta t}{2h}(-\frac{a\Delta t}{2h})}
        \cos(\frac{k\pi}{n + 1}), \quad k = 1, \dots, n \\
            &= 1 + i\frac{a\Delta t}{h}\cos(\frac{k\pi}{n + 1}) \\
\end{align*}
En ole juuri opiskellut kompleksilukuja enkä täysin hahmota, miten
imaginäärinen osa tässä vaikuttaa, mutta olennaista lienee vain itseisarvo
\[
  |\lambda_k| = \sqrt{1^2 + (\frac{a\Delta t}{h}\cos(\frac{k\pi}{n + 1}))^2},
\]
mikä on $k$:n läpikäymällä välillä aina suurempaa kuin 1.
Tämän perusteella menetelmä olisi aina epästabiili.

\newpage
\section*{2.}

\[
  \frac{u_j^{(k+1)} - u_j^{(k)}}{\Delta t} + \frac{u_j^{(k+1)} - u_{j-1}^{(k+1)}}{h}
  - \frac{u_{j-1}^{(k+1)} - 2u_j^{(k+1)} + u_{j+1}^{(k+1)}}{h^2} = 0
\]
Sijoitetaan $u_j^{(k)} = \xi^k e^{ij\varphi}$:
\begin{align*}
  \frac{(\xi^{k+1} - \xi^k)e^{ij\varphi}}{\Delta t}
  + \frac{\xi^{k+1}(e^{ij\varphi} - e^{i(j-1)\varphi})}{h}
  - \frac{\xi^{k+1}(e^{i(j-1)\varphi} - 2e^{ij\varphi} + e^{i(j+1)\varphi})}{h^2} &= 0 \\
  \frac{\xi^{k+1}e^{ij\varphi}(1 - \frac{1}{\xi})}{\Delta t}
  + \frac{\xi^{k+1}e^{ij\varphi}(1 - e^{-i\varphi})}{h}
  - \frac{\xi^{k+1}e^{ij\varphi}(e^{-i\varphi} - 2 + e^{i\varphi})}{h^2} &= 0 \\
  \frac{1 - \frac{1}{\xi}}{\Delta t} + \frac{1 - e^{-i\varphi}}{h}
  - \frac{e^{-i\varphi} - 2 + e^{i\varphi}}{h^2} &= 0 \\
  \frac{1}{\Delta t} - \frac{1}{\xi\Delta t}
  + \frac{1}{h}(1 - (\cos\varphi - i\sin\varphi))
  - \frac{1}{h^2}(\cos\varphi - i\sin\varphi - 2 + \cos\varphi + i\sin\varphi) &= 0 \\
  \frac{1}{\Delta t} - \frac{1}{\xi\Delta t}
  + \frac{1}{h} - \frac{\cos\varphi - i\sin\varphi}{h}
  - \frac{2\cos\varphi - 2}{h^2} &= 0 \\
\end{align*}
\begin{align*}
  \frac{1}{\xi\Delta t} &= \frac{1}{\Delta t} + \frac{1}{h}
  - \frac{\cos\varphi - i\sin\varphi}{h} - \frac{2\cos\varphi - 2}{h^2} \\
  \frac{1}{\xi} &= 1 + \frac{\Delta t}{h} (1 - (1 + \frac{2}{h})\cos\varphi + i\sin\varphi + \frac{2}{h}) \\
  \xi &= \frac{1}{1 + \frac{\Delta t}{h} (1 - (1 + \frac{2}{h})\cos\varphi + i\sin\varphi + \frac{2}{h})} \\
      &= \frac{1}{1 + \frac{\Delta t}{h} ((1 + \frac{2}{h})(1 - \cos\varphi) + i\sin\varphi)}
\end{align*}
Tässä $|\xi| \leq 1$, kun
\[
  \frac{\Delta t}{h} ((1 + \frac{2}{h})(1 - \cos\varphi) + i\sin\varphi) \geq 0
\]
tai
\[
  \frac{\Delta t}{h} ((1 + \frac{2}{h})(1 - \cos\varphi) + i\sin\varphi) \leq -2.
\]
Koska $1 + \frac{2}{h} > 1$ ja $0 \leq 1 - \cos\varphi \leq 2$, niin termi
$(1 + \frac{2}{h})(1 - \cos\varphi)$ on aina $\geq 0$.
Myös kerroin $\frac{\Delta t}{h} > 0$, koska $\Delta t, h > 0$.
Siis
\[
  \frac{\Delta t}{h}((1 + \frac{2}{h})(1 - \cos\varphi)) \geq 0,
\]
ja nimittäjän reaaliosa on aina $\geq 1$. Niinpä myös koko nimittäjän
itseisarvo on aina $\geq 1$, mistä seuraa $|\xi| \leq 1$.
Siis tämä diskretointi on ehdoitta stabiili.

\newpage
\section*{4.}

Annettujen oletusten kanssa ($\epsilon = \mu = 1$ ja $f = 0$) yhtälöt ovat
\[
  \frac{\partial E_1}{\partial t} + \frac{\partial H_2}{\partial x_3} = 0
\]
ja
\[
  \frac{\partial H_2}{\partial t} + \frac{\partial E_1}{\partial x_3} = 0.
\]
Lasketaan $E_1$ pisteissä $i = 1,\dots,N-1$ ajan hetkillä $k = 0,1,\dots$ ja $H_2$
pisteissä $i + \frac{1}{2}$ ajan hetkillä $k + \frac{1}{2}$, hilavälillä $h$ ja
aika-askelella $\Delta t$.  Diskretoidaan jokainen laskettava arvo
keskeisdifferenssillä. Merkitään $E = E_1$, $H = H_2$ ja $x = x_3$
paikkadiskretoinnin alaindeksien selkeyttämiseksi. Paikkadiskretoinnista
saadaan
\begin{align*}
  \frac{\partial E(x_i,t)}{\partial t} + \frac{H_{i+\frac{1}{2}} - H_{i-\frac{1}{2}}}{h} &= 0 \\
  \frac{\partial H(x_{i+\frac{1}{2}},t)}{\partial t} + \frac{E_{i+1} - E_i}{h} &= 0 \\
\end{align*}
ja aikadiskretoinnista
\begin{align*}
  \frac{E_i^{(k+1)} - E_i^{(k)}}{\Delta t}
  + \frac{H_{i+\frac{1}{2}}^{(k+\frac{1}{2})} - H_{i-\frac{1}{2}}^{(k+\frac{1}{2})}}{h} &= 0 \\
  \frac{H_{i+\frac{1}{2}}^{(k+\frac{1}{2})} - H_{i+\frac{1}{2}}^{(k-\frac{1}{2})}}{\Delta t}
  + \frac{E_{i+1}^{(k)} - E_i^{(k)}}{h} &= 0. \\
\end{align*}
Ratkaistaan $E_i^{(k+1)}$ ja $H_{i+\frac{1}{2}}^{(k+\frac{1}{2}}$:
\begin{align*}
  E_i^{(k+1)} &= E_i^{(k)} - \frac{\Delta t}{h}(
    H_{i+\frac{1}{2}}^{(k+\frac{1}{2})} - H_{i-\frac{1}{2}}^{(k+\frac{1}{2})}) \\
  H_{i+\frac{1}{2}}^{(k+\frac{1}{2})} &= H_{i+\frac{1}{2}}^{(k-\frac{1}{2})}
  - \frac{\Delta t}{h} (E_{i+1}^{(k)} - E_i^{(k)}) \\
\end{align*}

\end{document}

%! TEX program = xelatex

\documentclass{article}
\usepackage[a4paper, margin=3cm]{geometry}
\setlength{\parindent}{0pt}
\setlength{\parskip}{1em}
\usepackage{fontspec}
\setmainfont{Lato}

\usepackage{amsmath,amssymb,amsthm}

\title{}
\author{Mikael Myyrä}
\date{}

\begin{document}

\pagestyle{empty}

\section*{3.}

Poissonin yhtälö
\[
  \frac{\partial^2 u}{\partial x^2} = f(x) \text{ alueessa } \Omega = (0, 1)
\]
Sijoitetaan tähän
\begin{align*}
  u(x) &= -\frac{1}{6}x^3 + \frac{1}{12}(x^4 + x) \\
  \frac{\partial u}{\partial x} &= \frac{1}{3}x^3 - \frac{1}{2}x^2 + \frac{1}{12} \\
  \frac{\partial^2 u}{\partial x^2} &= x^2 - x. \\
\end{align*}

Saadaan $x^2 - x = f(x)$. Siis lähdetermi $f = x^2 - x$.

Reunaehdot saadaan suoraan laskemalla.
\begin{align*}
  u(0) &= -\frac{1}{6}*0^3 + \frac{1}{12}(0^4 + 0) = 0 \\
  u(1) &= -\frac{1}{6}*1^3 + \frac{1}{12}(1^4 + 1) = 0 \\
\end{align*}

\newpage

\section*{5.}

\subsection*{(a)}

Laskenta-alue on $x$:lle $x \in [0, 1]$
ja $t$:lle $t \in [0, 2000dt = 0.2]$.

\subsection*{(b)}

En ole ihan varma, tulkitsenko kysymystä oikein, mutta
nähdäkseni ainoa yhtälö, joka on voimassa jokai\-sessa alueen
sisäpisteessä on ratkaistava ODY,
\[
  \frac{\partial u}{\partial t} = \alpha \frac{\partial^2 u}{\partial x^2}.
\]

\subsection*{(c)}

Koodissa $u$:n muutos ajan suhteen on
\texttt{dt*A*U(:,indt-1)}, missä \texttt{A} on
$u$:n toisen paikkaderivaatan diskreetin approksimaation kerroinmatriisi
ja \texttt{U(:,indt-1)} on $u$:n arvo edellisellä aika-askelella.
Termistä kerrointa tässä ei näy, joten sen arvo on 1.

\subsection*{(d)}

Reunaehtona on $u(0, t) = u(1, t) = 0$ ja alkuehtona $u(x, 0) = y(x) = x(1-x)$.

\subsection*{(e)}

Ratkaistavassa ODYssä ei ole lähdetermiä (/ lähdetermi on vakiosti nolla).
Sen sijaan reunaehtojen toteutumiseksi täytyy $u$:n arvo asettaa
nollaksi alueen reunoilla, mitä voisi ajatella lähdetermin lisäämisenä.
Ehkä myös sitä, että alkuehto poikkeaa nollasta, voisi ajatella
lähdeterminä, joka on vaikuttanut ennen ajan hetkeä $t = 0$,
mutta tämä ei tunnu oikein täsmälliseltä.
Vastaan siis, että lähdetermi on alueen reunalla.

\end{document}

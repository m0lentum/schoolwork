%! TEX program = xelatex

\documentclass{article}
\usepackage[a4paper, margin=3cm]{geometry}
\setlength{\parindent}{0pt}
\setlength{\parskip}{1em}
\usepackage{titlesec}
\titlespacing\section{0pt}{16pt plus 4pt minus 2pt}{2pt plus 2pt minus 2pt}
\usepackage{fontspec}
\setmainfont{Lato}
\usepackage[finnish]{babel}

\usepackage{amsmath,amssymb,amsthm}
% \usepackage{algorithmicx} % pseudocode
% \usepackage{algpseudocode} % pseudocode
% \usepackage{hyperref} % links
% \usepackage{graphicx} % images
% \usepackage{pgfplots} % plots
% \pgfplotsset{compat=1.16} % plots

% verbatim code
% \usepackage{listings}
% \usepackage{xcolor}
% \definecolor{purple}{RGB}{135,20,85}
% \definecolor{gray}{RGB}{100,100,100}
% \lstset{
%   basicstyle=\ttfamily,
%   keywordstyle=\color{purple},
%   commentstyle=\color{gray},
% }

% matrices with customizable stretch
% as per https://tex.stackexchange.com/questions/14071/how-can-i-increase-the-line-spacing-in-a-matrix
\makeatletter
\renewcommand*\env@matrix[1][\arraystretch]{%
  \edef\arraystretch{#1}%
  \hskip -\arraycolsep
  \let\@ifnextchar\new@ifnextchar
  \array{*\c@MaxMatrixCols c}}
\makeatother

% usually I want images 350pt wide and centered
\newcommand{\img}[1]{
  \begin{center}
    \includegraphics[width=350pt]{#1}
  \end{center}
}

%
% begin document
%

% \usepackage[authordate,noibid,backend=biber]{biblatex-chicago}
% \addbibresource{}


\title{TIES501: Aiemmin tehtyihin pro gradu -töihin tutustuminen}
\author{Mikael Myyrä}
\date{\number\day.\number\month.\number\year}

\begin{document}
\maketitle

\section*{Ajoneuvojen internetin haasteet ja ratkaisut (Käyhty 2022)}

Heti aluksi huomioni kiinnittyi kirjoituksen kieliasuun, joka on hyvin
arkikielistä (esim. ''Ensin tuli 4G-verkot'') ja paikoin vaikeaselkoista.
Määritelmäluku alkaa käsitteellä ''ajoneuvojen kommunikointi kaikkeen'', jota
ei selitetä lainkaan. Samoin ITS:n käsite jää epäselväksi; kirjoittaja
luettelee vain käyttötapauksia, mutta ei määrittele itse käsitettä. Myöhemmin
tekstissä esiintyy paljon käsitteitä, joita ei määritellä lainkaan. Työn
rakenne on johdonmukainen, mutta muulta osin raportointi on mielestäni heikkoa.

Myös lähteiden käyttö on puutteellista. Useiden sivujen mittaisissa osuuksissa
on paikoin vain yksi lähde, mikä on erityisesti kirjallisuuskatsauksen
kontekstissa arveluttavaa. Lähdeviitteet on pääosin koottu kappaleiden
loppuihin, jolloin on mahdotonta tietää, mikä väite tulee mistäkin lähteestä.

Lähteiden hakuprosessia tai ylipäätään tutkimusmenetelmää ei ole dokumentoitu
mitenkään. Käytetty menetelmä ei ole systemaattinen kirjallisuuskatsaus, joten
hakujen tarkkaa toistettavuutta ei vaadita, mutta perustelujen puuttuminen
kokonaan ei ole hyväksyttävää. Sekä tehtävä-, perehtymi\-nen- että
toteutus-kriteerien suhteen työ on heikko.

Antaisin tästä arvosanan 1.

\section*{An automatic method for assessing spiking of tibial
tubercles associated with knee osteoarthritis (Patron 2022)}

Teksti on tyyliltään ja kieliasultaan laadukasta sekä lähteiden käyttö
monipuolista ja kattavaa. Aikaisemman kirjallisuuden läpikäynti osoittaa
tuntemusta aihealueesta, ja menetelmien käsittely on erinomaisen
perusteellista. Myös tulosten analysointi on huolellista, ja menetelmien
rajoitukset on otettu huomioon rehellisesti. Kaikilta osin kyseessä on erittäin
huoliteltu työ, josta on vaikea löytää virheitä. Asiavirheitä en tosin pysty
tunnistamaan, koska en tunne aihealuetta.

Antaisin tästä arvosanan 5.

\section*{Time tracking in software maintenance service (Kaihlavirta 2022)}

Työ alkaa vahvasti ydinkäsitteen määrittelyllä ja tutkimustarpeen perustelulla
aikaisempaan tutkimukseen peilaten. Käsitteiden määrittely on kattavaa ja
monipuoliseen lähdekirjallisuuteen perustuvaa. Kokonaisuus on rakenteeltaan
johdonmukainen ja tutkimusmenetelmä hyvin perusteltu ja dokumentoitu.

En ole aiemmin nähnyt graduissa dokumentoitavan kirjallisuushakua yhtä
systemaattisesti kuin tässä (lukuunottamatta graduja, joiden pääasiallinen
tutkimusmenetelmä on systemaattinen kirjallisuuskartoitus). Tämä osoittaa
vahvaa asiaan vihkiytymistä ja viestii todennäköisesti hyvästä suoriutumisesta
arvosteluperusteiden prosessi-kategoriassa (oma-aloitteisuus ja vastuullisuus).

Menetelmän soveltamisen puutteet on myös otettu huomioon. Haastattelujen
toteutus ainoastaan jälkikäteen ja iteraatioiden puute vähentävät tulosten
uskottavuutta, mutta pitkällistä iterointia ei voi gradun laajuudessa
odottaakaan. Tuloksia ei voida yleistää koeryhmän ulkopuolelle, mutta
työstä saatiin uutta tietoa, joka voi olla hyödyksi tulevan tutkimuksen
taustatietona.

Antaisin tästä arvosanan 4.

\section*{Nuxt.js-kirjastolla toteutetun verkkosivuston nopeus ja
hakukone\-optimointi (Vitikainen 2022)}

Pieniä tyylivirheitä lukuunottamatta tekstissä ei ole merkittäviä puutteita.
Käsitteet on määritelty tyydyttävällä tarkkuudella ja käyttäen kohtuullista
määrää lähteitä. Tutkimuksen tarve on perusteltu ja tutkimusmenetelmään
perehdytty, joskin hieman suppean lähdekirjallisuuden avulla.
Suunnittelutieteessä tärkeät mittaus ja iterointi toteutuvat hyvin nopeuden
osalta, mutta hakukoneoptimointia on vaikeampi mitata, ja sen tarkastelussa on
tyydytty toteamaan, että se on mahdollista.

Antaisin tästä arvosanan 3.

\end{document}

%! TEX program = xelatex

\documentclass[utf8]{gradu3}

\usepackage{csquotes}
\usepackage{amsmath,amssymb,amsthm}
\usepackage{biblatex}
\usepackage[bookmarksopen,bookmarksnumbered,linktocpage]{hyperref}

\addbibresource{sources.bib}

\title{Pro gradu -tutkielman tutkimussuunnitelma}
\translatedtitle{Master's thesis research plan}
\author{Mikael Myyrä}
\contactinformation{\texttt{mikael.b.myyra@jyu.fi}}
\supervisor{Sanna Mönkölä, Tuomo Rossi}
\studyline{Teknis-matemaattinen mallintaminen}
\tiivistelma{-}
\abstract{-}
\avainsanat{-}
\keywords{-}

\begin{document}

\maketitle
\mainmatter
\sloppypar

\chapter{Johdanto}

- tutkimuksen idea, miksi se on tärkeä sekä tieteellisesti että käytännössä

Tutkimuksen aiheena on yleistetyn differenssimenetelmän
(engl. Discrete Exterior Calculus, DEC) ja kontrollimenetelmän
soveltaminen kaksiulotteiseen akustiikkasimulatioon.
Näitä menetelmiä on tutkittu suhteellisen vähän,
ja niillä on kiinnostavia ominaisuuksia
ratkaisun tarkkuuden ja tehokkuuden näkökulmista.

Tarkemmilla simulointimenetelmillä pystytään tekemään luotettavampia arvioita
todellisten systeemien käyttäytymisestä.
Nopeammilla menetelmillä puolestaan nopeutetaan iterointia
esim. rakenteiden suunnittelutyössä.

Tämä on uusi sovelluskohde DEC:n ja kontrollimenetelmän yhdistelmälle,
joten tutkimuksesta saadaan uutta tietoa menetelmien käyttäytymisestä eri tehtävissä.

\chapter{Kirjallisuuskartoitus}

- menetelmä: voidaan esitellä hakusanat, hakuprosessi, hakukoneet ja tietokannat (hyödyllistä tietoa kirjata itselle muistiin, ei välttämättä tule lopulliseen graduun)

Kirjallisuuden haku alkoi tutkimusryhmän asiantuntijoilta saaduista avainlähteistä
ja niiden lähdeviitteistä.
Lisäksi tehtiin täydentävä haku Google Scholar -hakukoneella
avainsanoilla ''Discrete Exterior Calculus acoustics'' ja ''exact controllability''.

- tulokset (tiivistetty kuvaus löytyneistä artikkeleista)

Menetelmä perustuu differentiaalimuotoihin, joiden teoriaa käsitellään
oppikirjoissa \parencite{abraham_manifolds_2012} ja \parencite{lee_introduction_2012}
sekä vähemmän formaalisti artikkelissa \parencite{blair_perot_differential_2014}
ja kurssimateriaalissa \parencite{crane_digital_2013}.
Muita DEC:ssä tarvittavia matemaattisia käsitteitä
käsitellään mm. artikkeleissa \parencite{lohi_whitney_2021},
\parencite{engquist_absorbing_1977}, \parencite{mur_finite-element_1993}
ja \parencite{plessix_review_2006} sekä luentomonisteessa \parencite{gillette_notes_2009}.

Kontrollimenetelmän matemaattista teoriaa käsitellään artikkeleissa
\parencite{glowinski_ensuring_1992}, \parencite{lasiecka_exact_1989},
\parencite{lasiecka_exact_1989}, \parencite{lions_exact_1988}
ja \parencite{bristeau_controllability_1998}.

Aikaisempia DEC-menetelmän sovelluksia ovat mm.
\parencite{bossavit_discretization_2005}, \parencite{desbrun_discrete_2005},
\parencite{hirani_numerical_2015}, \parencite{vandekerckhove_mimetic_2014},
\parencite{nitschke_discrete_2017},
\parencite{rabina_numerical_2014} ja \parencite{monkola_discrete_2022}.

- kerrotaan, mitä tutkittavasta aiheesta tiedetään entuudestaan
(metodilähteitä mainittava)

\chapter{Tutkimusaihe/tutkimuskysymys}

Ensisijainen tutkimuskysymys on,
miten DEC ja kontrollimenetelmä soveltuvat akustiikkatehtävän ratkaisuun.

Tätä tarkastellaan ainakin seuraavien konkreettisempien kysymysten kautta:
Miten tarkasti DEC toistaa tunnetun analyyttisen ratkaisun,
ja miten tämä riippuu käytetystä laskentaverkosta?
Kuinka nopeasti kontrollimenetelmä konvergoi
verrattuna ajasta riippuvaan asymptoottiseen ratkaisijaan?

\chapter{Tutkimusstrategia/metodi ja sen valintaperusteet}

Tuntemistani tutkimusmenetelmistä tämä muistuttaa eniten suunnittelutiedettä,
mutta sekään ei kuvaa tätä kovin tarkasti.
TODO: tutustu laskennallisen fysiikan / numeriikan menetelmäkirjallisuuteen
jos sellaista on

\chapter{Aineiston keruu}

Tutkimukseen ei tarvita ihmisosallistujia.
Aineisto saadaan toteutettujen simulointikokeiden tuloksista,
ja sen käsittely on koekohtaista.
Pääsääntöisesti kokeen tulokset voidaan ilmoittaa sellaisenaan tutkimustuloksina.

\chapter{Tulokset}

TODO: Ei kai tässä vaiheessa vielä tarvitse olla tuloksia ja johtopäätöksiä?
Vai mitä näillä tässä haetaan?

\chapter{Johtopäätökset}

(ks. edellinen luku)

\chapter{Lähdeluettelo}

\printbibliography

\end{document}

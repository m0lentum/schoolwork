%! TEX program = xelatex

\documentclass[utf8,english]{gradu3}

\usepackage{graphicx}
\usepackage{csquotes}
\usepackage{amsmath,amssymb,amsthm}
\usepackage{biblatex}
\usepackage[bookmarksopen,bookmarksnumbered,linktocpage]{hyperref}

\addbibresource{sources.bib}

\title{Discrete Exterior Calculus and Exact Controllability for Time-Harmonic Acoustic Wave Simulation}
\translatedtitle{Diskreetti ulkoinen laskenta ja kontrollimenetelmä aikaharmonisessa akustiikkasimulaatiossa}
\author{Mikael Myyrä}
\contactinformation{\texttt{mikael.b.myyra@jyu.fi}}
\supervisor{Sanna Mönkölä, Tuomo Rossi, Jonni Lohi}
\studyline{Teknis-matemaattinen mallintaminen (TODO: translate)}
\tiivistelma{-}
\abstract{-}
\avainsanat{
diskreetti ulkoinen laskenta,
kontrollimenetelmä,
akustiikka,
simulointi
}
\keywords{
discrete exterior calculus,
exact controllability,
acoustics,
simulation
}

\begin{document}

\maketitle
\mainmatter
\sloppypar


\chapter{Introduction}



\chapter{Model}

The phenomenon we wish to simulate is the scattering
of acoustic waves by a sound-soft obstacle (TODO: what does sound-soft mean).
We consider acoustic waves described by the differential equation

\begin{equation}
  \frac{\partial^2 \phi}{\partial t^2} - c^2 \nabla^2\phi = 0
\end{equation}

where $\phi$ is a velocity potential and $c$ is the speed of wave propagation.
This equation is equivalent to the first-order system

\begin{equation}\label{wave1stOrd}
  \begin{cases}
    \frac{\partial p}{\partial t} - c^2\nabla \cdot \mathbf{v} = 0 \\
    \frac{\partial \mathbf{v}}{\partial t} - \nabla p = 0 \\
  \end{cases}
\end{equation}

where $p = \frac{\partial \phi}{\partial t}$ is acoustic pressure
and $\mathbf{v} = \nabla \phi$ is the velocity
of particles perturbed by the wave.

The system \eqref{wave1stOrd} applies within the spatial domain $\Omega$
and the time domain $[0, T]$.
We model the scattering obstacle by excluding its interior from $\Omega$
and applying an incoming wave $\phi_{inc}$ on its boundary $\Gamma_{sca}$
as a Dirichlet boundary condition

\[
  \begin{cases}
    p = \frac{\partial \phi_{inc}}{\partial t} \\
    \mathbf{v} = \nabla \phi_{inc} \\
  \end{cases}
  \quad \text{in } \Gamma_{sca} \times [0, T].
\]

On the outer boundary of the domain $\Gamma_{ext}$
we model the uninterrupted propagation of waves past the boundary
with a first-order absorbing Engqvist-Majda boundary condition
\parencite{engquist_absorbing_1977}

\[
  \frac{1}{c}\frac{\partial\phi}{\partial t} + \mathbf{n} \cdot \nabla\phi = 0
  \quad \text{in } \Gamma_{ext} \times [0, T].
\]


TODO: maybe the boundary conditions would be better explained
in the experiments chapter?
Possibly remove this chapter altogether,
mention scattering in the introduction
and list where each part of the model is explained
(wave equation in DEC, boundary conditions to model scattering in experiments)



\chapter{Discrete Exterior Calculus}

The Discrete Exterior Calculus (DEC),
originally developed by Desbrun et al. \parencite*{desbrun_discrete_2005},
is a discretization method for differential equations
based on the exterior calculus of differential forms.
This chapter gives a brief introduction to the mathematical concepts
underlying the method, followed by a description of the DEC itself
and a DEC discretization of our acoustics equations.
The reader is assumed to be familiar with linear algebra and vector calculus.

\section{Differential forms and the continuous exterior calculus}

This section is meant as a somewhat informal and intuitive introduction
to differential forms, summarizing relevant sections of
\parencite{blair_perot_differential_2014} and \parencite{crane_digital_2013}.
For a more formal and comprehensive approach to the topic,
see a textbook on differential geometry such as \parencite{lee_introduction_2012}
or \parencite{abraham_manifolds_2012}.

\subsection{Differential forms}

Notationally, a differential form looks like an \textit{integrand},
e.g. the $dx$ in $\int dx$ or the $5\,dx\,dy$ in $\iint 5\,dx\,dy$.
Indeed, the things we integrate are all technically differential forms.
However, this seems rather abstract.
Are these objects meaningful outside of integration,
and if so, what do they do?
This section looks at differential forms as functions,
taking inspiration from \parencite{crane_digital_2013}.

Differential forms are in many ways analogous to vectors.
To illustrate the similarities and differences,
consider the dot product operation
for two vectors in linear algebra.
Expressed in matrix form, the dot product between vectors
$\mathbf{a} = (a_1, \dots, a_n)^T$ and $\mathbf{b} = (b_1, \dots, b_n)^T$
is

\[
  \mathbf{a} \cdot \mathbf{b} = \mathbf{a}^T \mathbf{b}
  = \begin{bmatrix}
    a_1 & \dots & a_n
  \end{bmatrix}
  \begin{bmatrix}
    b_1 \\ \vdots \\ b_n
  \end{bmatrix}.
\]

The dot product measures the length of $\mathbf{b}$ in the direction of $\mathbf{a}$,
and to do this, $\mathbf{a}$ is transposed into a \textit{row vector}.
This row vector can be thought of as a function
that takes a vector and measures it along $\mathbf{a}$,

\[
  \alpha(\mathbf{b}) = \mathbf{a}^T \mathbf{b}.
\]

In the language of differential forms,
a function like this is called a \textit{1-form},
because it takes a single vector as its parameter
and computes its length along one dimension.
We can also measure higher-dimensional quantities.
A 2-form is a two-dimensional object that takes two vectors,
which define a parallelogram in a plane,
and computes the area of this parallelogram
projected onto the 2-form.
Similarly, 3-forms measure the volumes of parallelepipeds
defined by three vectors, and so on for higher dimensions.
Additionally, we can define 0-forms
as objects that take zero vectors as input and return a measurement,
making them equivalent to scalar fields.

% - basis vectors dx, dy (, dz? is it easier to explain in 3d?)
% - wedge product
% - integration
% - differentiation
% - Hodge star
% - why this is useful: coordinate-free formulas (simplicity/conciseness),
%   curved space, higher dimensional space
% - mention formal definition as antisymmetric tensors?

\section{Computation mesh}

The discrete spatial structure at the root of the DEC is the \textit{simplicial complex}:
essentially a mesh (in the computer graphics sense)
consisting of triangles in two dimensions,
tetrahedra in three dimensions, or \textit{k-simplices} in $k$ dimensions.
We will focus on the two-dimensional case here for illustration.
This section is based on \parencite{desbrun_discrete_2006}.

\subsection{Simplices}

A \textit{k-simplex} is the simplest type of $k$-dimensional element
in a polyhedral mesh, defined as
the convex hull of $k + 1$ geometrically distinct points.
In concrete terms, a 0-simplex is a single point (also called a \textit{vertex}),
a 1-simplex is a line segment,
a 2-simplex is a triangle,
a 3-simplex is a tetrahedron, and so on.

Every simplex with dimension $k > 0$
has a \textit{boundary} consisting of $(k-1)$-simplices.
For instance, every triangle (2-simplex) is bounded by three line segments (1-simplices),
and every line segment is bounded by two vertices.
Each $(k-1)$-simplex on the boundary of a $k$-simplex $\sigma_k$
is called a \textit{$(k-1)$-face} of $\sigma_k$.

\subsection{Simplicial complex}

A simplicial complex $\mathcal{K}$ is a set of simplices satisfying the rules

\begin{itemize}
  \item every face of every simplex in $\mathcal{K}$ is also in $\mathcal{K}$
  \item any two simplices in $\mathcal{K}$ either do not intersect at all
    or share an entire face.
\end{itemize}

In the case of a 2D triangle mesh, this means
that every edge and vertex of every triangle is also part of the complex,
and there are no overlapping triangles
or edges/vertices that aren't part of a triangle's boundary.

A similar complex could be defined without requiring
all its elements to be simplicial.
For instance, a rectilinear grid is a common such structure.
A complex like this is called a \textit{cell complex}
and its $k$-dimensional elements \textit{$k$-cells}.
A simplex is a special case of a cell,
and the concepts of boundary and face apply to all cells.
We will use a simplicial complex in this thesis,
however, non-simplicial cells become relevant in the \textit{dual mesh}
(section \ref{dual_mesh}).

\subsection{Orientation of a simplex}



\subsection{Dual mesh}\label{dual_mesh}

\section{Discrete differential forms}

% - forms integrated over mesh elements
% - discrete Hodge star and exterior derivative as mesh relations
% - exterior derivative is exact
% - chains and cochains

\parencite{desbrun_discrete_2006}

\section{Discretizing the model}

% - this one should be mostly copyable from my notes

\section{Harmonic Hodge operators}

\section{Whitney forms}



\chapter{Exact controllability}

% - most of this chapter should be copyable from notes too

\section{Objective}

\section{Adjoint state method}

\section{Conjugate gradient algorithm}



\chapter{Experiments}

\section{Implementation}

\section{Results}

\section{Challenges}

% - mesh quality matters a lot, stability can be a challenge
% - future work: Hodge optimized dual
% (maybe this section should be in the Discussion chapter?)



\chapter{Discussion}



\chapter{Conclusion}



\printbibliography

\end{document}

%! TEX program = xelatex

\documentclass{beamer}
\setlength{\parindent}{0pt}
\setlength{\parskip}{1em}
\usepackage{fontspec}
\setmainfont{Lato}
\usepackage[english]{babel}

\usepackage{amsmath,amssymb,amsthm}
% \usepackage{algorithmicx} % pseudocode
% \usepackage{algpseudocode} % pseudocode
\usepackage{hyperref} % links
% \usepackage{graphicx} % images
% \usepackage{pgfplots} % plots
% \pgfplotsset{compat=1.16} % plots

% verbatim code
% \usepackage{listings}
% \usepackage{xcolor}
% \definecolor{purple}{RGB}{135,20,85}
% \definecolor{gray}{RGB}{100,100,100}
% \lstset{
%   basicstyle=\ttfamily,
%   keywordstyle=\color{purple},
%   commentstyle=\color{gray},
% }

% matrices with customizable stretch
% as per https://tex.stackexchange.com/questions/14071/how-can-i-increase-the-line-spacing-in-a-matrix
\makeatletter
\renewcommand*\env@matrix[1][\arraystretch]{%
  \edef\arraystretch{#1}%
  \hskip -\arraycolsep
  \let\@ifnextchar\new@ifnextchar
  \array{*\c@MaxMatrixCols c}}
\makeatother

% usually I want images 350pt wide and centered
\newcommand{\img}[1]{
  \begin{center}
    \includegraphics[width=350pt]{#1}
  \end{center}
}

%
% begin document
%

% \usepackage[authordate,noibid,backend=biber]{biblatex-chicago}
% \addbibresource{}


\title{Diskreetti ulkoinen laskenta ja kontrollimenetelmä aikaharmonisessa akustiikkasimulaatiossa}
\author{Mikael Myyrä}
\date{Pro gradu -seminaari, kevät 2023}

\begin{document}

\frame{\titlepage}

\begin{frame}
  \frametitle{Aaltoyhtälö}

  Mallinnetaan akustisia aaltoja: ääntä ja vastaavia ilmiöitä.
  Tarkastellaan aaltoyhtälöä yhtälöryhmän muodossa:

  \[
    \begin{cases}
      \frac{\partial p}{\partial t} - c^2\nabla \cdot \mathbf{v} = f \\
      \frac{\partial \mathbf{v}}{\partial t} - \nabla p = 0 \\
    \end{cases}
  \]

  \pause

  jossa muuttujat
  \begin{itemize}
    \item $\mathbf{v}$: värähtelevien hiukkasten nopeus
    \item $p$: akustinen paine
  \end{itemize}

  (Ratkaisu vaatii lisäksi alkutilanteen ja reunaehdot.)

  \pause

  \textbf{Jatkuva} malli
  $\rightarrow$ tietokoneella ratkaisu vaatii \textbf{diskretointia}
\end{frame}

\begin{frame}
  \frametitle{Diskreetti ulkoinen laskenta}

  Engl. \textit{Discrete Exterior Calculus} (DEC)

  JYU:n tieteellisen laskennan tutkimusryhmän aihe

  Esimerkkejä: \url{https://sites.google.com/jyu.fi/gfd}
\end{frame}

\begin{frame}
  \frametitle{Diskreetti ulkoinen laskenta}

  Rakennetaan kolmioverkko, joka peittää laskenta-alueen

  \pause

  Lisäksi tarvitaan \textit{duaaliverkko}, joka yhdistää kolmioiden keskipisteet

  \pause

  Laskettaville muuttujille on kummassakin verkossa kolme mahdollista paikkaa:
  kärkipisteet, särmät ja tahkot

  \pause

  Kaksi operaatiota yhdistää verkon elementtejä:
  \begin{itemize}
    \item ulkoderivaatta siirtyy kärkipisteistä särmiin ja särmistä tahkoihin
    \item Hodgen tähti siirtyy verkosta duaaliverkkoon ja takaisin
  \end{itemize}

  Näiden avulla voidaan määritellä kaikki tehtävässä tarvittavat operaatiot
\end{frame}

\begin{frame}
  \frametitle{Diskreetti ulkoinen laskenta}
  Hienoa:

  Yleistyy monenlaisiin avaruuksiin:
  moniulotteiset, epäeuklidiset
  (esim. aika-avaruus, pallon pinta)

  Tarkka ulkoderivaattaoperaatio, kaikki virhe Hodge-operaattorissa
  -> helpottaa tarkkuuden analysointia ja parantamista

  Nopea: ei vaadi lineaaristen yhtälöryhmien ratkaisua

  (tämä dia on varmaan liikaa aikarajan puitteissa,
  ei ehdi selittää näitä väitteitä mitenkään)
\end{frame}

\begin{frame}
  \frametitle{Aikaharmonisuus}

  Sama tila toistuu jonkin ajan $T$ jälkeen: $u(T) = u(0)$

  Esimerkki-gif ei- ja kyllä-aikaharmonisesta tilanteesta?

  Tulkinta: kaikki lähteneet aallot ovat kulkeneet päätepisteeseensä asti.
  Esim. jos simuloidaan kaiutinta tai ultraäänilähetintä,
  tällainen tila kertoo kaikki suunnat mihin laitteesta lähtee ääntä

  (Tämä dia on ehkä tarpeeton tässä kohtaa,
  helpompi selittää tulkinta visuaalisen esimerkin kanssa)
\end{frame}

\begin{frame}
  \frametitle{Kontrollimenetelmä}

  Etsitään aikaharmoninen tila: alkutilanne, josta aloittamalla
  simulaatio palaa samaan tilaan ajan $T$ kuluttua

  \pause

  Tämä on optimointitehtävä: minimoi $|u(T) - u(0)|$
  siten että $u$ toteuttaa aaltoyhtälön

  Voidaan käyttää tavallisiin optimointitehtäviin kehitettyjä algoritmeja!
\end{frame}

\begin{frame}
  \frametitle{Toteutus}

  Python-kirjasto PyDEC toteuttaa DEC:n perusoperaatiot

  Lähdekoodit GitHubissa: \url{https://github.com/m0lentum/decathlon} \\
  (huomaa käyttäjänimessä O:n paikalla nolla)
\end{frame}

\end{document}

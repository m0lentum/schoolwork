%! TEX program = xelatex

\documentclass{article}
\usepackage[a4paper, margin=3cm]{geometry}
\setlength{\parindent}{0pt}
\setlength{\parskip}{1em}
\usepackage{titlesec}
\titlespacing\section{0pt}{16pt plus 4pt minus 2pt}{2pt plus 2pt minus 2pt}
\usepackage{fontspec}
\setmainfont{Lato}
\usepackage[finnish]{babel}

\usepackage{amsmath,amssymb,amsthm}
% \usepackage{algorithmicx} % pseudocode
% \usepackage{algpseudocode} % pseudocode
% \usepackage{hyperref} % links
% \usepackage{graphicx} % images
% \usepackage{pgfplots} % plots
% \pgfplotsset{compat=1.16} % plots

% verbatim code
% \usepackage{listings}
% \usepackage{xcolor}
% \definecolor{purple}{RGB}{135,20,85}
% \definecolor{gray}{RGB}{100,100,100}
% \lstset{
%   basicstyle=\ttfamily,
%   keywordstyle=\color{purple},
%   commentstyle=\color{gray},
% }

% matrices with customizable stretch
% as per https://tex.stackexchange.com/questions/14071/how-can-i-increase-the-line-spacing-in-a-matrix
\makeatletter
\renewcommand*\env@matrix[1][\arraystretch]{%
  \edef\arraystretch{#1}%
  \hskip -\arraycolsep
  \let\@ifnextchar\new@ifnextchar
  \array{*\c@MaxMatrixCols c}}
\makeatother

% usually I want images 350pt wide and centered
\newcommand{\img}[1]{
  \begin{center}
    \includegraphics[width=350pt]{#1}
  \end{center}
}

%
% begin document
%

% \usepackage[authordate,noibid,backend=biber]{biblatex-chicago}
% \addbibresource{}


\title{TIES5011 lukupiiriraportti: Systemaattiset kirjallisuuskartoitukset ja -katsaukset}
\author{Arkipäivänörtit}
\date{}

\begin{document}
\maketitle

Etätapaaminen 31.10.2022 klo 18:00—18:45.
Paikalla Anne Heino, Sanni Marjanen ja Mikael Myyrä.
Raportin kirjoittaja Mikael Myyrä.

\section*{Menetelmät ja niiden tarkoitukset}

Kirjallisuuskartoituksen ja -katsauksen tarkoitus on koota kokonaiskuva
tiettyyn tutkimusalaan tai aiheseen liittyvästä kirjallisuudesta. Menetelmän
yksityiskohdat riippuvat kysymyksistä, joihin kokonaiskuvan perusteella
pyritään vastaamaan. Jos tarkoitus on saada metatietoa kirjallisuudesta
itsestään, puhutaan kirjallisuuskartoituksesta. Tällöin tutkimuskysymyksenä voi
olla esimerkiksi alan trendien tai puutteellisesti tutkittujen osa-alueiden
tunnistaminen. Sen sijaan jos tavoitteena on yhdistää tutkimuksissa tuotettua
tietoa ja tehdä johtopäätöksiä, kyseessä on kirjallisuuskatsaus.

Näillä menetelmillä toteutettujen tutkimusten kohdeyleisö on yleensä muut
tutkijat. Kokonaiskuva aiheesta ei sinällään välttämättä vie alan kehitystä
eteenpäin, mutta se mahdollistaa tulevan tutkimuksen kohdentamisen 
mahdollisimman hyödyllisiin aiheisiin.

Koimme yllättäväksi, että tietotekniikan alalla nämä tutkimusmenetelmät ovat
yleistyneet vasta 2000-luvulla. Tutkijan on tärkeä tuntea alansa olemassaoleva
tieto mahdollisimman kokonaisvaltaisesti voidakseen kehittää aidosti uutta
tietoa, ja ilman kokoavia tutkimuksia tällainen tuntemus tuntuisi vaikealta
saavuttaa.

\section*{Tutkimuksen kulku}

Kirjallisuuskartoituksen toteuttamiseen on useita ohjeita ja suuntaviivoja,
joissa yksityiskohdat vaihtelevat. Tarkastelimme kurssilla annettuja
Petersenin ym. (2008) sekä Kitchenhamin ja Chartersin (2007) ohjeita.
Näiden kolme pääasiallista vaihetta olivat suunnittelu, toteutus ja raportointi.

Olennaista suunnittelussa on tutkimuskysymysten määrittely ja niihin vastaamaan
sopivan menetelmän valinta. Tämän jälkeen voidaan kehittää hakuprotokolla,
jonka perusteella kirjallisuutta lähdetään hakemaan. On tärkeää määritellä
käytetyt hakusanat, hakukoneet sekä kriteerit artikkelien valintaan ja
poissulkemiseen riittävän yksityiskohtaisesti, että lukija pystyy halutessaan
toistamaan haun ja saamaan vastaavan lopputuloksen. Subjektiivista tulkintaa
vaativia sääntöjä kannattaa välttää.

Tutkimuskysymyksistä riippuen voi olla hyödyllistä tarkastella
vertaisarvioitujen julkaisujen lisäksi epävirallista ns. harmaata
kirjallisuutta, kuten esitelmiä, yritysten julkaisuja ja blogeja. Tällainen
tieto voi olla erityisen hyödyllistä, jos tavoitteena on tunnistaa
puutteellisesti tutkittuja aihealueita tulevaa tutkimustyötä varten.

Toteutusvaiheessa on tärkeää käyttää aineiston hallintaan sopivia työkaluja.
Suuren tietomassan käsittely on haastavaa, ja esim. taulukkolaskenta-
tai tietokantasovellukset auttavat johdonmukaisesti käytettynä pitämään
tiedot tallessa ja järjestyksessä.

\section*{Soveltaminen graduun}

Kirjallisuuskartoitus tai -katsaus on kiinnostava menetelmä graduun
erityisesti jatko-opintoja suunnitteleville opiskelijoille. Kartoituksesta
saatava kokonaiskuva alasta on vahva pohja erikoistumiselle. Haasteena
tässä vaiheessa on kuitenkin työmäärän hallinta.

Hakutulosten määrää on vaikea arvioida etukäteen, joten niiden määrä saattaa
osoittautua yllättävän suureksi. Tällöin aineistoa täytyy pienentää, jotta sen
läpikäynti onnistuu rajatun työmäärän puitteissa. Tutkimuskysymyksistä riippuen
artikkelien määrää voidaan vähentää esimerkiksi valintakriteerejä muuttamalla
tai satunnaisotannalla, mutta joka tapauksessa tulosten luotettavuus kärsii
tästä jonkin verran.

Myös tulosten laadunvarmistuksesta joudutaan tinkimään. Artikkelien valinta
olisi suositeltavaa tehdä useamman tutkijan yhteistyönä, mutta tällaiselle ei
ole gradussa aikaa. Koko tekstin lukeminen ja tulosten luotettavuuden arviointi
on työläs operaatio, jota ei voida tehdä suurelle aineistolle.

Raportointivaiheeseen kuuluvaa työn julkaisua ja levittämistä ei gradun
kohdalla todennäköisesti tehdä lainkaan. Koska tavoite ei ole vaikuttaa alan
tutkimukseen laajasti, työmäärän aiheuttamat rajoitteet eivät ole merkittävä
haitta. On kuitenkin syytä mainita rajoitteista gradun tekstissä ja arvioida
niiden vaikutusta tutkimuksen luotettavuuteen.

Menetelmä muistuttaa IT-tiedekunnan tavanomaisten kandidaatintutkielmien
menetelmää, ''perinteistä'' kirjallisuuskatsausta. Eroavaisuutena on haun
toistettavuus ja kattavuus. Siksi samankaltaiset aiheet sopivat myös
kirjallisuuskartoitusgraduun. Esimerkki tällaisesta aiheesta voisi olla
Mikaelin kandidaatintutkielman aihe, katsaus reaaliaikaisista
nestesimulaatiomenetelmistä, laajennettuna systemaattiseksi kartoitukseksi.

\section*{Lähteet}

Petersen, K., Feldt, R., Mujtaba, S. \& Mattsson, M. (2008).
Systematic mapping studies in software engineering.
In Ease (Vol. 8, pp. 68-77).

Kitchenham, B. \& Charters, S. (2007). “Guidelines for performing
systematic literature reviews in software engineering,”
Tech. Rep. EBSE 2007-001, 2007. N.B.

\end{document}

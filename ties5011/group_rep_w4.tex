%! TEX program = xelatex

\documentclass{article}
\usepackage[a4paper, margin=3cm]{geometry}
\setlength{\parindent}{0pt}
\setlength{\parskip}{1em}
\usepackage{titlesec}
\titlespacing\section{0pt}{16pt plus 4pt minus 2pt}{2pt plus 2pt minus 2pt}
\usepackage{fontspec}
\setmainfont{Lato}
\usepackage[finnish]{babel}

\usepackage{amsmath,amssymb,amsthm}
% \usepackage{algorithmicx} % pseudocode
% \usepackage{algpseudocode} % pseudocode
% \usepackage{hyperref} % links
% \usepackage{graphicx} % images
% \usepackage{pgfplots} % plots
% \pgfplotsset{compat=1.16} % plots

% verbatim code
% \usepackage{listings}
% \usepackage{xcolor}
% \definecolor{purple}{RGB}{135,20,85}
% \definecolor{gray}{RGB}{100,100,100}
% \lstset{
%   basicstyle=\ttfamily,
%   keywordstyle=\color{purple},
%   commentstyle=\color{gray},
% }

% matrices with customizable stretch
% as per https://tex.stackexchange.com/questions/14071/how-can-i-increase-the-line-spacing-in-a-matrix
\makeatletter
\renewcommand*\env@matrix[1][\arraystretch]{%
  \edef\arraystretch{#1}%
  \hskip -\arraycolsep
  \let\@ifnextchar\new@ifnextchar
  \array{*\c@MaxMatrixCols c}}
\makeatother

% usually I want images 350pt wide and centered
\newcommand{\img}[1]{
  \begin{center}
    \includegraphics[width=350pt]{#1}
  \end{center}
}

%
% begin document
%

% \usepackage[authordate,noibid,backend=biber]{biblatex-chicago}
% \addbibresource{}


\title{TIES5011 lukupiiriraportti: Käsitteellis-teoreettinen tutkimus}
\author{Arkipäivänörtit}
\date{}

\begin{document}
\maketitle

Etätapaaminen 21.11.2022 klo 18:00—18:35.
Paikalla Anne Heino, Sanni Marjanen ja Mikael Myyrä.
Raportin kirjoittaja Mikael Myyrä.

\section*{Menetelmä}

Käsitteellis-teoreettinen tutkimus on sekundääristä tutkimusta, joka
tarkastelee aikaisemman tutkimuksen tuloksia ja etsii niistä yhdistäviä
piirteitä teorioiden, mallien, viitekehyksien ja käsitteiden muodossa.
Tällaisella tutkimuksella ei tuoteta suoraan uusia empiirisiä tuloksia, mutta
sillä on tärkeä rooli tulevan tutkimuksen suuntaamisessa ja alan yhteisen
sanaston rakentamisessa.

Teorian kehittäminen voi tapahtua induktiivisesti empiiristen tutkimustulosten
perusteella tai deduktiivisesti päättelemällä alkaen tunnetuista aksioomista.
Induktiivisesti kehitetyn teorian tulisi tunnistaa ja mahdollisesti selittää
olennaisia yhteisiä piirteitä tutkimusten tuloksista. Tässä olennaisuus vaatii
tutkijalta subjektiivista tulkintaa. Deduktiivista tutkimusta tehdään paljon
matematiikan alalla, mutta muilta aloilta emme keksineet konkreettisia
esimerkkejä. Luonnontieteen matemaattisiin malleihin voisi varmaankin myös
soveltaa tällaista tutkimustapaa.

Teorioita on monenlaisia eri tarkoituksiin (mm. kuvailu ja luokittelu,
selittäminen, ennakointi, suunnittelu), ja teorian käsitteen määrittely
yleisellä tasolla on haastava filosofinen kysymys. Koimme kuitenkin
teoreettisen tutkimuksen tavanomaisen prosessin varsin selkeäksi. Kyseessä on
ikään kuin edistyneempi muoto kirjallisuuskatsauksesta, jossa kirjallisuuden
tuloksia pyritään selittämään. Tämä vaatii kirjallisuuskatsaukseen verrattuna
syvempää ymmärrystä lähteiden sisällöistä, mutta prosessi on oleellisesti sama.

Pohdimme, mitä tarkoittaa Järvisen väittämä ''IT:n vaikutusten tutkimus usein
nojaa 'kahteen jalkaan', tietojenkäsittelytieteeseen ja johonkin
referenssitieteeseen''. Päädyimme tulkintaan, että tässä on puhe IT:n ja
ihmisen tai yhteiskunnan välisten vuorovaikutuksen tutkimisesta, jolloin
ihmisen osallisuus on huomioitava ihmistieteen menetelmin.

\section*{Soveltuvuus graduun}

Tämä on tähän mennessä käsitellyistä menetelmistä haastavin. Varsinkin
selittävien teorioiden kehittäminen vaatii syvää asiantuntemusta, jollaiseen on
vaikea päästä maisterivaiheessa. Helpoimmalta vaikuttava tutkimustyyppi olisi
käsiteanalyysi, jossa voidaan keskittyä tutkimuksissa käytettyyn kieleen
tarvitsematta tulosten syvällistä tarkastelua. Helpoimmillaankin työmäärä on
kuitenkin suuri, ja menetelmä soveltuu parhaiten opiskelijoille, jotka ovat
aiemmin suorittaneet tutkinnon muulla alalla tai aikovat jatkaa saman aiheen
tarkastelua jatko-opinnoissa.

\section*{Ryhmän lukemat lähteet}

Järvinen, ?. (????). Teoreettis-käsitteellinen tutkimus.

Pirttimäki, V. (2007). Conceptual analysis of business intelligence.
South African journal of information management.

Brinkmann, M., \& Heine, M. (2019).
Can Blockchain Leverage for New Public Governance?:
A Con\-ceptual Analysis on Process Level.
In Proceedings of the 12th International Conference on
Theory and Practice of Electronic Governance(pp. 338-341). ACM.

Koskela, T., Kassinen, O., Harjula, E., \& Ylianttila, M. (2013).
P2P group management systems: A conceptual analysis.
ACM Computing Surveys (CSUR), 45(2), 20

\end{document}

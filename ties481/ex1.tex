%! TEX program = xelatex

\documentclass{article}
\usepackage[a4paper, margin=3cm]{geometry}
\setlength{\parindent}{0pt}
\setlength{\parskip}{1em}
\usepackage{fontspec}
\setmainfont{Lato}

\usepackage{amsmath,amssymb,amsthm}
% \usepackage{algorithmicx}
% \usepackage{algpseudocode}
% \usepackage{hyperref}
% \usepackage{graphicx}
% \usepackage{pgfplots}
% \pgfplotsset{compat=1.16}

\usepackage{verbatim}
\usepackage{listings}
\usepackage{xcolor}
\definecolor{purple}{RGB}{135,20,85}
\definecolor{gray}{RGB}{100,100,100}
\lstset{
  basicstyle=\ttfamily,
  keywordstyle=\color{purple},
  commentstyle=\color{gray},
}

% matrices with customizable stretch
% as per https://tex.stackexchange.com/questions/14071/how-can-i-increase-the-line-spacing-in-a-matrix
\makeatletter
\renewcommand*\env@matrix[1][\arraystretch]{%
  \edef\arraystretch{#1}%
  \hskip -\arraycolsep
  \let\@ifnextchar\new@ifnextchar
  \array{*\c@MaxMatrixCols c}}
\makeatother

% usually I want images 350pt wide and centered
\newcommand{\img}[1]{
  \begin{center}
    \includegraphics[width=350pt]{#1}
  \end{center}
}

%
% begin document
%

\title{TIES481, First simulation assignment}
\author{Mikael Myyrä}
\date{\number\day.\number\month.\number\year}

\begin{document}
\maketitle

Additional feature: sometimes operations will fail / be ineffective and require the
patient to go through the process again. The goal is to see what effect the
error rate has on congestion in the system if the rate of new patients stays
the same.

\section*{Process-based}

This is the more intuitive approach of the two to me, so I'll go over it first.

The system contains facilities with some capacity and patients taking up that capacity.
A single patient going through the entire system seems like a good candidate
for the main process type here because it has a start and end, unlike the
operation of a facility, which continues forever. Steps of this process:
\begin{enumerate}
  \item{Patient enters the system.}
  \item{Patient is sent to a preparation facility. Reserve a spot from the
    facility with the most space or wait for one to become available as another
    patient moves to the next step.}
  \item{Patient moves to the operation facility. We only have one operation
    facility, so wait for it to have room if it's full.}
  \item{Patient moves to a recovery facility. Reserve a spot from the facility
    with the most space or wait for one to become available.}
  \item{If the operation failed, the patient goes back to step 2 after some
    time. Otherwise, the patient exits the system.}
\end{enumerate}

Some amount of time passes between each of these events. This will require a
scheduler system and a method for generating random time values in a given
distribution. Top-level data structures will be needed to store the
current capacity of each facility. The ability for a patient to wait for room
to open up in a facility also requires some structure for handling it, e.g. a
FIFO queue of patient IDs per facility.

A patient would store its current state and the time when it plans to
transition to the next state. Patients could be stored in an array in arbitrary
order and iterated over every time the scheduler updates, or we could use a
priority queue to have them sorted by which one has a state transition
occurring next.

For the goals of average throughput time and time the operating theater is
blocked due to being full, we can store time spent in the system with each
patient's data structure, collecting statistics (sum of all times and number of
patients to compute average) as they leave, and have a tracking variable in the
scheduler system for time the recovery facilities are all full.

\section*{Event-based}

The components and events are essentially the same as in the process-based model.
The system still contains facilities and patients. A scheduler and top-level
structures for facility capacity are also still there. Events correspond to
steps of the patient process outlined in the previous section, with the
difference that when they happen they would schedule a new event corresponding
to the next state transition, instead of storing the time of next transition
in the patient structure. The scheduler, then, would operate on events rather
than patients, and patients would be treated as a simple resource much like
facilities.

Since patients appear and disappear, a slab seems like a good data structure
for storing them. Events would then refer to patients by an identifier
consisting of their index and generation in the slab.
Events, on the other hand, would probably be best served by a priority queue
sorted by time.

Tracking the desired statistics would also work in much the same way as before;
a top-level variable to track time when the recovery facilities are full and a
variable per patient to track their total time in the system.

\end{document}

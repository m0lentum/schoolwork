%! TEX program = xelatex

\documentclass{article}
\usepackage[a4paper, margin=3cm]{geometry}
\setlength{\parindent}{0pt}
\setlength{\parskip}{1em}
\usepackage{fontspec}
\setmainfont{Lato}

\usepackage{amsmath,amssymb,amsthm}
% \usepackage{algorithmicx}
% \usepackage{algpseudocode}
% \usepackage{hyperref}
% \usepackage{graphicx}
% \usepackage{pgfplots}
% \pgfplotsset{compat=1.16}

\usepackage{verbatim}
\usepackage{listings}
\usepackage{xcolor}
\definecolor{purple}{RGB}{135,20,85}
\definecolor{gray}{RGB}{100,100,100}
\lstset{
  basicstyle=\ttfamily,
  keywordstyle=\color{purple},
  commentstyle=\color{gray},
}

% matrices with customizable stretch
% as per https://tex.stackexchange.com/questions/14071/how-can-i-increase-the-line-spacing-in-a-matrix
\makeatletter
\renewcommand*\env@matrix[1][\arraystretch]{%
  \edef\arraystretch{#1}%
  \hskip -\arraycolsep
  \let\@ifnextchar\new@ifnextchar
  \array{*\c@MaxMatrixCols c}}
\makeatother

% usually I want images 350pt wide and centered
\newcommand{\img}[1]{
  \begin{center}
    \includegraphics[width=350pt]{#1}
  \end{center}
}

%
% begin document
%

\usepackage[authordate,noibid,backend=biber]{biblatex-chicago}
\addbibresource{term_paper_sources.bib}

\title{TIES481 Term Paper: Procedural Noise in Simulation}
\author{Mikael Myyrä}
\date{\number\day.\number\month.\number\year}

\begin{document}
\maketitle

\section*{Preface}

Procedural noise is a class of pseudorandom functions which is closely related
to random number generators and commonly used particularly in computer graphics
applications. Since randomness is an important part of many simulations, as a
computer graphics enthusiast I was curious to find out how often noise is used
to replace or complement traditional RNGs. This also seemed like a good excuse
to dive deeper into the theory of noise functions than I have in the past.

This turned out to be a fairly difficult thing to search for because the
keywords ''noise'' and ''simulation'' tended to bring up simulations of
physical noise (sound), so I took the approach of first looking up specific
noise functions and then searching for their names and ''simulation''.
In this paper I cover a few different noise functions along with
usages in simulation that I came across.

\section*{Definition}

Intuitively speaking, noise can be thought of as a random continuous signal.
In particular, \textit{white noise} is a signal containing an equal amount of
every frequency with random phase \parencite{lagae_survey_2010}. In discrete
time this corresponds to a set of serially uncorrelated random samples, much
like those generated by RNGs. Noise functions in general, then, are essentially
picking out parts of white noise to create desired signal shapes by
band-limiting the frequency distribution \parencite{lagae_survey_2010}.

What differentiates typical implementations of discrete white noise from the
sequences produced by traditional RNGs is the capability for random access,
meaning any element of the sequence can be evaluated at any time. To this end,
hash functions such as the ones outlined by
\citeauthor{eiserloh_noise-based_2017} in their Game Developers Conference
presentation \parencite*{eiserloh_noise-based_2017} are used in place of the
stateful recurrence relations found in RNGs. Hash functions are pure functions
that take a seed value and an integer and transform the integer into another
uncorrelated value.

\section*{Types of noise}

\subsection*{White noise}

Regular white noise (i.e. just a hash function) can be used as a drop-in
replacement for traditional RNGs by making the position parameter an internal
state variable. To generate a new number, one would then increment the
position and return the hash function's value evaluated at the new position.
\parencite{eiserloh_noise-based_2017}

I did not find any mention of this technique being used in simulations. It
seems useful to me, but it looks like it's either not used at all or not talked
about prominently enough for my online searches to hit. I explain my thoughts
on this in the Conclusion section.

\subsection*{Perlin noise}

The most commonly used noise function in computer graphics, Perlin noise
\parencite{perlin_image_1985,perlin_improving_2002} is the archetypical
\textit{lattice gradient noise} \parencite{lagae_survey_2010}. It uses a hash
function at integer coordinates to generate samples of white noise and random
gradient vectors. Values for points outside of the integer lattice are obtained
from the nearest integer points' values by spline interpolation. This produces
a random signal with only one frequency. Multiple layers of the noise with
different spacings on the lattice (called ''octaves'') can be summed to
add frequencies to the signal and generate more detailed patterns.

The hash function used to generate values on the integer lattice varies a
little between implementations (e.g. Perlin's original
\parencite*{perlin_image_1985}, Perlin's improved version
\parencite*{perlin_improving_2002}, \textcite{kensler_better_2008} with
improved axial decorrelation). In common with all of these is that a scrambled
lookup table of the first 256 integers is used. An index corresponding to the
input point is obtained from this table and used to retrieve a gradient vector
from another array.

The original purpose of Perlin noise is generating surface and volume textures
for graphics rendering, but I found some applications of it in simulations as
well. \textcite{prieto_traffic_2012} used one-dimensional Perlin noise to
generate traffic for computer network simulations.
\textcite{barufaldi_computational_2021} used 2D Perlin noise interpreted as
density to generate simulated breast tissue for the development of mammogram
technology. \textcite{jakes_perlin_2019} used it to generate patterns of a
medical condition called fibrosis for simulated study.
\textcite{wang_inherent_2014} used a 3D Perlin noise velocity field to guide
the movement of simulated swarms of insects.

\subsection*{Simplex noise}

Simplex noise \parencite{perlin_real-time_2002} is a variant of Perlin noise
where instead of the integer lattice, random values are generated on the
vertices of a grid of simplices (triangles in 2D, tetrahedra in 3D etc.),
reducing interpolation work and producing patterns which tend to align less
with the principal coordinate axes. The hash function used is also somewhat
different, using a faster bit-manipulation algorithm whereas Perlin noise used
a series of table lookups.

Simplex noise is mostly identical to Perlin noise and used for the same
purposes. Thus, it technically could have been used in all the studies
mentioned in the Perlin noise section. For some examples of publications
specifically mentioning simplex noise, \textcite{kahraman_influence_2017} used
it to generate fog for simulated testing of computer vision algorithms, and
\textcite{perez-garcia_simulation_2020} to generate brain tissue for machine
learning -based simulations of brain surgeries for epilepsy.

\subsection*{Curl noise}

Another kind of lattice gradient noise useful in particle simulations is
obtained by interpreting Perlin noise as a potential field and taking its curl,
producing a vector field with zero divergence
\parencite{bridson_curl-noise_2007}. Zero divergence is a defining feature of
the velocity of incompressible fluids, and this kind of noise is commonly used
to simulate the movement of particles for turbulent visual effects such as
smoke (e.g. \textcite{schechter_evolving_2008}).

\subsection*{Other noises}

The survey of \textcite{lagae_survey_2010} lists a number of other kinds of
noise.

Explicit noises such as wavelet noise are generated and stored ahead of
time, which makes them unsuitable for generating randomness for long
simulations, as they would require enough memory to store every random sample
ever used by the simulation up-front. They could still be used to pre-generate
materials like many of the mentioned studies did with Perlin noise, but I did
not find any publications doing this.

Sparse convolution noises place instances of a ''kernel'' function at random
positions with random weights and are evaluated by taking the sum of all of
them. These, too, seem to be mostly useful for graphical purposes, as I did not
find any usages of these noises in simulation.

\section*{Conclusion}

Noise seems to be mostly useful in numerical simulations of mechanical
phenomena, which isn't exactly what this course is about but is very relevant
to my other studies. The closest thing to discrete event simulation I found is
the network traffic simulation of \textcite{prieto_traffic_2012}. This makes
sense given that much of the advantage of most noises comes from being
continuous in space, which isn't typically relevant in a discrete event
setting.

I think there are some potential benefits to the use of white noise as a
replacement for traditional RNG in simulations. In particular, when comparing
two simulation configurations with the same seed (like we did in Assignment 3),
noise would make it easier to ensure both configurations get the same random
values. In our hospital example, care needed to be taken to always do random
rolls in the same order. With noise, we could instead associate each patient
with an x coordinate and each of their service times with a y coordinate,
which could then be used to evaluate 2D white noise at any time, removing the
need to consider order. This would also make it easier to run parts of the
simulation in parallel without losing determinism if needed.

I didn't find as many usages of noise in simulations as I had hoped, so this
essay ended up being less about simulation and more about noise itself.
Perhaps this was a poor choice of topic for the course, but I am nonetheless
happy with how much I learned, and I'm sure I will find this information useful
in future projects.

\printbibliography

\end{document}

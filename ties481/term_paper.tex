%! TEX program = xelatex

\documentclass{article}
\usepackage[a4paper, margin=3cm]{geometry}
\setlength{\parindent}{0pt}
\setlength{\parskip}{1em}
\usepackage{fontspec}
\setmainfont{Lato}

\usepackage{amsmath,amssymb,amsthm}
% \usepackage{algorithmicx}
% \usepackage{algpseudocode}
% \usepackage{hyperref}
% \usepackage{graphicx}
% \usepackage{pgfplots}
% \pgfplotsset{compat=1.16}

\usepackage{verbatim}
\usepackage{listings}
\usepackage{xcolor}
\definecolor{purple}{RGB}{135,20,85}
\definecolor{gray}{RGB}{100,100,100}
\lstset{
  basicstyle=\ttfamily,
  keywordstyle=\color{purple},
  commentstyle=\color{gray},
}

% matrices with customizable stretch
% as per https://tex.stackexchange.com/questions/14071/how-can-i-increase-the-line-spacing-in-a-matrix
\makeatletter
\renewcommand*\env@matrix[1][\arraystretch]{%
  \edef\arraystretch{#1}%
  \hskip -\arraycolsep
  \let\@ifnextchar\new@ifnextchar
  \array{*\c@MaxMatrixCols c}}
\makeatother

% usually I want images 350pt wide and centered
\newcommand{\img}[1]{
  \begin{center}
    \includegraphics[width=350pt]{#1}
  \end{center}
}

%
% begin document
%

\usepackage[authordate,noibid,backend=biber]{biblatex-chicago}
\addbibresource{term_paper_sources.bib}

\title{TIES481 Term Paper: Procedural Noise in Simulation}
\author{Mikael Myyrä}
\date{\number\day.\number\month.\number\year}

\begin{document}
\maketitle

\section*{Preface}

Procedural noise is a class of pseudorandom functions which is closely related
to random number generators and commonly used particularly in computer graphics
applications. Since randomness is an important part of many simulations, as a
computer graphics enthusiast I was curious to find out how often noise is used
to replace or complement traditional RNGs. This also seemed like a good excuse
to dive deeper into the theory of noise functions than I have in the past.

This turned out to be a fairly difficult thing to search for because the
keywords ''noise'' and ''simulation'' tended to bring up simulations of
physical noise (sound), so I took the approach of first looking up specific
noise functions and then searching for their names and ''simulation''.
In this paper I cover a few different noise functions along with
usages in simulation that I came across.

\section*{Definition}

Intuitively speaking, noise can be thought of as a random continuous signal.
In particular, \textit{white noise} is a signal containing an equal amount of
every frequency with random phase \parencite{lagae_survey_2010}. In discrete
time this corresponds to a set of serially uncorrelated random samples, much
like those generated by RNGs. Noise functions in general, then, are essentially
picking out parts of white noise to create desired signal shapes by
band-limiting the frequency distribution \parencite{lagae_survey_2010}.

What differentiates typical implementations of discrete white noise from the
sequences produced by traditional RNGs is the requirement of \textit{random
access}, meaning any element of the sequence must be able to be evaluated at
any time. To this end, \textbf{hash functions} such as the ones outlined by
\citeauthor{eiserloh_noise-based_2017} in their Game Developers Conference
presentation \parencite*{eiserloh_noise-based_2017} are used in place of the
stateful recurrence relations found in RNGs. Hash functions take a seed value
and an integer and transform the integer into another uncorrelated value.

\section*{Kinds of noise}

\subsection*{White noise}

Pure white noise (i.e. just a hash function) can be used as a drop-in
replacement for traditional RNGs by making the position parameter an internal
state variable.  To generate a new number, one would then increment the
position and return the hash function's value at the new position.
\parencite{eiserloh_noise-based_2017}

I did not find any mention of this technique being used in simulations. If it's
used, it's likely left out of publications as an unimportant implementation
detail.

\subsection*{Perlin noise}

The most commonly used noise function in computer graphics, Perlin noise
\parencite{perlin_image_1985,perlin_improving_2002} is the archetypical
\textit{lattice gradient noise} \parencite{lagae_survey_2010}. It uses a hash
function to generate samples of white noise and random gradient vectors at
integer coordinates. Values for points outside of the integer lattice are
obtained from the nearest integer points' values by spline interpolation.
Multiple instances of the noise with different spacings on the lattice (called
''octaves'') can be summed to generate more detailed patterns.

The hash function used to generate values on the integer lattice varies a
little between implementations (e.g. Perlin's original
\parencite*{perlin_image_1985}, Perlin's improved version
\parencite*{perlin_improving_2002}, \textcite{kensler_better_2008} with
improved axial decorrelation).  In common with all of these is that a scrambled
lookup table of the first 256 integers is used. An index corresponding to the
input point is taken from this table and used to retrieve a gradient vector
from another array.

The original purpose of Perlin noise is generating surface and volume textures
for graphics rendering, but I found some applications of it in simulations as
well.  \textcite{prieto_traffic_2012} used one-dimensional Perlin noise to
generate simulated network traffic. \textcite{barufaldi_computational_2021}
used 2D Perlin noise interpreted as density to generate simulated breast tissue
for the development of mammogram technology. \textcite{jakes_perlin_2019} used
it to generate patterns of a medical condition called fibrosis for simulated
study. \textcite{wang_inherent_2014} used a 3D Perlin noise velocity field to
guide the movement of simulated swarms of insects.

\printbibliography

\end{document}

%! TEX program = xelatex

\documentclass{article}
\usepackage[a4paper, margin=3cm]{geometry}
\setlength{\parindent}{0pt}
\setlength{\parskip}{1em}
%\usepackage{fontspec}
%\setmainfont{Lato}

\usepackage{amsmath,amssymb,amsthm}
%\usepackage{hyperref}
%\usepackage{verbatim}
%\usepackage{graphicx}
%\usepackage{pgfplots}
%\pgfplotsset{compat=1.16}

\title{MATA114 Harjoitus 7}
\author{Mikael Myyrä}
\date{}

\begin{document}
\maketitle

\section*{1.}

\subsection*{(a)}

\[
  \frac{d^2 y}{d x^2} - 2x\frac{dy}{dx} + x^5y = 0
\]
Merkitään $t = x$, $x_1(t) = y(x)$ ja $x_2 = \frac{dx_1}{dt} = \frac{dy}{dx}$.
Saadaan
\[
  \begin{cases}
    \frac{dx_1}{dt} = x_2 \\
    \frac{dx_2}{dt} - 2tx_2 + t^5x_1 = 0. \\
  \end{cases}
\]

\subsection*{(b)}

\[
  \frac{d^5 y}{d x^5} + 8\frac{d^2 y}{d x^2} - y = 0
\]
Merkitään $t = x$, $x_1(t) = y(x)$, $x_2 = \frac{dx_1}{dt} = \frac{dy}{dx}$,
$x_3 = \frac{dx_2}{dt} = \frac{d^2 y}{d x^2}$ jne. $x_5$:n asti, jonka
ensimmäinen derivaatta on $\frac{d^5 y}{d x^5}$.
Saadaan
\[
  \begin{cases}
    \frac{dx_1}{dt} = x_2 \\
    \frac{dx_2}{dt} = x_3 \\
    \frac{dx_3}{dt} = x_4 \\
    \frac{dx_4}{dt} = x_5 \\
    \frac{dx_5}{dt} + 8x_3 - x_1 = 0. \\
  \end{cases}
\]

\section*{2.}

\[
  \begin{cases}
    \frac{dx}{dt} = y \\
    \frac{dy}{dt} = x \\
  \end{cases}
\]
Ensimmäisestä yhtälöstä saadaan $y = \frac{dx}{dt}$ ja $\frac{dy}{dt} = \frac{d^2 x}{d t^2}$.
Sijoittamalla tämä toiseen yhtälöön saadaan
\[
  \frac{d^2 x}{d t^2} = x \iff \frac{d^2 x}{d t^2} - x = 0 \\
\]
joka voidaan ratkaista karakteristisen yhtälön avulla:
\begin{align*}
  r^2 - 1 &= 0 \\
  r^2 &= 1 \\
  r &= \pm 1 \\
\end{align*}
Yhtälön ratkaisu on
\[
  x = C_1e^t + C_2e^{-t}, \quad C_1,C_2 \in \mathbb{R}.
\]
Tästä saadaan $y$:
\begin{align*}
  y &= \frac{dx}{dt} \\
    &= C_1e^t - C_2e^{-t},
\end{align*}
missä vakiot $C_1$ ja $C_2$ ovat samat kuin $x$:ssä.

\section*{3.}

\[
  \begin{cases}
    \frac{dx}{dt} = x + y \\
    \frac{dy}{dt} = 4x - 2y \\
  \end{cases}
\]
Ratkaistaan ensimmäisestä yhtälöstä $y$:
\[
  y = \frac{dx}{dt} - x
\]
ja
\[
  \frac{dy}{dt} = \frac{d^2 x}{d t^2} - \frac{dx}{dt}.
\]
Sijoitetaan toiseen yhtälöön:
\begin{align*}
  \frac{d^2 x}{d t^2} - \frac{dx}{dt} &= 4x - 2(\frac{dx}{dt} - x) \\
  \frac{d^2 x}{d t^2} + \frac{dx}{dt} - 6x &= 0 \\
\end{align*}
Ratkaistaan karakteristisen yhtälön avulla:
\begin{align*}
  r^2 + r - 6 &= 0 \\
  r &= \frac{-1 \pm \sqrt{1 + 24}}{2} \\
    &= -\frac{1}{2} \pm \frac{5}{2} \\
\end{align*}
Saadaan ratkaisu
\[
  x = C_1e^{2x} + C_2e^{-3x}, \quad C_1,C_2 \in \mathbb{R}.
\]
Ratkaistaan y:
\begin{align*}
  y &= \frac{dx}{dt} - x \\
    &= 2C_1e^{2x} - 3C_2e^{-3x} - (C_1e^{2x} + C_2e^{-3x}) \\
    &= C_1e^{2x} - 4C_2e^{-3x} \\
\end{align*}
Vastaus:
\[
  \begin{cases}
    x = C_1e^{2x} + C_2e^{-3x} \\
    y = C_1e^{2x} - 4C_2e^{-3x}
  \end{cases},
  \quad C_1,C_2 \in \mathbb{R}
\]

\section*{4.}

\[
  \begin{cases}
    \frac{dx}{dt} = x + y + 2e^t \\
    \frac{dy}{dt} = 4x + y - e^t
  \end{cases}
\]
Ratkaistaan ensimmäisestä yhtälöstä $y$:
\[
  y = \frac{dx}{dt} - x - 2e^t
\]
ja
\[
  \frac{dy}{dt} = \frac{d^2 x}{d t^2} - \frac{dx}{dt} - 2e^t.
\]
Sijoitetaan toiseen yhtälöön:
\begin{align*}
  \frac{d^2 x}{d t^2} - \frac{dx}{dt} - 2e^t &= 4x + \frac{dx}{dt} - x - 2e^t - e^t \\
  \frac{d^2 x}{d t^2} - 2\frac{dx}{dt} - 3x &= -e^t \\
\end{align*}
Ratkaistaan vastaava homogeeninen yhtälö karakteristisen yhtälön avulla:
\begin{align*}
  r^2 - 2r - 3 &= 0 \\
  r &= \frac{2 \pm \sqrt{4 + 12}}{2} \\
    &= 1 \pm 2 \\
\end{align*}
Homogeenisen yhtälön ratkaisu on siis
\[
  x = C_1e^{3t} + C_2e^{-t}, \quad C_1,C_2 \in \mathbb{R}.
\]
Etsitään epähomogeenisen yhtälön yksittäisratkaisua yritteellä $x = Ae^t$:
\begin{align*}
  Ae^t - 2Ae^t - 3Ae^t &= -e^t \\
  -4Ae^t &= -e^t \\
  A &= -\frac{1}{4} \\
\end{align*}
Saadaan
\[
  x = C_1e^{3t} + C_2e^{-t} + \frac{1}{4}e^t, \quad C_1,C_2 \in \mathbb{R}.
\]
Sijoitetaan $y$:n yhtälöön:
\begin{align*}
  y &= \frac{dx}{dt} - x - 2e^t \\
    &= 3C_1e^{3t} - C_2e^{-t} + \frac{1}{4}e^t - (C_1e^{3t} + C_2e^{-t} + \frac{1}{4}e^t) - 2e^t \\
    &= 2C_1e^{3t} - 2C_2e^{-t} - 2e^t \\
\end{align*}
Vastaus:
\[
  \begin{cases}
    x = C_1e^{3t} + C_2e^{-t} + \frac{1}{4}e^t \\
    y = 2C_1e^{3t} - 2C_2e^{-t} - 2e^t \\
  \end{cases}, \quad C_1,C_2 \in \mathbb{R}.
\]

\section*{5.}

\begin{align*}
  \begin{cases}
    \frac{dx}{dt} = 2x - 5y + \sin 2t, \quad & x(0) = 0 \\
    \frac{dy}{dt} = x - 2y + t, & y(0) = 1 \\
  \end{cases}
\end{align*}
Ratkaistaan toisesta yhtälöstä $x$:
\[
  x = \frac{dy}{dt} + 2y - t
\]
ja
\[
  \frac{dx}{dt} = \frac{d^2 y}{d t^2} + 2\frac{dy}{dt} - 1.
\]
Sijoitetaan ensimmäiseen yhtälöön:
\begin{align*}
  \frac{d^2 y}{d t^2} + 2\frac{dy}{dt} - 1 &= 2(\frac{dy}{dt} + 2y - t) - 5y + \sin 2t \\
  \frac{d^2 y}{d t^2} + y &= 1 - 2t + \sin 2t \\
\end{align*}
Ratkaistaan vastaava HY karakteristisen yhtälön avulla:
\begin{align*}
  r^2 + 1 &= 0 \\
  r &= \pm i \\
\end{align*}
Yhtälön ratkaisu on
\[
  y = C_1\cos t + C_2\sin t, \quad C_1,C_2 \in \mathbb{R}.
\]
Epähomogeenisen yhtälön ratkaisu vakioiden varioinnilla,
$C_1 = C_1(t)$ ja $C_2 = C_2(t)$.
\[
  \frac{dy}{dt} = \frac{dC_1}{dt}\cos t - C_1\sin t + \frac{dC_2}{dt}\sin t + C_2\cos t
\]
Lisäoletus: $\frac{dC_1}{dt}\cos t + \frac{dC_2}{dt}\sin t = 0$
\[
  \frac{dy}{dt} = -C_1\sin t + C_2\cos t
\]
ja
\[
  \frac{d^2 y}{d t^2} = -\frac{dC_1}{dt}\sin t - C_1\cos t + \frac{dC_2}{dt}\cos t - C_2\sin t
\]
Sijoitetaan DY:n:
\begin{align*}
  (-\frac{dC_1}{dt}\sin t - C_1\cos t + \frac{dC_2}{dt}\cos t - C_2\sin t)
  + (C_1\cos t + C_2\sin t) &= 1 - 2t + \sin 2t \\
  -\frac{dC_1}{dt}\sin t + \frac{dC_2}{dt}\cos t &= 1 - 2t + \sin 2t \\
\end{align*}

Tällä ei näytä olevan ratkaisua. Tein varmaankin jossain kohtaa laskuvirheen.
Yritän huomenna löytää sen, jos muistan.

\section*{6.}

\begin{align*}
  \begin{cases}
    \frac{dx}{dt} = 3x - 4y + e^t, \quad & x(0) = 1 \\
    \frac{dy}{dt} = x - y - e^t, & y(0) = -1 \\
  \end{cases}
\end{align*}

Ratkaistaan toisesta yhtälöstä $x$:
\[
  x = \frac{dy}{dt} + y + e^t
\]
ja
\[
  \frac{dx}{dt} = \frac{d^2 y}{d t^2} + \frac{dy}{dt} + e^t
\]
Sijoitetaan ensimmäiseen yhtälöön:
\begin{align*}
  \frac{d^2 y}{d t^2} + \frac{dy}{dt} + e^t &= 3(\frac{dy}{dt} + y + e^t) - 4y + e^t \\
  \frac{d^2 y}{d t^2} - 2\frac{dy}{dt} + y &= 3e^t \\
\end{align*}
Ratkaistaan vastaava HY karakteristisen yhtälön avulla:
\begin{align*}
  r^2 - 2r + 1 &= 0 \\
  r &= \frac{2 \pm \sqrt{4 - 4}}{2} \\
    &= 1
\end{align*}
Yhtälön ratkaisu on
\[
  y = C_1e^t + C_2te^t, \quad C_1,C_2 \in \mathbb{R}.
\]
$Ae^t$ on homogeenisen yhtälön ratkaisu, samoin $Ate^t$, joten etsitään
epähomogeenisen yhtälön ratkaisua yritteellä $y = At^2e^t$,
jolle $\frac{dy}{dt} = A(t^2 + 2t)e^t$ ja
$\frac{d^2 y}{d t^2} = A(t^2 + 4t + 2)e^t$:
\begin{align*}
  A(t^2 + 4t + 2)e^t - 2A(t^2 + 2t)e^t + At^2e^t &= 3e^t \\
  2Ae^t &= 3e^t \\
  A &= \frac{3}{2} \\
\end{align*}
Saadaan ratkaisu
\[
  y = C_1e^t + C_2te^t + \frac{3}{2}t^2e^t, \quad C_1,C_2 \in \mathbb{R}.
\]
Sijoitetaan $x$:n yhtälöön:
\begin{align*}
  x &= \frac{dy}{dt} + y + e^t \\
    &= (C_1e^t + C_2(t + 1)e^t + (3t + \frac{3}{2}t^2)e^t)
      + (C_1e^t + C_2te^t + \frac{3}{2}t^2e^t) + e^t \\
    &= (2C_1 + C_2 + 1)e^t + (2C_2 + 3)te^t + 3t^2e^t \\
\end{align*}
Alkuehto $y(0) = -1$:
\begin{align*}
  C_1e^0 + 0 + 0 &= -1 \\
  C_1 &= -1 \\
\end{align*}
Alkuehto $x(0) = 1$ (sijoitetaan myös $C_1 = -1$):
\begin{align*}
  (-2 + C_2 + 1)e^0 + 0 + 0 &= 1 \\
  C_2 - 1 &= 1 \\
  C_2 &= 2 \\
\end{align*}
Vastaus:
\[
  \begin{cases}
    x = e^t + 7te^t + 3t^2e^t \\
    y = -e^t + 2te^t + \frac{3}{2}t^2e^t
  \end{cases}, \quad C_1,C_2 \in \mathbb{R}.
\]

\end{document}

%! TEX program = xelatex

\documentclass{article}
\usepackage[a4paper, margin=3cm]{geometry}
\setlength{\parindent}{0pt}
\setlength{\parskip}{1em}
%\usepackage{fontspec}
%\setmainfont{Lato}

\usepackage{amsmath,amssymb,amsthm}
%\usepackage{hyperref}
%\usepackage{verbatim}
%\usepackage{graphicx}
%\usepackage{pgfplots}
%\pgfplotsset{compat=1.16}

\title{MATA114 Harjoitus 7}
\author{Mikael Myyrä}
\date{}

\begin{document}
\maketitle

\section*{1.}

\subsection*{(a)}

\[
  \frac{d^2 y}{d x^2} - 2x\frac{dy}{dx} + x^5y = 0
\]
Merkitään $t = x$, $x_1(t) = y(x)$ ja $x_2 = \frac{dx_1}{dt} = \frac{dy}{dx}$.
Saadaan
\[
  \begin{cases}
    \frac{dx_1}{dt} = x_2 \\
    \frac{dx_2}{dt} - 2tx_2 + t^5x_1 = 0. \\
  \end{cases}
\]

\subsection*{(b)}

\[
  \frac{d^5 y}{d x^5} + 8\frac{d^2 y}{d x^2} - y = 0
\]
Merkitään $t = x$, $x_1(t) = y(x)$, $x_2 = \frac{dx_1}{dt} = \frac{dy}{dx}$,
$x_3 = \frac{dx_2}{dt} = \frac{d^2 y}{d x^2}$ jne. $x_5$:n asti, jonka
ensimmäinen derivaatta on $\frac{d^5 y}{d x^5}$.
Saadaan
\[
  \begin{cases}
    \frac{dx_1}{dt} = x_2 \\
    \frac{dx_2}{dt} = x_3 \\
    \frac{dx_3}{dt} = x_4 \\
    \frac{dx_4}{dt} = x_5 \\
    \frac{dx_5}{dt} + 8x_3 - x_1 = 0. \\
  \end{cases}
\]

\section*{2.}

\[
  \begin{cases}
    \frac{dx}{dt} = y \\
    \frac{dy}{dt} = x \\
  \end{cases}
\]
Ensimmäisestä yhtälöstä saadaan $y = \frac{dx}{dt}$ ja $\frac{dy}{dt} = \frac{d^2 x}{d t^2}$.
Sijoittamalla tämä toiseen yhtälöön saadaan
\[
  \frac{d^2 x}{d t^2} = x \iff \frac{d^2 x}{d t^2} - x = 0 \\
\]
joka voidaan ratkaista karakteristisen yhtälön avulla:
\begin{align*}
  r^2 - 1 &= 0 \\
  r^2 &= 1 \\
  r &= \pm 1 \\
\end{align*}
Yhtälön ratkaisu on
\[
  x = C_1e^t + C_2e^{-t}, \quad C_1,C_2 \in \mathbb{R}.
\]
Tästä saadaan $y$:
\begin{align*}
  y &= \frac{dx}{dt} \\
    &= C_1e^t - C_2e^{-t},
\end{align*}
missä vakiot $C_1$ ja $C_2$ ovat samat kuin $x$:ssä.

\section*{3.}

\[
  \begin{cases}
    \frac{dx}{dt} = x + y \\
    \frac{dy}{dt} = 4x - 2y \\
  \end{cases}
\]
Ratkaistaan ensimmäisestä yhtälöstä $y$:
\[
  y = \frac{dx}{dt} - x
\]
ja
\[
  \frac{dy}{dt} = \frac{d^2 x}{d t^2} - \frac{dx}{dt}.
\]
Sijoitetaan toiseen yhtälöön:
\begin{align*}
  \frac{d^2 x}{d t^2} - \frac{dx}{dt} &= 4x - 2(\frac{dx}{dt} - x) \\
  \frac{d^2 x}{d t^2} + \frac{dx}{dt} - 6x &= 0 \\
\end{align*}
Ratkaistaan karakteristisen yhtälön avulla:
\begin{align*}
  r^2 + r - 6 &= 0 \\
  r &= \frac{-1 \pm \sqrt{1 + 24}}{2} \\
    &= -\frac{1}{2} \pm \frac{5}{2} \\
\end{align*}
Saadaan ratkaisu
\[
  x = C_1e^{2x} + C_2e^{-3x}, \quad C_1,C_2 \in \mathbb{R}.
\]
Ratkaistaan y:
\begin{align*}
  y &= \frac{dx}{dt} - x \\
    &= 2C_1e^{2x} - 3C_2e^{-3x} - (C_1e^{2x} + C_2e^{-3x}) \\
    &= C_1e^{2x} - 4C_2e^{-3x} \\
\end{align*}
Vastaus:
\[
  \begin{cases}
    x = C_1e^{2x} + C_2e^{-3x} \\
    y = C_1e^{2x} - 4C_2e^{-3x}
  \end{cases},
  \quad C_1,C_2 \in \mathbb{R}
\]

\section*{4.}

\[
  \begin{cases}
    \frac{dx}{dt} = x + y + 2e^t \\
    \frac{dy}{dt} = 4x + y - e^t
  \end{cases}
\]
Ratkaistaan ensimmäisestä yhtälöstä $y$:
\[
  y = \frac{dx}{dt} - x - 2e^t
\]
ja
\[
  \frac{dy}{dt} = \frac{d^2 x}{d t^2} - \frac{dx}{dt} - 2e^t.
\]
Sijoitetaan toiseen yhtälöön:
\begin{align*}
  \frac{d^2 x}{d t^2} - \frac{dx}{dt} - 2e^t &= 4x + \frac{dx}{dt} - x - 2e^t - e^t \\
  \frac{d^2 x}{d t^2} - 2\frac{dx}{dt} - 3x &= -e^t \\
\end{align*}
Ratkaistaan vastaava homogeeninen yhtälö karakteristisen yhtälön avulla:
\begin{align*}
  r^2 - 2r - 3 &= 0 \\
  r &= \frac{2 \pm \sqrt{4 + 12}}{2} \\
    &= 1 \pm 2 \\
\end{align*}
Homogeenisen yhtälön ratkaisu on siis
\[
  x = C_1e^{3t} + C_2e^{-t}, \quad C_1,C_2 \in \mathbb{R}.
\]
Etsitään epähomogeenisen yhtälön yksittäisratkaisua yritteellä $x = Ae^t$:
\begin{align*}
  Ae^t - 2Ae^t - 3Ae^t &= -e^t \\
  -4Ae^t &= -e^t \\
  A &= -\frac{1}{4} \\
\end{align*}
Saadaan
\[
  x = C_1e^{3t} + C_2e^{-t} + \frac{1}{4}e^t, \quad C_1,C_2 \in \mathbb{R}.
\]
Sijoitetaan $y$:n yhtälöön:
\begin{align*}
  y &= \frac{dx}{dt} - x - 2e^t \\
    &= 3C_1e^{3t} - C_2e^{-t} + \frac{1}{4}e^t - (C_1e^{3t} + C_2e^{-t} + \frac{1}{4}e^t) - 2e^t \\
    &= 2C_1e^{3t} - 2C_2e^{-t} - 2e^t \\
\end{align*}
Vastaus:
\[
  \begin{cases}
    x = C_1e^{3t} + C_2e^{-t} + \frac{1}{4}e^t \\
    y = 2C_1e^{3t} - 2C_2e^{-t} - 2e^t \\
  \end{cases}, \quad C_1,C_2 \in \mathbb{R}.
\]

\section*{5.}

\begin{align*}
  \begin{cases}
    \frac{dx}{dt} = 2x - 5y + \sin 2t, \quad & x(0) = 0 \\
    \frac{dy}{dt} = x - 2y + t, & y(0) = 1 \\
  \end{cases}
\end{align*}
Ratkaistaan toisesta yhtälöstä $x$:
\[
  x = \frac{dy}{dt} + 2y - t
\]
ja
\[
  \frac{dx}{dt} = \frac{d^2 y}{d t^2} + 2\frac{dy}{dt} - 1.
\]
Sijoitetaan ensimmäiseen yhtälöön:
\begin{align*}
  \frac{d^2 y}{d t^2} + 2\frac{dy}{dt} - 1 &= 2(\frac{dy}{dt} + 2y - t) - 5y + \sin 2t \\
  \frac{d^2 y}{d t^2} + y &= 1 - 2t + \sin 2t \\
\end{align*}
Ratkaistaan vastaava HY karakteristisen yhtälön avulla:
\begin{align*}
  r^2 + 1 &= 0 \\
  r &= \pm i \\
\end{align*}
Yhtälön ratkaisu on
\[
  y = C_1\cos t + C_2\sin t, \quad C_1,C_2 \in \mathbb{R}.
\]
Epähomogeenisen yhtälön ratkaisu vakioiden varioinnilla,
$C_1 = C_1(t)$ ja $C_2 = C_2(t)$.
\[
  \frac{dy}{dt} = \frac{dC_1}{dt}\cos t - C_1\sin t + \frac{dC_2}{dt}\sin t + C_2\cos t
\]
Lisäoletus: $\frac{dC_1}{dt}\cos t + \frac{dC_2}{dt}\sin t = 0$
\[
  \frac{dy}{dt} = -C_1\sin t + C_2\cos t
\]
ja
\[
  \frac{d^2 y}{d t^2} = -\frac{dC_1}{dt}\sin t - C_1\cos t + \frac{dC_2}{dt}\cos t - C_2\sin t
\]
Sijoitetaan DY:n:
\begin{align*}
  (-\frac{dC_1}{dt}\sin t - C_1\cos t + \frac{dC_2}{dt}\cos t - C_2\sin t)
  + (C_1\cos t + C_2\sin t) &= 1 - 2t + \sin 2t \\
  -\frac{dC_1}{dt}\sin t + \frac{dC_2}{dt}\cos t &= 1 - 2t + \sin 2t \\
\end{align*}

Monimutkainen lasku. Olisikohan yrite $y = A\sin 2t + Bt + C$ toiminut?
Kokeillaan:
\begin{align*}
  (-4A\sin 2t) + (A\sin 2t + Bt + C) &= 1 - 2t + \sin 2t \\
  -3A\sin 2t + Bt + C &= \sin 2t - 2t + 1 \\
  \begin{cases}
    A = -\frac{1}{3} \\
    B = -2 \\
    C = 1 \\
  \end{cases}
\end{align*}
Tästä saatiin ratkaisu helpommin. Yhtälön yleinen ratkaisu on siis
\[
  y = C_1\cos t + C_2\sin t -\frac{1}{3}\sin 2t - 2t + 1, \quad C_1,C_2 \in \mathbb{R}. \\
\]
Ratkaistaan $x$:
\begin{align*}
  x &= \frac{dy}{dt} + 2y - t \\
    &= (-C_1\sin t + C_2\cos t - \frac{2}{3}\cos 2t - 2)
    + 2(C_1\cos t + C_2\sin t - \frac{1}{3}\sin 2t - 2t + 1)
     - t \\
    &= (2C_2 - C_1)\sin t + (2C_1 + C_2)\cos t - \frac{2}{3}\sin 2t - \frac{2}{3}\cos t - 5t \\
\end{align*}
Alkuehto $y(0) = 1$:
\begin{align*}
  C_1\cos 0 + C_2\sin 0 -\frac{1}{3}\sin 0 - 0 + 1 &= 1 \\
  C_1 + 1 &= 1 \\
  C_1 &= 0 \\
\end{align*}
Alkuehto $x(0) = 0$ (sij. myös $C_1 = 0$):
\begin{align*}
  (2C_2 - 0)\sin 0 + (0 + C_2)\cos 0 - \frac{2}{3}\sin 0 - \frac{2}{3}\cos 0 - 0 &= 0
  C_2 - \frac{2}{3} &= 0 \\
  C_2 &= \frac{2}{3} \\
\end{align*}
Vastaus:
\[
  \begin{cases}
    x = \frac{4}{3}\sin t + \frac{2}{3}\cos t - \frac{2}{3}\sin 2t - \frac{2}{3}\cos t - 5t \\
    y = \frac{2}{3}\sin t - \frac{1}{3}\sin 2t - 2t + 1 \\
  \end{cases}, \quad C_1,C_2 \in \mathbb{R}.
\]

\section*{6.}

\begin{align*}
  \begin{cases}
    \frac{dx}{dt} = 3x - 4y + e^t, \quad & x(0) = 1 \\
    \frac{dy}{dt} = x - y - e^t, & y(0) = -1 \\
  \end{cases}
\end{align*}

Ratkaistaan toisesta yhtälöstä $x$:
\[
  x = \frac{dy}{dt} + y + e^t
\]
ja
\[
  \frac{dx}{dt} = \frac{d^2 y}{d t^2} + \frac{dy}{dt} + e^t
\]
Sijoitetaan ensimmäiseen yhtälöön:
\begin{align*}
  \frac{d^2 y}{d t^2} + \frac{dy}{dt} + e^t &= 3(\frac{dy}{dt} + y + e^t) - 4y + e^t \\
  \frac{d^2 y}{d t^2} - 2\frac{dy}{dt} + y &= 3e^t \\
\end{align*}
Ratkaistaan vastaava HY karakteristisen yhtälön avulla:
\begin{align*}
  r^2 - 2r + 1 &= 0 \\
  r &= \frac{2 \pm \sqrt{4 - 4}}{2} \\
    &= 1
\end{align*}
Yhtälön ratkaisu on
\[
  y = C_1e^t + C_2te^t, \quad C_1,C_2 \in \mathbb{R}.
\]
$Ae^t$ on homogeenisen yhtälön ratkaisu, samoin $Ate^t$, joten etsitään
epähomogeenisen yhtälön ratkaisua yritteellä $y = At^2e^t$,
jolle $\frac{dy}{dt} = A(t^2 + 2t)e^t$ ja
$\frac{d^2 y}{d t^2} = A(t^2 + 4t + 2)e^t$:
\begin{align*}
  A(t^2 + 4t + 2)e^t - 2A(t^2 + 2t)e^t + At^2e^t &= 3e^t \\
  2Ae^t &= 3e^t \\
  A &= \frac{3}{2} \\
\end{align*}
Saadaan ratkaisu
\[
  y = C_1e^t + C_2te^t + \frac{3}{2}t^2e^t, \quad C_1,C_2 \in \mathbb{R}.
\]
Sijoitetaan $x$:n yhtälöön:
\begin{align*}
  x &= \frac{dy}{dt} + y + e^t \\
    &= (C_1e^t + C_2(t + 1)e^t + (3t + \frac{3}{2}t^2)e^t)
      + (C_1e^t + C_2te^t + \frac{3}{2}t^2e^t) + e^t \\
    &= (2C_1 + C_2 + 1)e^t + (2C_2 + 3)te^t + 3t^2e^t \\
\end{align*}
Alkuehto $y(0) = -1$:
\begin{align*}
  C_1e^0 + 0 + 0 &= -1 \\
  C_1 &= -1 \\
\end{align*}
Alkuehto $x(0) = 1$ (sijoitetaan myös $C_1 = -1$):
\begin{align*}
  (-2 + C_2 + 1)e^0 + 0 + 0 &= 1 \\
  C_2 - 1 &= 1 \\
  C_2 &= 2 \\
\end{align*}
Vastaus:
\[
  \begin{cases}
    x = e^t + 7te^t + 3t^2e^t \\
    y = -e^t + 2te^t + \frac{3}{2}t^2e^t
  \end{cases}, \quad C_1,C_2 \in \mathbb{R}.
\]

\section*{7.}

\[
  \begin{cases}
    \frac{dx}{dt} = x - y + 4z \\
    \frac{dy}{dt} = 3x + 2y - z \\
    \frac{dz}{dt} = 2x + y - z \\
  \end{cases}
\]
Ratkaistaan ensimmäisestä yhtälöstä $y$:
\[
  y = -\frac{dx}{dt} + x + 4z
\]
ja
\[
  \frac{dy}{dt} = -\frac{d^2 x}{d t^2} + \frac{dx}{dt} + 4\frac{dz}{dt}
\]
Sijoitetaan toiseen yhtälöön tämä ja kolmannesta yhtälöstä $\frac{dz}{dt}$ ja
ratkaistaan $z$:
\begin{align*}
  -\frac{d^2 x}{d t^2} + \frac{dx}{dt} + 4\frac{dz}{dt} &= 3x + 2(-\frac{dx}{dt} + x + 4z) - z \\
  -\frac{d^2 x}{d t^2} + 3\frac{dx}{dt} + 4(2x + (-\frac{dx}{dt} + x + 4z) - z) &= 5x + 7z \\
  -\frac{d^2 x}{d t^2} - \frac{dx}{dt} + 12x + 12z &= 5x + 7z \\
  -\frac{d^2 x}{d t^2} - \frac{dx}{dt} + 8x &= -5z \\
  \frac{1}{5}\frac{d^2 x}{d t^2} + \frac{1}{5}\frac{dx}{dt} - \frac{8}{5}x &= z \\
\end{align*}
ja
\[
  \frac{dz}{dt} = \frac{1}{5}\frac{d^3 x}{d t^3} + \frac{1}{5}\frac{d^2 x}{d t^2} - \frac{8}{5}\frac{dx}{dt}.
\]
Sijoittamalla edellä lasketut tulokset kolmanteen yhtälöön saadaan
\begin{align*}
  \frac{1}{5}\frac{d^3 x}{d t^3} + \frac{1}{5}\frac{d^2 x}{d t^2} - \frac{8}{5}\frac{dx}{dt}
  &= 2x + (-\frac{dx}{dt} + x + 4z) - z \\
  \frac{1}{5}\frac{d^3 x}{d t^3} + \frac{1}{5}\frac{d^2 x}{d t^2} - \frac{3}{5}\frac{dx}{dt}
  - 3x - 3z &= 0 \\
  \frac{1}{5}\frac{d^3 x}{d t^3} + \frac{1}{5}\frac{d^2 x}{d t^2} - \frac{3}{5}\frac{dx}{dt}
  - 3x - 3(\frac{1}{5}\frac{d^2 x}{d t^2} + \frac{1}{5}\frac{dx}{dt} - \frac{8}{5}x) &= 0 \\
  \frac{1}{5}\frac{d^3 x}{d t^3} - \frac{2}{5}\frac{d^2 x}{d t^2} - \frac{6}{5}\frac{dx}{dt}
  + \frac{9}{5}x &= 0 \\
  \frac{d^3 x}{d t^3} - 2\frac{d^2 x}{d t^2} - 6\frac{dx}{dt} + 9x &= 0 \\
\end{align*}
Karakteristinen yhtälö:
\[
  r^3 - 2r^2 - 6r + 9 = 0
\]
Ratkaisen tämän Wolfram Alphalla ajan säästämiseksi.
Ratkaisut ovat $r = 3$ ja $r = -\frac{1}{2} \pm \frac{\sqrt{13}}{2}$,
joista saadaan
\[
  x = C_1e^{3x} + C_2e^{-\frac{1}{2} + \frac{\sqrt{13}}{2}} + C_3e^{-\frac{1}{2} - \frac{\sqrt{13}}{2}},
  \quad C_1,C_2,C_3 \in \mathbb{R}.
\]
Tämä ei näytä annetulta oikealta ratkaisulta, mutta en nopealla selauksella
löydä laskuistani virhettä. Aika loppuu kesken, ja näiden lukujen kanssa on sen
verran vaikea laskea eteenpäin, että jätän tämän kesken.

\section*{8.}

\subsection*{(a)}

Täytyy osoittaa, että yhtälöpari
\[
  \begin{cases}
    \frac{d}{dt}(C_1x_1 + C_2x_2) = p(t)(C_1x_1 + C_2x_2) + q(t)(C_1y_1 + C_2y_2) \\
    \frac{d}{dt}(C_1y_1 + C_2y_2) = r(t)(C_1x_1 + C_2x_2) + s(t)(C_1y_1 + C_2y_2) \\
  \end{cases}
\]
toteutuu kaikilla $C_1,C_2 \in \mathbb{R}$.
Ensimmäisestä yhtälöstä saadaan
\begin{align*}
  C_1\frac{dx_1}{dt} + C_2\frac{dx_2}{dt} &= C_1p(t)x_1 + C_2p(t)x_2 + C_1q(t)y_1 + C_2q(t)y_2 \\
  0 &= C_1(-\frac{dx_1}{dt} + p(t)x_1 + q(t)y_1) + C_2(-\frac{dx_2}{dt} + p(t)x_2 + q(t)y_2) \\
  \noalign{\text{Oletus:}} \\
  0 &= C_1 * 0 + C_2 * 0 \\
  0 &= 0 \\
\end{align*}
Siis ensimmäinen yhtälö toteutuu.
Vastaavasti toisesta yhtälöstä saadaan
\begin{align*}
  0 &= C_1(-\frac{dy_1}{dt} + r(t)x_1 + s(t)y_1) + C_2(-\frac{dy_2}{dt} + r(t)x_2 + q(t)y_2) \\
  0 &= C_1 * 0 + C_2 * 0 \\
  0 &= 0 \\
\end{align*}
Tämä todistaa väitteen.

\subsection*{(b)}

Täytyy osoittaa, että yhtälöpari
\[
  \begin{cases}
    \frac{d}{dt}(x_1 - x_2) = p(t)(x_1 - x_2) + q(t)(y_1 - y_2) \\
    \frac{d}{dt}(y_1 - y_2) = r(t)(x_1 - x_2) + s(t)(y_1 - y_2) \\
  \end{cases}
\]
toteutuu. Ensimmäisestä yhtälöstä saadaan
\begin{align*}
  \frac{dx_1}{dt} - \frac{dx_2}{dt} &= p(t)x_1 - p(t)x_2 + q(t)y_1 - q(t)y_2 \\
  0 &= (-\frac{dx_1}{dt} + p(t)x_1 + q(t)y_1) + (\frac{dx_2}{dt} - p(t)x_2 - q(t)y_2) \\
  \noalign{\text{Oletus:}} \\
  0 &= -f(t) + f(t) \\
  0 &= 0 \\
\end{align*}
Vastaavasti toisesta yhtälöstä saadaan
\begin{align*}
  \frac{dy_1}{dt} - \frac{dy_2}{dt} &= r(t)x_1 - r(t)x_2 + s(t)y_1 - s(t)y_2 \\
  0 &= (-\frac{dy_1}{dt} + r(t)x_1 + s(t)y_1) + (\frac{dx_2}{dt} - r(t)x_2 - s(t)y_2) \\
  0 &= -g(t) + g(t) \\
  0 &= 0 \qquad \qedsymbol \\
\end{align*}

\section*{9.}

\subsection*{(a)}

\[
  \begin{cases}
    \frac{dx}{dt} = x - xy \\
    \frac{dy}{dt} = y + 2xy \\
  \end{cases}
\]
Kriittisissä pisteissä $\frac{dx}{dt} = \frac{dy}{dt} = 0$.
Ensimmäisestä yhtälöstä
\begin{align*}
  x - xy &= 0 \\
  x(1 - y) &= 0 \\
  x &= 0 \vee y = 1 \\
\end{align*}
ja toisesta
\begin{align*}
  y + 2xy &= 0 \\
  y(1 + 2x) &= 0 \\
  y &= 0 \vee x = -\frac{1}{2}. \\
\end{align*}
Kriittiset pisteet ovat näiden pistejoukkojen leikkaus, eli pisteet
$(0, 0)$ ja $(-\frac{1}{2}, 1)$.

\subsection*{(b)}

\[
  \begin{cases}
    \frac{dx}{dt} = x - x^2 - xy \\
    \frac{dy}{dt} = \frac{1}{2}y - \frac{1}{4}y^2 - \frac{3}{4}xy \\
  \end{cases}
\]
Yhtälö 1:
\begin{align*}
  x - x^2 - xy &= 0 \\
  x(1 - x - y) &= 0 \\
  x &= 0 \vee y = -x + 1 \\
\end{align*}
Yhtälö 2:
\begin{align*}
  \frac{1}{2}y - \frac{1}{4}y^2 - \frac{3}{4}xy &= 0 \\
  y(\frac{1}{2} - \frac{1}{4}y - \frac{3}{4}x) &= 0 \\
  y &= 0 \vee y = -3x + 2 \\
\end{align*}
Kriittisiä pisteitä ovat $(0, 0)$, suorien
$y = -x + 1$ ja $y = -3x + 2$ leikkaupiste eli
\begin{align*}
  -x + 1 &= -3x + 2 \\
  2x &= 1 \\
  x &= \frac{1}{2} \\
\end{align*}
ja
$y = -\frac{1}{2} + 1 = \frac{1}{2}$.
Lisäksi kriittisiä pisteitä ovat pisteet $x = 0$, $y = -3*0 + 2 = 2$
ja $y = 0$, $0 = -x + 1 \iff x = 1$.

Siis kaikki kriittiset pisteet ovat $(0, 0)$, $(0, 2)$, $(\frac{1}{2}, \frac{1}{2})$
ja $(1, 0)$.

\section*{10.}

\[
  \begin{cases}
    \frac{dx}{dt} = x \\
    \frac{dy}{dt} = -2y \\
  \end{cases}
\]

\subsection*{(a)}

Systeemin yhtälöt ovat kokonaan erilliset, joten ne voidaan ratkaista erikseen.
Molemmat ovat myös separoituvia. Myös karakteristinen yhtälö toimisi, mutta
ratkaisen kertauksen vuoksi separoimalla. Ensimmäinen yhtälö:
\begin{align*}
  \frac{dx}{dt} &= x \\
  \text{Jos $x \neq 0$:} \\
  \frac{1}{x}\frac{dx}{dt} &= 1 \\
  \int \frac{1}{x}\,dx &= \int 1\,dt \\
  \ln |x| &= t + C_1, \quad C_1 \in \mathbb{R} \\
  x &= \pm e^{C_1}e^t \\
    &= C_2e^t, \quad C_2 \in \mathbb{R} \setminus \{0\} \\
\end{align*}
Koska $x \equiv 0$ on ratkaisu, niin
\[
  x = Ce^t, \quad C \in \mathbb{R}.
\]
Toinen yhtälö:
\begin{align*}
  \frac{dy}{dt} &= -2y \\
  \text{Jos $y \neq 0$:} \\
  \frac{1}{y}\frac{dy}{dt} &= -2 \\
  \int \frac{1}{y}\,dy &= -\int 2\,dt \\
  \ln |y| &= -2t + C_1, \quad C_1 \in \mathbb{R} \\
  y &= \pm e^{C_1}e^{-2t} \\
    &= C_2e^{-2t}, \quad C_2 \in \mathbb{R} \setminus \{0\} \\
\end{align*}
Koska $y \equiv 0$ on ratkaisu, niin
\[
  y = Ce^{-2t}, \quad C \in \mathbb{R}.
\]
Yhtälösysteemin ratkaisu on siis
\[
  \begin{cases}
    x = C_1e^t \\
    y = C_2e^{-2t} \\
  \end{cases}, C_1,C_2 \in \mathbb{R}.
\]
Eliminoidaan $t$:
\begin{align*}
  y &= C_2e^{-2t} \\
    &= \frac{C_2}{(e^t)^2} \\
    &= \frac{C_2}{(\frac{x}{C_1})^2} = \frac{C_1^2C_2}{x^2} \\
\end{align*}
Siis ratkaisukäyrät ovat muotoa $y = \frac{1}{x^2}$ kerrottuna vakiolla.
Kun $t$ kasvaa, niin $\frac{dx}{dt}$ kasvaa.
Parametrisoitu käyrä etenee siis vasemmalta oikealle.

\subsection*{(b)}

Kun $x \neq 0$, niin käyrän tangentin kulmakerroin on
\[
  \frac{dy}{dx} = \frac{\frac{dy}{dt}}{\frac{dx}{dt}} = \frac{-2y}{x}.
\]
Ratkaistaan yhtälö:
\begin{align*}
  \frac{dy}{dx} &= \frac{-2y}{x} \\
  \text{Jos $y \neq 0$:} \\
  \frac{1}{y}\frac{dy}{dx} &= -\frac{2}{x} \\
  \int \frac{1}{y}\,dy &= -2\int \frac{1}{x}\,dx \\
  \ln |y| &= -2\ln |x| + C_1, \quad C_1 \in \mathbb{R} \\
  |y| &= e^{-2\ln |x| + C_1} \\
      &= e^{C_1}(e^{\ln |x|})^{-2} \\
      &= C_2|x|^{-2}, \quad C_2 > 0 \\
  y &= \frac{C_3}{x^2}, \quad C_3 \in \mathbb{R} \setminus \{0\} \\
\end{align*}
Myös $y \equiv 0$ on ratkaisu, joten
\[
  y = \frac{C}{x^2}, \quad C \in \mathbb{R}.
\]

\end{document}

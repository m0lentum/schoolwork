%! TEX program = xelatex

\documentclass{article}
\usepackage[a4paper, margin=3cm]{geometry}
\setlength{\parindent}{0pt}
\setlength{\parskip}{1em}
%\usepackage{fontspec}
%\setmainfont{Lato}

\usepackage{amsmath,amssymb,amsthm}

\title{MATA114 Harjoitus 2}
\author{Mikael Myyrä}
\date{}

\begin{document}
\maketitle

\section*{1.}

\subsection*{(a)}

\begin{align*}
  \frac{dy}{dx} + 2y &= 6 \\
  \frac{dy}{dx} &= -2(y - 3) \\
  \text{Jos $y - 3 \neq 0$}: \\
  \frac{1}{y - 3} \frac{dy}{dx} &= -2 \\
  \int \frac{1}{y - 3} \,dy &= -\int 2 \, dx \\
  \ln |y - 3| &= -2x + C_1, \quad C_1 \in \mathbb{R} \\
  |y - 3| &= e^{C_1} e^{-2x} \\
          &= C_2 e^{-2x}, \quad C_2 > 0 \\
  y - 3 &= C_3 e^{-2x}, \quad C_3 \in \mathbb{R} \setminus \{0\} \\
  y &= C_3 e^{-2x} + 3 \\
\end{align*}

Jos $y - 3 = 0 \iff y = 3$, niin $\frac{dy}{dx} + 2y = 0 + 6 = 6$ ja yhtälö
toteutuu. Tämä vastaa yo. ratkaisussa tilannetta $C_3 = 0$. Siis yleinen
ratkaisu on
\[
  y = Ce^{-2x} + 3, \quad C \in \mathbb{R}.
\]

\subsection*{(b)}

Vastaava homogeeninen yhtälö on
\[
  \frac{dy}{dx} + 2y = 0.
\]
Ratkaistaan tämä separoimalla:
\begin{align*}
  \frac{dy}{dx} &= -2y \\
  \text{Jos $y \neq 0$:} \\
  \frac{1}{y} \frac{dy}{dx} &= -2 \\
  \int \frac{1}{y} \,dy &= -\int 2 \,dx \\
  \ln |y| &= -2x + C_1, \quad C_1 \in \mathbb{R} \\
  y &= \pm e^{C_1} e^{-2x} \\
    &= C_2 e^{-2x}, \quad C_2 \in \mathbb{R} \setminus \{0\} \\
\end{align*}

Jos $y = 0$, niin $\frac{dy}{dx} + 2y = 0 + 0 = 0$ ja yhtälö toteutuu.
Homogeenisen yhtälön yleinen ratkaisu on siis
\[
  y = Ce^{-2x}, \quad C \in \mathbb{R}.
\]

Kokeilemalla asettaa $\frac{dy}{dx} = 0$ löydetään epähomogeenisen yhtälön
yksittäisratkaisu $y = 3$. Sen yleinen ratkaisu on tämän ja homogeenisen
yhtälön yleisen ratkaisun summa
\[
  y = Ce^{-2x} + 3, \quad C \in \mathbb{R}.
\]

\subsection*{(c)}

Esimerkin yhtälö $\frac{dy}{dx} - 3y = x$ ei ole separoituva, koska
normaalimuodossa $\frac{dy}{dx} = x + 3y$ on x:n funktion ja y:n summa, ei
tulo. Jos tätä yrittäisi separoida, saataisiin
$\frac{1}{x + 3y} \frac{dy}{dx} = 1$, jota ei osata integroida.

\section*{2.}

Näissä $p(x)$ on vakio ja $q(x)$ on polynomi, joten epähomogeenisen yhtälön
yksittäisratkaisu voidaan etsiä arvaamalla.

\subsection*{(a)}

\[
  \frac{dy}{dx} + 5y = 2x - 1
\]

Ratkaistaan vastaava homogeeninen DY:
\begin{align*}
  \frac{dy}{dx} + 5y &= 0 \\
  \frac{dy}{dx} &= -5y \\
  \text{Jos $y \neq 0$}: \\
  \int \frac{1}{y} \,dy &= -\int 5 \,dx \\
  \ln |y| &= -5x + C_1, \quad C_1 \in \mathbb{R} \\
  y &= \pm e^{C_1} e^{-5x} \\
    &= C_2 e^{-5x}, \quad C_2 \in \mathbb{R} \setminus \{0\} \\
  \text{Koska $y = 0$ on ratkaisu:} \\
    &= Ce^{-5x}, \quad C \in \mathbb{R} \\
\end{align*}

Etsitään epähomogeenisen yhtälön yksittäisratkaisu sijoittamalla polynomiyrite
$y_e = Ax + B$ (jolle $\frac{dy_e}{dx} = A$):
\begin{align*}
  A + 5(Ax + B) &= 2x - 1 \\
  5Ax + A + 5B &= 2x - 1 \\
\end{align*}

Tästä saadaan yhtälöpari
\[
  \begin{cases}
    5A = 2 \\
    A + 5B = -1 \\
  \end{cases}
\]

jonka ratkaisu on
\[
  \begin{cases}
    A = \frac{2}{5} \\
    B = -\frac{7}{25}. \\
  \end{cases}
\]

Siis epähomogeenisen yhtälön yksittäisratkaisu on
\[
  y = \frac{2}{5}x - \frac{7}{25}
\]
ja yleinen ratkaisu tämän ja homogeenisen yhtälön yleisen ratkaisun summa
\[
  y = Ce^{-5x} + \frac{2}{5}x - \frac{7}{25}, \quad C \in \mathbb{R}.
\]

\subsection*{(b)}

\[
  \frac{dy}{dx} + 5y = x^2 - 1
\]

Vastaavan homogeenisen yhtälön ratkaisu on (kohdasta (a))
\[
  y = Ce^{-5x}, \quad C \in \mathbb{R}.
\]

Etsitään epähomogeenisen yhtälön yksittäisratkaisu sijoittamalla
$y_e = Ax^2 + Bx + C$, jolle $\frac{dy_e}{dx} = 2Ax + B$:
\begin{align*}
  2Ax + B + 5(Ax^2 + Bx + C) &= x^2 - 1 \\
  5Ax^2 + (2A + 5B)x + B + C &= x^2 - 1 \\
\end{align*}

Tästä saadaan yhtälöryhmä
\[
  \begin{cases}
    5A = 1 \iff A = \frac{1}{5} \\
    2A + 5B = 0 \implies \frac{2}{5} + 5B = 0 \iff B = -\frac{2}{25} \\
    B + C = -1 \implies -\frac{2}{25} + C = -1 \iff C = -\frac{23}{25}. \\
  \end{cases}
\]

Siis yhtälön yksittäisratkaisu on
\[
  y = \frac{1}{5}x^2 - \frac{2}{25}x - \frac{23}{25}
\]
ja yleinen ratkaisu
\[
  y = Ce^{-5x} + \frac{1}{5}x^2 - \frac{2}{25}x - \frac{23}{25}.
\]

\section*{3.}

\subsection*{(a)}

\[
  \frac{dy}{dx} - 2y = \sin x
\]

Ratkaistaan vastaava homogeeninen DY:
\begin{align*}
  \frac{dy}{dx} - 2y &= 0 \\
  \frac{dy}{dx} &= 2y \\
  \text{Jos $y \neq 0$:} \\
  \int \frac{1}{y} \,dy &= \int 2 \,dx \\
  \ln |y| &= 2x + C_1, \quad C_1 \in \mathbb{R} \\
  y &= \pm e^{C_1} e^{2x} \\
    &= C_2 e^{2x}, \quad C_2 \in \mathbb{R} \setminus \{0\} \\
  \text{Koska $y = 0$ on ratkaisu:} \\
    &= Ce^{2x}, \quad C \in \mathbb{R}. \\
\end{align*}

Etsitään yksittäisratkaisu sijoittamalla $y = A \sin x + B \cos x$
ja $\frac{dy}{dx} = A \cos x - B \sin x$.
\begin{align*}
  A \cos x - B \sin x - 2(A \sin x + B \cos x) &= \sin x \\
  (A - 2B) \cos x + (-2A - B) \sin x &= \sin x \\
\end{align*}

Saadaan yhtälöpari
\[
  \begin{cases}
    A - 2B = 0 \iff A = 2B \\
    -2A - B = 1 \implies -4B - B = 1 \iff B = -\frac{1}{5} \\
  \end{cases}
\]
jonka ratkaisu on
\[
  \begin{cases}
    A = -\frac{2}{5} \\
    B = -\frac{1}{5}. \\
  \end{cases}
\]
Siis eräs yksittäisratkaisu on
\[
  y = -\frac{2}{5} \sin x - \frac{1}{5} \cos x
\]
ja yleinen ratkaisu
\[
  y = Ce^{2x} - \frac{2}{5} \sin x - \frac{1}{5} \cos x, \quad C \in \mathbb{R}.
\]

\subsection*{(b)}

\[
  \frac{dy}{dx} - 2y = 3 \cos x
\]

Vastaavan homogeenisen yhtälön ratkaisu on (a)-kohdasta $Ce^{2x}$.

Samoin kuin (a):ssa, etsitään yksittäisratkaisu sijoittamalla $y = A \sin x + B
\cos x$ ja $\frac{dy}{dx} = A \cos x - B \sin x$.
\begin{align*}
  A \cos x - B \sin x - 2(A \sin x + B \cos x) &= 3 \cos x \\
  (A - 2B) \cos x + (-2A - B) \sin x &= 3 \cos x \\
\end{align*}

Saadaan yhtälöpari
\[
  \begin{cases}
    -2A - B = 0 \iff B = -2A \\
    A - 2B = 3 \implies 3A = 3 \iff A = 1 \\
  \end{cases}
\]
jonka ratkaisu on
\[
  \begin{cases}
    A = 1 \\
    B = -2. \\
  \end{cases}
\]
Siis eräs yksittäisratkaisu on
\[
  y = \sin x - 2 \cos x
\]
ja yleinen ratkaisu
\[
  y = Ce^{2x} + \sin x - 2 \cos x, \quad C \in \mathbb{R}.
\]

\subsection*{(c)}

\[
  \frac{dy}{dx} - 2y = 10 \sin 3x
\]

Vastaavan homogeenisen yhtälön ratkaisu on (a)-kohdasta $Ce^{2x}$.

Etsitään yksittäisratkaisu sijoittamalla $y = A \sin 3x + B \cos 3x$ ja
$\frac{dy}{dx} = 3A \cos 3x - 3B \sin 3x$.
\begin{align*}
  3A \cos 3x - 3B \sin 3x - 2(A \sin 3x + B \cos 3x) &= 10 \sin 3x \\
  (3A - 2B) \cos 3x + (-3B - 2A) \sin 3x &= 10 \sin 3x \\
\end{align*}

Saadaan yhtälöpari
\[
  \begin{cases}
    3A - 2B = 0 \iff A = \frac{2}{3}B \\
    -3B - 2A = 10 \implies -\frac{13}{3}B = 10 \iff B = -\frac{30}{13} \\
  \end{cases}
\]
jonka ratkaisu on
\[
  \begin{cases}
    A = -\frac{60}{13} \\
    B = -\frac{30}{13}. \\
  \end{cases}
\]
Siis eräs yksittäisratkaisu on
\[
  y = -\frac{60}{13} \sin 3x - \frac{30}{13} \cos 3x
\]
ja yleinen ratkaisu
\[
  y = Ce^{2x} - \frac{60}{13} \sin 3x - \frac{30}{13} \cos 3x, \quad C \in \mathbb{R}.
\]

\section*{4.}

Vastaava homogeeninen yhtälö on kaikissa kohdissa sama, joka ratkaistiin jo
edellisestä tehtävässä. Ratkaisu on $y = Ce^{2x}, \quad C \in \mathbb{R}$.

\subsection*{(a)}

\[
  \frac{dy}{dx} - 2y = 2e^x
\]

Etsitään yksittäisratkaisu sijoittamalla $y = Ae^x$ ja $\frac{dy}{dx} = Ae^x$.
\begin{align*}
  Ae^x - 2Ae^x &= 2e^x \\
  -Ae^x &= 2e^x \\
  A &= -2 \\
\end{align*}

Saadaan yksittäisratkaisu
\[
  y = -2e^x
\]
ja yleinen ratkaisu
\[
  y = Ce^{2x} - 2e^x, \quad C \in \mathbb{R}.
\]

\subsection*{(b)}

\[
  \frac{dy}{dx} - 2y = e^{2x}
\]

Etsitään yksittäisratkaisu sijoittamalla $y = Ae^{2x}$ ja $\frac{dy}{dx} = 2Ae^{2x}$.
\begin{align*}
  2Ae^{2x} - 2Ae^{2x} &= e^{2x} \\
  0 &= e^{2x} \\
\end{align*}

Tämä ei toiminutkaan, koska yrite hävitti yhtälön vasemman puolen. Kokeillaan
sen sijaan yritettä $y = Axe^{2x}$ ja $\frac{dy}{dx} = A(e^{2x} + 2xe^{2x}) =
(2Ax + A)(e^{2x})$.
\begin{align*}
  (2Ax + A)(e^{2x}) - 2Axe^{2x} &= e^{2x} \\
  Ae^{2x} &= e^{2x} \\
  A &= 1 \\
\end{align*}

Saadaan yksittäisratkaisu
\[
  y = xe^{2x}
\]
ja yleinen ratkaisu
\[
  y = Ce^{2x} + xe^{2x}, \quad C \in \mathbb{R}.
\]

\subsection*{(c)}

\[
  \frac{dy}{dx} - 2y = 2x + e^{2x}
\]

Tähän voisi toimia yrite $y = Ax + B + Cxe^{2x}$ ja
$\frac{dy}{dx} = A + (2Cx + C)e^{2x}$.
\begin{align*}
  A + (2Cx + C)e^{2x} - 2(Ax + B + Cxe^{2x}) &= 2x + e^{2x} \\
  -2Ax + Ce^{2x} + A - 2B &= 2x + e^{2x} \\
\end{align*}

Saadaan yhtälöryhmä
\[
  \begin{cases}
    -2A = 2 \iff A = -1 \\
    C = 1 \\
    A - 2B = 0 \implies -1 - 2B = 0 \iff B = -\frac{1}{2} \\
  \end{cases}
\]
Tästä saadaan yksittäisratkaisu
\[
  y = xe^{2x} - x - \frac{1}{2}
\]
ja yleinen ratkaisu
\[
  y = Ce^{2x} + xe^{2x} - x - \frac{1}{2}, \quad C \in \mathbb{R}.
\]

\subsection*{(d)}

\[
  \frac{dy}{dx} - 2y = e^{3x} - 5e^{2x}
\]

Kokeillaan yritettä $y = Ae^{3x} + Bxe^{2x}$,
$\frac{dy}{dx} = 3Ae^{3x} + (2Bx + B)e^{2x}$.
\begin{align*}
  3Ae^{3x} + (2Bx + B)e^{2x} - 2(Ae^{3x} + Bxe^{2x}) &= e^{3x} - 5e^{2x} \\
  Ae^{3x} + Be^{2x} &= e^{3x} - 5e^{2x} \\
\end{align*}

Tästä nähdään suoraan, että $A = 1$ ja $B = -5$, joten saadaan yksittäisratkaisu
\[
  y = e^{3x} - 5xe^{2x}.
\]
ja yleinen ratkaisu
\[
  y = Ce^{2x} + e^{3x} - 5xe^{2x}, \quad C \in \mathbb{R}.
\]

\subsection*{Lisätehtävä}

(b)-kohta teki konkreettiseksi, miksi luentomateriaalin luvun 3.4 taulukossa on
erikseen $e^{kx}$:lle tapaus $k = -a$. (c)- ja (d)-kohdista opin, että
yritteitä on helppo yhdistellä summaamalla, jos q(x) on helppojen funktioiden
summa.  Käytännön harjoitus auttoi hahmottamaan, miten yrite pitää valita, että
$q(x)$:ään kuulumattomia termejä ei synny tai ne kumoavat toisensa.

\section*{5.}

Tässäkin tehtävässä jokaisen kohdan vastaava homogeeninen yhtälö on sama,
$\frac{dy}{dx} + 2y = 0$.  Sen ratkaisu on $y = Ce^{-2x}, \quad C \in \mathbb{R}$.

\subsection*{(a)}

\[
  \begin{cases}
    \frac{dy}{dx} + 2y = e^{-5x} \\
    y(0) = 4 \\
  \end{cases}
\]

Etsitään yksittäisratkaisu sijoittamalla $y = Ae^{-5x}$:
\begin{align*}
  -5Ae^{-5x} + 2Ae^{-5x} &= e^{-5x} \\
  -3Ae^{-5x} &= e^{-5x} \\
  -3A &= 1 \\
  A &= -\frac{1}{3} \\
\end{align*}

Saadaan yksittäisratkaisu $y = -\frac{1}{3}e^{-5x}$ ja yleinen ratkaisu
$y = Ce^{-2x} - \frac{1}{3}e^{-5x}, \quad C \in \mathbb{R}$.
Sijoitetaan alkuehto $y(0) = 4$:
\begin{align*}
  Ce^{-2*0} - \frac{1}{3}e^{-5*0} &= 4 \\
  C - \frac{1}{3} &= 4 \\
  C &= \frac{13}{3} \\
\end{align*}

Tästä saadaan alkuehdon toteuttava yhtälö
\[
  y = \frac{13}{3}e^{-2x} - \frac{1}{3}e^{-5x}, \quad C \in \mathbb{R}.
\]

\subsection*{(b)}

\[
  \begin{cases}
    \frac{dy}{dx} + 2y = e^{-2x} \\
    y(0) = 2 \\
  \end{cases}
\]

Etsitään yksittäisratkaisu sijoittamalla $y = Axe^{-2x}$ ja
$\frac{dy}{dx} = (-2Ax + A)e^{-2x}$:
\begin{align*}
  (-2Ax + A)e^{-2x} + 2Axe^{-2x} &= e^{-2x} \\
  Ae^{-2x} &= e^{-2x} \\
  A &= 1 \\
\end{align*}

Saadaan yksittäisratkaisu
\[
  y = xe^{-2x}
\]
ja yleinen ratkaisu
\[
  y = Ce^{-2x} + xe^{-2x}, \quad C \in \mathbb{R}.
\]

Sijoitetaan alkuehto $y(0) = 2$:
\begin{align*}
  Ce^{-2*0} + 0*e^{-2*0} &= 2 \\
  C &= 2 \\
\end{align*}

Saadaan alkuehdon toteuttava yhtälö
\[
  y = 2e^{-2x} + xe^{-2x}, \quad C \in \mathbb{R}.
\]

\subsection*{(c)}

\[
  \begin{cases}
    \frac{dy}{dx} + 2y = e^{-5x} + e^{-2x} \\
    y(0) = 6 \\
  \end{cases}
\]

Etsitään yksittäisratkaisu sijoittamalla $y = Ae^{-5x} + Bxe^{-2x}$:
\begin{align*}
  -5Ae^{-5x} + (-2Bx + B)e^{-2x} + 2(Ae^{-5x} + Bxe^{-2x}) &= e^{-5x} + e^{-2x} \\
  -3Ae^{-5x} + Be^{-2x} &= e^{-5x} + e^{-2x} \\
  \begin{cases}
    -3A = 1 \iff A = -\frac{1}{3} \\
    B = 1 \\
  \end{cases}
\end{align*}

Saadaan yksittäisratkaisu
\[
  y = -\frac{1}{3}e^{-5x} + xe^{-2x}
\]
ja yleinen ratkaisu
\[
  y = Ce^{-2x} - \frac{1}{3}e^{-5x} + xe^{-2x}, \quad C \in \mathbb{R}.
\]

Sijoitetaan alkuehto $y(0) = 6$:
\begin{align*}
  Ce^{-2*0} - \frac{1}{3}e^{-5*0} + 0e^{-2*0} &= 6 \\
  C - \frac{1}{3} &= 6 \\
  C &= \frac{19}{3} \\
\end{align*}

Saadaan alkuehdon toteuttava yhtälö
\[
  y = \frac{19}{3}e^{-2x} - \frac{1}{3}e^{-5x} + xe^{-2x}, \quad C \in \mathbb{R}.
\]

\section*{6.}

\[
  \frac{dy}{dx} + xy = x^3
\]

\subsection*{(a)}

Ratkaistaan vastaava homogeeninen yhtälö:
\begin{align*}
  \frac{dy}{dx} + xy &= 0 \\
  \frac{dy}{dx} &= -xy \\
  \text{Jos $y \neq 0$:} \\
  \int \frac{1}{y} \,dy &= -\int x \,dx \\
  \ln |y| &= -\frac{1}{2}x^2 + C_1, \quad C_1 \in \mathbb{R} \\
  y &= \pm e^{C_1} e^{-\frac{1}{2}x^2} \\
    &= C_2 e^{-\frac{1}{2}x^2}, \quad C_2 \in \mathbb{R} \setminus \{0\} \\
  \text{Koska $y = 0$ on ratkaisu:} \\
    &= Ce^{-\frac{1}{2}x^2}, \quad C \in \mathbb{R} \\
\end{align*}

Etsitään yksittäisratkaisu vakion varioinnin avulla:
merkitään $C = C(x)$. Tällöin
\[
  \frac{dy}{dx} = \frac{dC}{dx}e^{-\frac{1}{2}x^2} - Cxe^{-\frac{1}{2}x^2}.
\]
Sijoitetaan yhtälöön:
\begin{align*}
  \frac{dC}{dx}e^{-\frac{1}{2}x^2} - Cxe^{-\frac{1}{2}x^2} + xCe^{-\frac{1}{2}x^2} &= x^3 \\
  \frac{dC}{dx}e^{-\frac{1}{2}x^2} &= x^3 \\
  \frac{dC}{dx} &= \frac{x^3}{e^{-\frac{1}{2}x^2}} \\
                &= x^3 e^{\frac{1}{2}x^2} \\
                &= x^2(xe^{\frac{1}{2}x^2}) \\
\end{align*}

Integroidaan osittaisintegroimalla:
\begin{align*}
  C &= \int x^2(xe^{\frac{1}{2}x^2}) \,dx \\
    &= x^2e^{\frac{1}{2}x^2} - \int 2xe^{\frac{1}{2}x^2} \,dx \\
    &= x^2e^{\frac{1}{2}x^2} - 2e^{\frac{1}{2}x^2} + D, \quad D \in \mathbb{R} \\
    &= (x^2 - 2)e^{\frac{1}{2}x^2} + D \\
\end{align*}

Koska mikä tahansa yksittäisratkaisu sopii, voidaan valita tästä $D = 0$.
Yksittäisratkaisu saadaan sijoittamalla:
\begin{align*}
  y &= Ce^{-\frac{1}{2}x^2} \\
    &= (x^2 - 2)e^{\frac{1}{2}x^2}e^{-\frac{1}{2}x^2} \\
    &= x^2 - 2 \\
\end{align*}

Nyt yleinen ratkaisu on
\[
  y = Ce^{-\frac{1}{2}x^2} + x^2 - 2, \quad C \in \mathbb{R}.
\]

\subsection*{(b)}

Tässä $y$:n kerroinfunktio $p(x) = x$.
Integroiva tekijä on muotoa $e^{\int p\,dx}$.
Nyt integroivaksi tekijäksi saadaan $e^{\frac{1}{2}x^2}$.
Kerrotaan alkuperäinen yhtälö tällä ja ratkaistaan:
\begin{align*}
  e^{\frac{1}{2}x^2}(\frac{dy}{dx} + xy) &= e^{\frac{1}{2}x^2}x^3 \\
  \frac{d}{dx}(e^{\frac{1}{2}x^2}y) &= e^{\frac{1}{2}x^2}x^3 \\
  e^{\frac{1}{2}x^2}y &= \int e^{\frac{1}{2}x^2}x^3 \,dx \\
                      &= (x^2 - 2)e^{\frac{1}{2}x^2} + C, \quad C \in \mathbb{R} \text{ (laskettu (a)-kohdassa)} \\
  y &= x^2 - 2 + \frac{C}{e^{\frac{1}{2}x^2}} \\
    &= Ce^{-\frac{1}{2}x^2} + x^2 - 2 \\
\end{align*}

\subsection*{(c)}

Molemmissa menetelmissä on puolensa. Integroivan tekijän idea seuraa melko
suoraan integrointisäännöistä, minkä vuoksi se tuntuu helpommalta "keksiä
itse", mutta vakion variointi tuntuu kuitenkin jostain syystä helpommalta
hahmottaa ja muistaa. Ehkä siksi, että se on jatkoa homogeenisen yhtälön
ratkaisun ja yritteen arvaamisen menetelmään, jota tuli jo harjoiteltua
useamman tehtävän verran. Pitäisi laskea integroivalla tekijällä enemmän,
että voisin päättää, kumpaa oikeasti käytän mieluummin.

\section*{7.}

\subsection*{(a)}

\[
  \frac{dy}{dx} - \frac{2y}{x} = x^2
\]

Integroiva tekijä: $f(x) = e^{\int -\frac{2}{x} \,dx} = e^{-2 \ln |x|}
= |x|^{-2} = \frac{1}{x^2}$.
Kerrotaan yhtälö tällä ja ratkaistaan.
\begin{align*}
  \frac{1}{x^2} (\frac{dy}{dx} - \frac{2y}{x}) &= \frac{1}{x^2}x^2 \\
  \frac{d}{dx}(\frac{1}{x^2}y) &= 1 \\
  \frac{1}{x^2}y &= \int 1 \,dx \\
                 &= x + C, \quad C \in \mathbb{R} \\
  y &= x^3 + Cx^2 \\
\end{align*}

\subsection*{(b)}

\[
  \frac{dy}{dx} + \frac{2y}{x} = \frac{1}{x^2}
\]

Integroiva tekijä: $f(x) = e^{\int \frac{2}{x}\,dx} = e^{2 \ln |x|}
= |x|^2 = x^2$.
Kerrotaan yhtälö tällä ja ratkaistaan.
\begin{align*}
  x^2(\frac{dy}{dx} + \frac{2y}{x}) &= x^2\frac{1}{x^2} \\
  \frac{d}{dx}(x^2y) &= 1 \\
  x^2y &= \int 1\,dx \\
       &= x + C, \quad C \in \mathbb{R} \\
  y &= \frac{1}{x} + \frac{C}{x^2} \\
\end{align*}

\subsection*{(c)}

\[
  \frac{dy}{dx} + 2e^xy = e^x
\]

Integroiva tekijä: $f(x) = e^{\int 2e^x\,dx} = e^{2e^x}$.
Kerrotaan ja ratkaistaan.
\begin{align*}
  e^{2e^x}(\frac{dy}{dx} + 2e^xy) &= e^{2e^x}e^x \\
  \frac{d}{dx}(e^{2e^x}y) &= e^{2e^x + x} \\
  e^{2e^x}y &= \int e^{2e^x + x}\,dx \\
            &= (2e^x + 1)e^{2e^x + x} + C, \quad C \in \mathbb{R} \\
  y &= (2e^x + 1)e^x + Ce^{-2e^x} \\
    &= Ce^{-2e^x} + 2e^{2x} + e^x \\
\end{align*}

\subsection*{8.}

\[
  \begin{cases}
    \frac{dy}{dx} - y = -2x + 2 \\
    y(0) = 2 \\
  \end{cases}
\]

\subsection*{(a)}

Tässä $p(x) = -1$, $P(x) = -x$, $q(x) = -2x + 2$, $x_0 = 0$ ja $y_0 = 2$.

\begin{align*}
  y &= y_0e^{-P(x)} + e^{-P(x)} \int_{x_0}^{x} e^{P(s)}q(s)\,ds \\
    &= 2e^x + e^x \int_0^x e^{-s}(-2s + 2)\,ds \\
    &= 2e^x + e^x \Big[-2(-se^{-s} + e^{-s}) - 2e^{-s}\Big]_{s=0}^{s=x} \\
    &= 2e^x + e^x (2xe^{-x} - 4e^{-x} + 4) \\
    &= 2e^x + 2x - 4 + 4e^x \\
    &= 6e^x + 2x - 4 \\
\end{align*}

\subsection*{(b)}

Yleinen ratkaisu integroivan tekijän $f(x) = e^{-x}$ avulla.
\begin{align*}
  e^{-x}(\frac{dy}{dx} - y) &= e^{-x}(-2x + 2) \\
  \frac{d}{dx}(e^{-x}y) &= -2xe^{-x} + 2e^{-x} \\
  e^{-x}y &= \int -2xe^{-x} + 2e^{-x}\,dx \\
          &= -2(-xe^{-x} + e^{-x}) -2e^{-x} + C, \quad C \in \mathbb{R} \\
          &= 2xe^{-x} - 2e^{-x} - 2e^{-x} + C \\
          &= 2xe^{-x} - 4e^{-x} + C \\
  y &= 2x - 4 + Ce^x \\
\end{align*}

Sijoitetaan alkuehto $y(0) = 2$:
\begin{align*}
  2*0 - 4 + Ce^0 &= 2 \\
  C - 4 &= 2 \\
  C &= 6 \\
\end{align*}

Saadaan alkuarvotehtävän ratkaisu
\[
  y = 6e^x + 2x - 4.
\]

\subsection*{Lisätehtävä}

Ratkaisukaavoissa, joilla saa suoraan laskettua tehtävän loppuun asti, on
tiettyä eleganssia (ohjelmoijana tulee mieleen, että sellainen olisi helppo
muuttaa koodiksi). Käsin laskiessa kuitenkin tällaisten kaavojen käyttö tuntuu
ulkoa opettelulta, ja pidän enemmän keinoista, jotka on helppo muistaa, kun
ymmärtää periaatteen. Myös huolimattomuusvirheiden vaara on suurempi
monimutkaisemman kaavan käsin laskennassa, kuten huomasin tätäkin tehtävää
tehdessäni. Siksi käytän mieluummin jälkimmäistä tapaa.

\end{document}

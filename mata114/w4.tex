%! TEX program = xelatex

\documentclass{article}
\usepackage[a4paper, margin=3cm]{geometry}
\setlength{\parindent}{0pt}
\setlength{\parskip}{1em}
%\usepackage{fontspec}
%\setmainfont{Lato}

\usepackage{amsmath,amssymb,amsthm}
%\usepackage{pgfplots}
%\pgfplotsset{compat=1.16}

\title{MATA114 Harjoitus 4}
\author{Mikael Myyrä}
\date{}

\begin{document}
\maketitle

\section*{1.}

\subsection*{(a)}

\[
  \frac{d^2 y}{d x^2} - 2\frac{dy}{dx} - 3y = 0
\]

Karakteristinen yhtälö:
\begin{align*}
  r^2 - 2r - 3 &= 0 \\
  r &= \frac{2 \pm \sqrt{4 + 12}}{2} \\
  r &= 3 \vee r = -1 \\
\end{align*}
Ratkaisut ovat reaaliset ja erisuuret, joten yhtälön yleinen ratkaisu on
\[
  y = Ae^{3x} + Be^{-x}, \quad A, B \in \mathbb{R}.
\]

\subsection*{(b)}

\[
  \frac{d^2 y}{d x^2} - 2\frac{dy}{dx} + y = 0
\]

Karakteristinen yhtälö:
\begin{align*}
  r^2 - 2r + r &= 0 \\
  r &= \frac{2 \pm \sqrt{4 - 4}}{2} \\
    &= 1 \\
\end{align*}
Ratkaisu on reaalinen ja kaksinkertainen, joten yhtälön yleinen ratkaisu on
\[
  y = Ae^x + Bxe^x, \quad A, B \in \mathbb{R}.
\]

\subsection*{(c)}

\[
  \frac{d^2 y}{d x^2} - 2\frac{dy}{dx} + 5y = 0
\]

Karakteristinen yhtälö:
\begin{align*}
  r^2 - 2r + 5 &= 0 \\
  r &= \frac{2 \pm \sqrt{4 - 20}}{2} \\
    &= 1 \pm 2i \\
\end{align*}
Ratkaisu on kompleksinen, joten yhtälön yleinen ratkaisu on
\[
  y = Ae^x\cos(2x) + Be^x\sin(2x), \quad A, B \in \mathbb{R}.
\]

\section*{2.}

\subsection*{(a)}

\[
  \frac{d^2 y}{d x^2} + 8\frac{dy}{dx} + 16y = 0
\]

Karakteristinen yhtälö:
\begin{align*}
  r^2 + 8r + 16 &= 0 \\
  r &= \frac{-8 \pm \sqrt{64 - 64}}{2} \\
    &= -4 \\
\end{align*}
Yhtälön ratkaisu:
\[
  y = Ae^{-4x} + Bxe^{-4x}, \quad A,B \in \mathbb{R}
\]

\subsection*{(b)}

\[
  9\frac{d^2 y}{d x^2} + 6\frac{dy}{dx} + y = 0
\]

Karakteristinen yhtälö:
\begin{align*}
  9r^2 + 6r + 1 &= 0 \\
  r &= \frac{-6 \pm \sqrt{36 - 36}}{18} \\
    &= -\frac{1}{3} \\
\end{align*}
Yhtälön ratkaisu:
\[
  y = Ae^{-\frac{1}{3}x} + Bxe^{-4x}, \quad A,B \in \mathbb{R}
\]

\subsection*{(c)}

\[
  \frac{d^2 y}{d x^2} + \frac{dy}{dx} + y = 0
\]

Karakteristinen yhtälö:
\begin{align*}
  r^2 + r + 1 &= 0 \\
  r &= \frac{-1 \pm \sqrt{1 - 4}}{2} \\
    &= -\frac{1}{2} \pm \frac{\sqrt{3}}{2}i \\
\end{align*}
Yhtälön ratkaisu:
\[
  y = Ae^{-\frac{1}{2}x}\sin(\frac{\sqrt{3}}{2}x)
    + Be^{-\frac{1}{2}x}\cos(\frac{\sqrt{3}}{2}x), \quad A,B \in \mathbb{R}
\]

\section*{3.}

\subsection*{(a)}

\[
  \frac{d^2 y}{d x^2} - 7\frac{dy}{dx} = 0
\]
Karakteristinen yhtälö toimii tähänkin, koska tämä on edelleen vakiokertoiminen
lineaarinen homogeeninen DY, vaikka y:n kerroin on nolla.
\begin{align*}
  r^2 - 7r &= 0 \\
  \implies r &= 0 \vee r = 7 \\
\end{align*}
Yhtälön ratkaisu:
\[
  y = A + Be^{7x}, \quad A,B \in \mathbb{R}
\]

\subsection*{(b)}

\[
  2\frac{d^2 y}{d x^2} - (\frac{dy}{dx})^2 = 0
\]

Muuttujanvaihdolla $u = \frac{dy}{dx}$ saadaan separoituva ensimmäisen
kertaluvun DY.
\begin{align*}
  2\frac{du}{dx} - u^2 &= 0 \\
  \text{Jos $u \neq 0$:} \\
  \frac{2}{u^2}\frac{du}{dx} &= 1 \\
  \int \frac{2}{u^2}\,du &= \int 1\,dx \\
  -\frac{2}{u} &= x + A, \quad A \in \mathbb{R} \\
  u &= -\frac{2}{x + A} \\
  \text{Tai} \\
  u &\equiv 0
\end{align*}

$y$ saadaan integroimalla:
\[
  y = \int u\,dx = -2\ln|x + A| + B, \quad A,B \in \mathbb{R}
\]

\section*{4.}

\subsection*{(a)}

\[
  \begin{cases}
    2\frac{d^2 y}{d x^2} + 5\frac{dy}{dx} - 3y = 0 \\
    y(0) = 1 \\
    \frac{dy}{dx}(0) = 0 \\
  \end{cases}
\]

Ratkaistaan DY karakteristisen yhtälön avulla:
\begin{align*}
  2r^2 + 5r - 3 &= 0 \\
  r &= \frac{-5 \pm \sqrt{25 + 24}}{4} \\
  r &= \frac{1}{2} \vee r = -3 \\
\end{align*}

Yhtälön ratkaisu on
\[
  y = Ae^{\frac{1}{2}x} + Be^{-3x}, \quad A,B \in \mathbb{R}.
\]
Sijoitetaan ensimmäinen alkuehto $y(0) = 1$:
\begin{align*}
  Ae^{\frac{1}{2}*0} + Be^{-3*0} &= 1 \\
  A + B &= 1 \\
\end{align*}

Sijoitetaan toinen alkuehto $\frac{dy}{dx}(0) = 0$ y:n derivaattaan:
\begin{align*}
  \frac{1}{2}Ae^{\frac{1}{2}*0} - 3Be^{-3*0} &= 0 \\
  \frac{1}{2}A - 3B &= 0 \\
\end{align*}

Saadaan yhtälöpari
\[
  \begin{cases}
    A + B = 1 \\
    \frac{1}{2}A - 3B = 0 \\
  \end{cases}
\]
jonka ratkaisu on $A = \frac{6}{7}$ ja $B = \frac{1}{7}$.
Siis alkuarvotehtävän ratkaisu on
\[
  y = \frac{6}{7}e^{\frac{1}{2}x} + \frac{1}{7}e^{-3x}.
\]

\section*{5.}

\[
  \begin{cases}
    2\frac{d^2 y}{d x^2} + \frac{dy}{dx} - 10y = 0 \\
    y(1) = 5 \\
    \frac{dy}{dx}(1) = 2 \\
  \end{cases}
\]

\subsection*{(a)}

Samalla tavalla kuin edellisessä tehtävässä, ratkaistaan DY
karakteristisen yhtälön avulla:
\begin{align*}
  2r^2 + r - 10 &= 0 \\
  r &= \frac{-1 \pm \sqrt{1 + 80}}{4} \\
  r &= 2 \vee r = -\frac{5}{2} \\
\end{align*}
Yhtälön ratkaisu on
\[
  y = Ae^{2x} + Be^{-\frac{5}{2}x}, \quad A, B \in \mathbb{R}.
\]

Sijoitetaan $y(1) = 5$:
\[
  Ae^{2} + Be^{-\frac{5}{2}} = 5
\]
ja $\frac{dy}{dx}(1) = 2$:
\[
  2Ae^{2} - \frac{5}{2}Be^{-\frac{5}{2}} = 2
\]
Ensimmäisestä saadaan
\[
  A = \frac{5 - Be^{-\frac{5}{2}}}{e^2}
\]
ja sijoittamalla tämä toiseen
\begin{align*}
  2e^2(\frac{5 - Be^{-\frac{5}{2}}}{e^2}) - \frac{5}{2}Be^{-\frac{5}{2}} &= 2 \\
  10 - 2Be^{-\frac{5}{2}} - \frac{5}{2}Be^{-\frac{5}{2}} &= 2 \\
  -\frac{9}{2}e^{-\frac{5}{2}}B &= -8 \\
  B &= \frac{16}{9}e^{\frac{5}{2}} \\
\end{align*}
Siis
\begin{align*}
  A &= \frac{5 - \frac{16}{9}}{e^2} = \frac{29}{9}e^{-2} \\
\end{align*}
ja tehtävän ratkaisu on
\begin{align*}
  y &= \frac{29}{9}e^{-2}e^{2x} + \frac{16}{9}e^{\frac{5}{2}}e^{-\frac{5}{2}x} \\
    &= \frac{29}{9}e^{2x - 2} + \frac{16}{9}e^{\frac{5}{2}(1 - x)}. \\
\end{align*}

\subsection*{(b)}

Jatketaan vinkin kanssa karakteristisen yhtälön ratkaisusta
$r = 2 \vee r = -\frac{5}{2}$:
\[
  y = Ae^{2(x-1)} + Be^{-\frac{5}{2}(x-1)}
\]
Kun tähän sijoitetaan alkuehdot, saadaan
\begin{align*}
  Ae^{2(1-1)} + Be^{-\frac{5}{2}(1-1)} &= 5 \\
  A + B &= 5
\end{align*}
ja
\begin{align*}
  2Ae^{2(1-1)} - \frac{5}{2}Be^{-\frac{5}{2}(1-1)} &= 2 \\
  2A - \frac{5}{2}B &= 2 \\
\end{align*}
Tästä saadaan $A = \frac{29}{9}$, $B = \frac{16}{9}$, ja
\[
  y = \frac{29}{9}e^{2(x-1)} + \frac{16}{9}e^{-\frac{5}{2}(x-1)}.
\]

\section*{6.}

\subsection*{(a)}

\[
  \frac{d^2 y}{d x^2} - 3\frac{dy}{dx} + 2y = 4x
\]

Ratkaistaan vastaava HY karakteristisen yhtälön avulla:
\begin{align*}
  r^2 - 3r + 2 &= 0 \\
  r &= \frac{3 \pm \sqrt{9 - 8}}{2} \\
  r &= 2 \vee r = 1 \\
\end{align*}
HY:n ratkaisu on
\[
  C_1e^x + C_2e^{2x}, \quad C_1,C_2 \in \mathbb{R}.
\]
Etsitään epähomogeenisen yhtälön yksittäisratkaisu käyttäen yritettä
$y = Ax + B$ ($\frac{dy}{dx} = A$, $\frac{d^2 y}{d x^2} = 0$):
\begin{align*}
  0 - 3A + 2(Ax + B) &= 4x \\
  2Ax - 3A + 2B &= 4x \\
\end{align*}
Saadaan yhtälöpari
\[
  \begin{cases}
    2A = 4 \\
    -3A + 2B = 0 \\
  \end{cases}
\]
ja ratkaisu
\[
  \begin{cases}
    A = 2 \\
    B = 3. \\
  \end{cases}
\]
Näistä saadaan DY:n yleinen ratkaisu
\[
  y = C_1e^x + C_2e^{2x} + 2x + 3.
\]

\section*{7.}

\subsection*{(a)}

\[
  \frac{d^2 y}{d x^2} - y = 4e^x
\]

Ratkaistaan vastaava HY karakteristisen yhtälön avulla:
\begin{align*}
  r^2 - 1 &= 0 \\
  r &= \sqrt{1} = \pm 1 \\
\end{align*}
HY:n ratkaisu on
\[
  y = C_1e^x + C_2e^{-x}, \quad C_1,C_2 \in \mathbb{R}.
\]
Etsitään epähomogeenisen yhtälön yksittäisratkaisu käyttäen yritettä
$y = Axe^x$ ($\frac{dy}{dx} = A(x + 1)e^x$, $\frac{d^2 y}{d x^2} = A(x + 2)e^x$).

($y = Ae^x$ ei toimi, koska se on HY:n ratkaisu.)
\begin{align*}
  A(x + 2)e^x - Axe^x &= 4e^x \\
  2Ae^x &= 4e^x \\
  A &= 2 \\
\end{align*}

DY:n yleinen ratkaisu on siis
\[
  y = C_1e^x + C_2e^{-x} + 2xe^x, \quad C_1,C_2 \in \mathbb{R}.
\]

\section*{8.}

\subsection*{(a)}

\[
  \begin{cases}
    \frac{d^2 y}{d x^2} - 4\frac{dy}{dx} + 4y = e^{2x} \\
    y(0) = 1 \\
    \frac{dy}{dx}(0) = 0 \\
  \end{cases}
\]

Ratkaistaan DY. Ratkaistaan ensin vastaava HY karakteristisen yhtälön
avulla:
\begin{align*}
  r^2 - 4r + 4 &= 0 \\
  r &= \frac{4 \pm \sqrt{16 - 16}}{2} \\
    &= 2 \\
\end{align*}
HY:n ratkaisu on
\[
  y = C_1xe^{2x} + C_2e^{2x}, \quad C_1,C_2 \in \mathbb{R}.
\]

Etsitään epähomogeenisen yhtälön yksittäisratkaisu.
Koska $Ae^{2x}$ ja $Axe^{2x}$ ovat HY:n ratkaisuja, valitaan
yritteeksi $y = Ax^2e^{2x}$, jolle $\frac{dy}{dx} = A(2x + 2x^2)e^{2x}$
ja $\frac{d^2 y}{d x^2} = A(2 + 4x + 2(2x + 2x^2))e^{2x}
= A(2 + 8x + 4x^2)e^{2x}$.
\begin{align*}
  A(2 + 8x + 4x^2)e^{2x} - 4A(2x + 2x^2)e^{2x} + 4Ax^2e^{2x} &= e^{2x} \\
  (2A + 8Ax + 4Ax^2 - 8Ax - 8Ax^2 + 4Ax^2)e^{2x} &= e^{2x} \\
  2A &= 1 \\
  A &= \frac{1}{2} \\
\end{align*}

Saadaan DY:n yleinen ratkaisu
\[
  y = C_1xe^{2x} + C_2e^{2x} + \frac{1}{2}x^2e^{2x}, \quad C_1,C_2 \in \mathbb{R}.
\]

Sijoitetaan alkuehdot.

$y(0) = 1$:
\begin{align*}
  C_2e^{2*0} &= 1
  C_2 &= 1
\end{align*}

$\frac{dy}{dx}(0) = 0$:
\begin{align*}
  C_1(2*0 + 1)e^{2*0} + 2C_2e^{2*0} &= 0 \\
  C_1 + 2 &= 0 \\
  C_1 &= -2 \\
\end{align*}

Saadaan alkuarvotehtävän ratkaisuksi
\[
  y = -2xe^{2x} + e^{2x} + \frac{1}{2}x^2e^{2x}, \quad C_1,C_2 \in \mathbb{R}.
\]

\section*{9.}

\subsection*{(a)}

\[
  \frac{d^2 y}{d x^2} + 2\frac{dy}{dx} + 2y = e^x\sin x
\]

Ratkaistaan vastaava HY karakteristisen yhtälön avulla:
\begin{align*}
  r^2 + 2r + 2 &= 0 \\
  r &= \frac{-2 \pm \sqrt{4 - 8}}{2} \\
    &= -1 \pm i \\
\end{align*}
Saadaan ratkaisuksi
\[
  y = C_1e^{-x}\cos x + C_2e^{-x}\sin x, \quad C_1,C_2 \in \mathbb{R}.
\]

Etsitään epähomogeenisen yhtälön yksittäisratkaisua yritteellä
$Ae^x\sin x + Be^x\cos x$.
\begin{align*}
  (2Ae^x\cos x &- 2Be^x\sin x) \\
      &+ 2(Ae^x(\sin x + \cos x) + Be^x(\cos x - \sin x)) \\
      &+ 2(Ae^x\sin x + Be^x \cos x) \\
      &\qquad= e^x\sin x \\
  (2A + 2A + 2B + 2B)e^x\cos x &+ (-2B + 2A - 2B + 2A)e^x\sin x = e^x\sin x \\
\end{align*}
Saadaan yhtälöpari
\[
  \begin{cases}
    4A + 4B = 0 \\
    4A - 4B = 1 \\
  \end{cases}
\]
jonka ratkaisu on
\[
  \begin{cases}
    A = \frac{1}{8} \\
    B = -\frac{1}{8}. \\
  \end{cases}
\]
Saadaan yleinen ratkaisu
\[
  y = C_1e^{-x}\cos x + C_2e^{-x}\sin x + \frac{1}{8}e^x\sin x - \frac{1}{8}e^x\cos x, \quad C_1,C_2 \in \mathbb{R}.
\]

\subsection*{(b)}

\[
  \frac{d^2 y}{d x^2} + 2\frac{dy}{dx} + 2y = e^{-x}\sin x
\]

(a)-kohdasta saadaan vastaavan HY:n ratkaisuksi
\[
  y = C_1e^{-x}\cos x + C_2e^{-x}\sin x, \quad C_1,C_2 \in \mathbb{R}.
\]
Tässä $Ae^{-x}\sin x + Be^{-x}\cos x$ on HY:n ratkaisu, joten se täytyy kertoa x:llä,
jotta se toimii yritteenä. Saadaan
\[
  y = xe^{-x}(A\sin x + B\cos x).
\]
Tämän derivointi menee jo vaikeaksi, joten tehdään se vaiheittain erikseen:
\begin{align*}
  \frac{dy}{dx} &= xe^{-x}(A\cos x - B\sin x) + (e^{-x} - xe^{-x})(A\sin x + B\cos x) \\
                &= (Axe^{-x} + Be^{-x} - Bxe^{-x})\cos x + (-Bxe^{-x} + Ae^{-x} - Axe^{-x})\sin x \\
                &= ((A - B)x + B)e^{-x}\cos x + ((-A - B)x + A)e^{-x}\sin x \\
\end{align*}
\begin{align*}
  \frac{d^2 y}{d x^2} &= (A - B)e^{-x}\cos x + ((A-B)x + B)(-e^{-x}\cos x - e^{-x}\sin x) \\
                      &\qquad+ (-A - B)e^{-x}\sin x + ((-A-B)x + A)(-e^{-x}\sin x + e^{-x}\cos x) \\
                      &= (A - B - (A-B)x - B + (-A-B)x + A)e^{-x}\cos x \\
                      &\qquad + (-(A-B)x - B - A - B - (-A-B)x - A)e^{-x}\sin x \\
                      &= (-2Ax + 2A - 2B)e^{-x}\cos x + (2Bx - 2A - 2B)e^{-x}\sin x \\
\end{align*}

Sijoitetaan nämä yhtälöön:
\begin{align*}
  &(-2Ax + 2A - 2B)e^{-x}\cos x + (2Bx - 2A - 2B)e^{-x}\sin x \\
  &\qquad + 2(((A - B)x + B)e^{-x}\cos x + ((-A - B)x + A)e^{-x}\sin x) \\
  &\qquad + 2(xe^{-x}(A\sin x + B\cos x)) \\
  &\qquad\qquad = e^{-x}\sin x \\
  \iff&(-2Ax + 2A - 2B + 2Ax - 2Bx + 2B + 2Bx)e^{-x}\cos x \\
  &\qquad + (2Bx - 2A - 2B - 2Ax - 2Bx + 2A + 2Ax)e^{-x}\sin x \\
  &\qquad\qquad= e^{-x}\sin x \\
  \iff&2Ae^{-x}\cos x - 2Be^{-x}\sin x = e^{-x}\sin x \\
\end{align*}

Saadaan $2A = 0 \iff A = 0$ ja $-2B = 1 \iff B = -\frac{1}{2}$.
Siis DY:n ratkaisu on
\[
  y = e^{-x}(C_1\cos x + C_2\sin x - \frac{1}{2}x\cos x), \quad C_1, C_2 \in \mathbb{R}.
\]

\section*{10.}

\[
  \frac{d^2 y}{d t^2} + \frac{k}{m}y = 0
\]

Ratkaisemalla DY saadaan punnuksen liikerata.
Karakteristinen yhtälö:
\begin{align*}
  r^2 + \frac{k}{m} &= 0 \\
  r &= \sqrt{-\frac{k}{m}} \\
    &= \pm \frac{\sqrt{k}}{\sqrt{m}}i
\end{align*}

Nyt $k = 90000$, joten $r = \pm \frac{300}{\sqrt{m}}i$.

Saadaan yhtälön ratkaisuksi
\[
  y = A\cos(\frac{300}{\sqrt{m}}t) + B\sin(\frac{300}{\sqrt{m}}t), \quad A,B \in \mathbb{R}.
\]
Taajuus on
\[
  \frac{2\pi}{\frac{300}{\sqrt{m}}} = \frac{2\pi \sqrt{m}}{300}.
\]
Kun taajuus on 10 Hz, niin massaksi saadaan
\[
  m = (10*\frac{300}{2\pi})^2 = \frac{2250000}{\pi^2}.
\]
ja yhtälöksi
\[
  y = A\cos(\frac{\pi}{5}t) + B\sin(\frac{\pi}{5}t), \quad A,B \in \mathbb{R}.
\]

Kun määrätään alun sijainti (-1cm) ja nopeus (2 cm/s) saadaan alkuarvotehtävä
$y(0) = -1$ ja $\frac{dy}{dt}(0) = 2$. Sijoitetaan:
\begin{align*}
  A\cos(0) + B\sin(0) &= -1 \\
  A &= -1 \\
\end{align*}
ja
\begin{align*}
  -\frac{\pi}{5}A\sin(0) + \frac{\pi}{5}B\cos(0) &= 2 \\
  \frac{\pi}{5}B &= 2 \\
  B &= \frac{10}{\pi} \\
\end{align*}
Saadaan yhtälö
\[
  y = -\cos(\frac{\pi}{5}t) + \frac{10}{\pi}\sin(\frac{\pi}{5}t).
\]
Suurin poikkeama lepotilasta on amplitudi, joka on
\[
  R = \sqrt{A^2 + B^2} = \sqrt{1 + \frac{100}{\pi^2}}.
\]

\end{document}

%! TEX program = xelatex

\documentclass{article}
\usepackage[a4paper, margin=3cm]{geometry}
\setlength{\parindent}{0pt}
\setlength{\parskip}{1em}
%\usepackage{fontspec}
%\setmainfont{Lato}

\usepackage{amsmath,amssymb,amsthm}
%\usepackage{verbatim}
%\usepackage{graphicx}
%\usepackage{pgfplots}
%\pgfplotsset{compat=1.16}

\title{MATA114 Harjoitus 6}
\author{Mikael Myyrä}
\date{}

\begin{document}
\maketitle

\section*{1.}

\begin{align*}
  \sum_{n=1}^{\infty} na_nx^{n-1} &= 2\sum_{n=0}^{\infty} a_nx^n \\
                                  &= \sum_{n=1}^{\infty} 2a_{n-1}x^{n-1} \\
  \sum_{n=1}^{\infty} na_nx^{n-1} - \sum_{n=1}^{\infty} 2a_{n-1}x^{n-1} &= 0 \\
\end{align*}
Tämä toteutuu kaikille $x$ vain, jos kaikki summien kertoimet ovat samat, eli
\[
  na_n = 2a_{n-1} \iff a_n = \frac{2a_{n-1}}{n}.
\]
Tästä saadaan kertoimet vapaan vakion $a_0$ suhteen:
\[
  a_1 = \frac{2a_0}{1},
  \quad a_2 = \frac{2a_1}{2} = \frac{2^2}{2 * 1}a_0,
  \quad a_3 = \frac{2a_2}{3} = \frac{2^3}{3 * 2 * 1}a_0,
  \quad \dots
\]
Yleinen kaava tälle lukujonolle on
\[
  a_n = \frac{2^n}{n!}.
\]
$\sum_{n=0}^{\infty}a_nx^n$ on sarjaesitys funktiosta $e^{2x}$.

\section*{2.}

\[
  \frac{d^2 y}{d x^2} - y = 0
\]

\subsection*{(a)}

Sarjan $y = \sum_{n=0}^{\infty} a_nx^n$ toinen derivaatta on
\[
  \frac{d^2 y}{d x^2} = \sum_{n=2}^{\infty} n(n-1)a_nx^{n-2}.
\]
Sijoitetaan nämä yhtälöön:

\begin{align*}
  \sum_{n=2}^{\infty} n(n-1)a_nx^{n-2} - \sum_{n=0}^{\infty} a_nx^n &= 0 \\
  \sum_{n=0}^{\infty}(n+2)(n+1)a_{n+2}x^n - \sum_{n=0}^{\infty} a_nx^n &= 0 \\
\end{align*}
Saadaan rekursioyhtälö
\begin{align*}
  (n+2)(n+1)a_{n+2} - a_n &= 0 \\
  a_{n+2} &= \frac{a_n}{(n+2)(n+1)}. \\
\end{align*}
$a_0$ ja $a_1$ ovat vapaita vakioita. Niitä seuraavat kertoimet ovat
\[
  a_2 = \frac{a_0}{2 * 1}, \quad
  a_3 = \frac{a_1}{3 * 2}, \quad
  a_4 = \frac{a_2}{4 * 3} = \frac{a_0}{4!}, \quad
  a_5 = \frac{a_3}{5 * 4} = \frac{a_1}{5!}
\]
jne.

Yleinen kaava parillisen indeksin kertoimille on
\[
  a_{2k} = \frac{a_0}{(2k)!}, \quad k = 1,2,\dots
\]
ja parittomille indekseille
\[
  a_{2k+1} = \frac{a_1}{(2k+1)!}.
\]
Valitsemalla $a_0 = 1, a_1 = 0$ ja $a_0 = 0, a_1 = 1$ saadaan DY:n
yksittäisratkaisut
\[
  y_1 = 1 + \sum_{k=1}^{\infty}\frac{1}{(2k)!}x^{2k}
\]
ja
\[
  y_2 = x + \sum_{k=1}^{\infty}\frac{1}{(2k+1)!}x^{2k+1}.
\]
Nämä ovat lineaarisesti riippumattomat, joten yhtälön yleinen ratkaisu on
\[
  y = a_0\Big(1 + \sum_{k=1}^{\infty}\frac{1}{(2k)!}x^{2k}\Big)
  + a_1\Big(x + \sum_{k=1}^{\infty}\frac{1}{(2k+1)!}x^{2k+1}\Big), \quad a_0,a_1 \in \mathbb{R}.
\]

\subsection*{(b)}

Ratkaistaan karakteristinen yhtälö:
\begin{align*}
  r^2 - 1 &= 0 \\
  r &= \pm 1 \\
\end{align*}
Tästä saadaan yhtälön ratkaisuksi
\[
  y = C_1e^{x} + C_2e^{-x}, \quad C_1,C_2 \in \mathbb{R}.
\]
Edellisen kohdan sarjaratkaisu on sarjaesitys funktiolle
\begin{align*}
  y &= a_0\cosh x + a_1\sinh x \\
    &= \frac{a_0}{2}(e^x + e^{-x}) + \frac{a_1}{2}(e^x - e^{-x}) \\
    &= \frac{a_0 + a_1}{2}e^x + \frac{a_0 - a_1}{2}e^{-x}. \\
\end{align*}
Mitkä tahansa kaksi reaalilukua $C_1$ ja $C_2$ saadaan tämän
kertoimista valitsemalla sopivat $a_0$ ja $a_1$. Siis ratkaisut ovat samat.

\section*{3.}

\[
  \frac{d^2 y}{d x^2} = xy
\]

Sijoitetaan $y = \sum_{n=0}^{\infty} a_nx^n$ ja
$\frac{d^2 y}{d x^2} = \sum_{n=2}^{\infty} n(n-1)a_nx^{n-2}$:
\begin{align*}
  \sum_{n=2}^{\infty} n(n-1)a_nx^{n-2} &= x\sum_{n=0}^{\infty} a_nx^n \\
                                       &= \sum_{n=0}^{\infty} a_nx^{n+1} \\
                                       &= \sum_{n=1}^{\infty} a_{n-1}x^{n} \\
  \sum_{n=0}^{\infty} (n+2)(n+1)a_{n+2}x^{n} - \sum_{n=1}^{\infty} a_{n-1}x^{n} &= 0\\
  2a_2 + \sum_{n=1}^{\infty} ((n+2)(n+1)a_{n+2} - a_{n-1})x^{n} &= 0\\
\end{align*}
Tästä saadaan
\[
  a_2 = 0
\]
ja
\begin{align*}
  (n+2)(n+1)a_{n+2} - a_{n-1} &= 0 \\
  (n+3)(n+2)a_{n+3} - a_n &= 0 \\
  a_{n+3} &= \frac{1}{(n+3)(n+2)}a_n. \\
\end{align*}

$a_0$ ja $a_1$ ovat vapaita vakioita. Näitä seuraavat kertoimet ovat
\[
  a_3 = \frac{1}{3*2}a_0, \quad
  a_4 = \frac{1}{4*3}a_1, \quad
  a_5 = \frac{1}{5*4}a_2 = 0, \quad
\]
\[
  a_6 = \frac{1}{6*5}a_3 = \frac{1}{6*5*3*2}a_0, \quad
  a_7 = \frac{1}{7*6}a_4 = \frac{1}{7*6*4*3}a_1, \quad
  a_8 = 0,
\]
jne. Merkitään joka kolmatta kerrointa $a_{3k}, \quad k = 1,2,\dots$.

$k = 1$:
\[
  a_3 = \frac{1}{3k * (3k-1)}a_0
\]
$k = 2$:
\[
  a_6 = \frac{1}{3k * (3k-1) * 3 * 2}a_0 = \frac{1}{3k * (3k-1) * 3(k-1) * (3(k-1)-1)}a_0
\]
$k = 3$:
\[
  a_9 = \frac{1}{3k * (3k-1) * 6 * 5 * 3 * 2}a_0
\]
Kolmella jaolliset termit nimittäjän tulossa saadaan yhdistettyä kertomaksi
$3^kk!$. Muita termejä ei saada vastaavasti yhdistettyä, joten ne joudutaan
esittämään lukujonona. Siis
\[
  a_{3k} = \frac{1}{3^kk! * 2 * 5 * \dots * (3k-1)}a_0.
\]
Vastaavasti joka kolmatta seuraaville kertoimille saadaan
\[
  a_{3k+1} = \frac{1}{3^kk! * 4 * 7 * \dots * (3k+1)}a_1.
\]
Valitsemalla $a_0 = 1, a_1 = 0$ ja $a_0 = 0, a_1 = 1$ ja sijoittamalla nämä
yritteeseen $\sum_{n=0}^{\infty} a_nx^n$ saadaan DY:lle ratkaisut
\[
  y_1 = 1 + \sum_{k=1}^{\infty} \frac{1}{3^kk! * 2 * 5 * \dots * (3k-1)}x^{3k}
\]
ja
\[
  y_2 = x + \sum_{k=1}^{\infty} \frac{1}{3^kk! * 4 * 7 * \dots * (3k+1)}x^{3k+1}
\]
joiden lineaarinen verho on DY:n yleinen ratkaisu.

\section*{4.}

\[
  \begin{cases}
    \frac{d^2 y}{d x^2} + x\frac{dy}{dx} + 2y = 0 \\
    y(0) = 1 \\
    \frac{dy}{dx}(0) = 2 \\
  \end{cases}
\]

Sijoitetaan yhtälöön sarja $y = \sum_{n=0}^{\infty} a_nx^n$:
\begin{align*}
  \sum_{n=2}^{\infty} n(n-1)a_nx^{n-2} + x\sum_{n=1}^{\infty} na_nx^{n-1} + 2\sum_{n=0}^{\infty} a_nx^n &= 0 \\
  \sum_{n=0}^{\infty} (n+2)(n+1)a_{n+2}x^{n} + \sum_{n=1}^{\infty} na_nx^{n} + \sum_{n=0}^{\infty} 2a_nx^n &= 0 \\
  2a_2 + 2a_0 + \sum_{n=1}^{\infty} ((n+2)(n+1)a_{n+2} + (n + 2)a_n)x^n &= 0 \\
\end{align*}
Saadaan
\[
  2a_2 + 2a_0 = 0 \iff a_2 = -a_0
\]
ja
\begin{align*}
  (n+2)(n+1)a_{n+2} + (n+2)a_n &= 0 \\
  a_{n+2} &= -\frac{n+2}{(n+2)(n+1)}a_n \\
          &= -\frac{1}{n+1}a_n \\
\end{align*}

$a_0$ ja $a_1$ ovat vapaita vakioita. Tästä yhtälöstä saatava $a_2 = -a_0$
toteuttaa aikaisemmin saadun ehdon $a_2$:lle. Näitä seuraavat parilliset kertoimet ovat
\[
  a_4 = -\frac{1}{3}a_2 = \frac{1}{3}a_0, \quad
  a_6 = -\frac{1}{5}a_4 = -\frac{1}{5*3}a_0, \quad
  a_8 = -\frac{1}{7}a_6 = \frac{1}{7*5*3}a_0,\,\dots
\]
ja parittomat
\[
  a_3 = -\frac{1}{2}a_1,
  a_5 = -\frac{1}{4}a_3 = \frac{1}{4*2}a_1,
  a_7 = -\frac{1}{6}a_5 = -\frac{1}{6*4*2}a_1,\,\dots
\]
Yleiset kaavat näille ovat
\[
  a_{2k} = (-1)^k \frac{2^kk!}{(2k)!}a_0, \quad k = 1,2,\dots
\]
parillisille indekseille ja
\[
  a_{2k+1} = (-1)^k \frac{1}{2^kk!}a_1
\]
parittomille. Näistä saadaan yhtälölle ratkaisu
\[
  y = a_0\Big(1 + \sum_{k=1}^{\infty} (-1)^k \frac{2^kk!}{(2k)!}x^{2k}\Big)
  + a_1\Big(x + \sum_{k=1}^{\infty} (-1)^k \frac{1}{2^kk!}x^{2k+1} \Big).
\]
Pisteessä $x = 0$ y:n arvo on $a_0$ ja derivaatta $a_1$, joten alkuehdot
toteuttaa yhtälö
\[
  y = \Big(1 + \sum_{k=1}^{\infty} (-1)^k \frac{2^kk!}{(2k)!}x^{2k}\Big)
  + 2\Big(x + \sum_{k=1}^{\infty} (-1)^k \frac{1}{2^kk!}x^{2k+1} \Big).
\]

\section*{5.}

\subsection*{(a)}

\[
  \frac{d^2 y}{d x^2} = (x-1)^2y
\]

Sijoitetaan $y = \sum_{n=0}^{\infty} a_n(x-1)^n$ ja $\frac{d^2 y}{d x^2} =
\sum_{n=2}^{\infty} n(n-1)a_n(x-1)^{n-2}$:
\begin{align*}
  \sum_{n=2}^{\infty} n(n-1)a_n(x-1)^{n-2} &= (x-1)^2 \sum_{n=0}^{\infty} a_n(x-1)^n \\
                                           &= \sum_{n=0}^{\infty} a_n(x-1)^{n+2} \\
  \sum_{n=0}^{\infty} (n+2)(n+1)a_{n+2}(x-1)^{n} - \sum_{n=2}^{\infty} a_{n-2}(x-1)^n &= 0 \\
  2a_2 + 6a_3(x-1) + \sum_{n=2}^{\infty} ((n+2)(n+1)a_{n+2} - a_{n-2})(x-1)^n &= 0 \\
\end{align*}
Saadaan $2a_2 + 6a_3(x-1) = 0$, mikä on totta kaikilla $x$ vain, kun $a_2 = a_3 = 0$,
ja
\begin{align*}
  (n+2)(n+1)a_{n+2} - a_{n-2} &= 0 \\
  (n+4)(n+3)a_{n+4} - a_n &= 0 \\
  a_{n+4} &= \frac{1}{(n+4)(n+3)}a_n. \\
\end{align*}
Tässä on samanlainen rakenne kuin tehtävässä 3, vain nollasta eroavien
kerrointen välimatka on eri. Vastaavilla päättelyillä saadaan ratkaisu
\begin{align*}
  y &= a_0\Big(1 + \sum_{k=1}^{\infty}\frac{1}{4^kk!*3*7*\dots*(4k-1)} (x-1)^{4k}\Big) \\
    &\quad+ a_1\Big(x-1 + \sum_{k=1}^{\infty}\frac{1}{4^kk!*5*9*\dots*(4k+1)} (x-1)^{4k+1}\Big).
\end{align*}

\subsection*{(b)}

\[
  \begin{cases}
    \frac{d^2 y}{d x^2} = (x-1)^2y \\
    y(1) = 2 \\
    \frac{dy}{dx}(1) = -3 \\
  \end{cases}
\]

Yhtälön ratkaisu saatiin (a)-kohdasta.
Pisteessä $x = 1$ kaikki summien termit ovat nollia, joten
$y(1) = a_0$ ja $\frac{dy}{dx}(1) = a_1$.
Siis alkuehdot toteuttava ratkaisu on
\begin{align*}
  y &= 2\Big(1 + \sum_{k=1}^{\infty}\frac{1}{4^kk!*3*7*\dots*(4k-1)} (x-1)^{4k}\Big) \\
    &\quad- 3\Big(x-1 + \sum_{k=1}^{\infty}\frac{1}{4^kk!*5*9*\dots*(4k+1)} (x-1)^{4k+1}\Big).
\end{align*}

\end{document}

%! TEX program = xelatex

\documentclass{article}
\usepackage[a4paper, margin=3cm]{geometry}
\setlength{\parindent}{0pt}
\setlength{\parskip}{1em}
%\usepackage{fontspec}
%\setmainfont{Lato}

\usepackage{amsmath,amssymb,amsthm}
%\usepackage{verbatim}
%\usepackage{graphicx}
%\usepackage{pgfplots}
%\pgfplotsset{compat=1.16}

\title{MATA114 Harjoitus 6}
\author{Mikael Myyrä}
\date{}

\begin{document}
\maketitle

\section*{1.}

\begin{align*}
  \sum_{n=1}^{\infty} na_nx^{n-1} &= 2\sum_{n=0}^{\infty} a_nx^n \\
                                  &= \sum_{n=1}^{\infty} 2a_{n-1}x^{n-1} \\
  \sum_{n=1}^{\infty} na_nx^{n-1} - \sum_{n=1}^{\infty} 2a_{n-1}x^{n-1} &= 0 \\
\end{align*}
Tämä toteutuu kaikille $x$ vain, jos kaikki summien kertoimet ovat samat, eli
\[
  na_n = 2a_{n-1} \iff a_n = \frac{2a_{n-1}}{n}.
\]
Tästä saadaan kertoimet vapaan vakion $a_0$ suhteen:
\[
  a_1 = \frac{2a_0}{1},
  \quad a_2 = \frac{2a_1}{2} = \frac{2^2}{2 * 1}a_0,
  \quad a_3 = \frac{2a_2}{3} = \frac{2^3}{3 * 2 * 1}a_0,
  \quad \dots
\]
Yleinen kaava tälle lukujonolle on
\[
  a_n = \frac{2^n}{n!}.
\]
$\sum_{n=0}^{\infty}a_nx^n$ on sarjaesitys funktiosta $e^{2x}$.

\section*{2.}

\[
  \frac{d^2 y}{d x^2} - y = 0
\]

\subsection*{(a)}

Sarjan $y = \sum_{n=0}^{\infty} a_nx^n$ toinen derivaatta on
\[
  \frac{d^2 y}{d x^2} = \sum_{n=2}^{\infty} n(n-1)a_nx^{n-2}.
\]
Sijoitetaan nämä yhtälöön:

\begin{align*}
  \sum_{n=2}^{\infty} n(n-1)a_nx^{n-2} - \sum_{n=0}^{\infty} a_nx^n &= 0 \\
  \sum_{n=0}^{\infty}(n+2)(n+1)a_{n+2}x^n - \sum_{n=0}^{\infty} a_nx^n &= 0 \\
\end{align*}
Saadaan rekursioyhtälö
\begin{align*}
  (n+2)(n+1)a_{n+2} - a_n &= 0 \\
  a_{n+2} &= \frac{a_n}{(n+2)(n+1)}. \\
\end{align*}
$a_0$ ja $a_1$ ovat vapaita vakioita. Niitä seuraavat kertoimet ovat
\[
  a_2 = \frac{a_0}{2 * 1}, \quad
  a_3 = \frac{a_1}{3 * 2}, \quad
  a_4 = \frac{a_2}{4 * 3} = \frac{a_0}{4!}, \quad
  a_5 = \frac{a_3}{5 * 4} = \frac{a_1}{5!}
\]
jne.

Yleinen kaava parillisen indeksin kertoimille on
\[
  a_{2k} = \frac{a_0}{(2k)!}, \quad k = 0,1,\dots
\]
ja parittomille indekseille
\[
  a_{2k+1} = \frac{a_1}{(2k+1)!}.
\]
Saadaan DY:n ratkaisu
\[
  y = a_0\Big(\sum_{k=0}^{\infty}\frac{1}{(2k)!}x^{2k}\Big)
  + a_1\Big(\sum_{k_0}^{\infty}\frac{1}{(2k+1)!}x^{2k+1}\Big), \quad a_0,a_1 \in \mathbb{R}.
\]

\subsection*{(b)}

Ratkaistaan karakteristinen yhtälö:
\begin{align*}
  r^2 - 1 &= 0 \\
  r &= \pm 1 \\
\end{align*}
Tästä saadaan yhtälön ratkaisuksi
\[
  y = C_1e^{x} + C_2e^{-x}, \quad C_1,C_2 \in \mathbb{R}.
\]
Edellisen kohdan sarjaratkaisu on sarjaesitys funktiolle
\begin{align*}
  y &= a_0\cosh x + a_1\sinh x \\
    &= \frac{a_0}{2}(e^x + e^{-x}) + \frac{a_1}{2}(e^x - e^{-x}) \\
    &= \frac{a_0 + a_1}{2}e^x + \frac{a_0 - a_1}{2}e^{-x}, \\
\end{align*}
joka on sama funktio kuin tämä, vain eri tavalla muotoilluilla vapailla
vakioilla. Siis ratkaisut ovat samat.

\end{document}

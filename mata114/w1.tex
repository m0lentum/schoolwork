%! TEX program = xelatex

\documentclass{article}
\usepackage[a4paper, margin=3cm]{geometry}
\setlength{\parindent}{0pt}
%\usepackage{fontspec}
%\setmainfont{Lato}

\usepackage{amsmath,amssymb,amsthm}

\title{MATA114 Harjoitus 1}
\author{Mikael Myyrä}
\date{}

\begin{document}
\maketitle

\section*{1.}

\subsection*{A}
\[
  y' - 2xy = \sin x
\]
Tämä on tavallinen DY, jos $y$ on $x$:n funktio.
\[
  y'(x) - 2xy(x) = \sin x
\]

\subsection*{B}
\[
  \frac{df}{dx} = f^2 + x^2
\]
Tämä on tavallinen DY.
\[
  \frac{df}{dx}(x) = (f(x))^2
\]

\subsection*{C}
\[
  \frac{d^2 x}{dt^2} - 3 \frac{dx}{dt} + 6x = 0
\]
Tämä on tavallinen DY (funktiona x(t)).
\[
  \frac{d^2 x}{dt^2}(t) - 3 \frac{dx}{dt}(t) + 6x(t) = 0
\]

\subsection*{D}
\[
  \frac{du}{dt} = \frac{d^2 u}{dx^2}
\]
Tämä on osittaisdifferentiaaliyhtälö (funktiona u(t, x)), ei tavallinen DY.
\[
  \frac{du}{dt}(t, x) = \frac{d^2 u}{dx^2}(t, x)
\]

\subsection*{E}
\[
  (v'')^3 - v \sin x = 0
\]
Tämä on tavallinen DY, jos $v$ on $x$:n funktio.
\[
  (v''(x))^3 - v(x) \sin x = 0
\]

\subsection*{F}
\[
  y - \frac{x^2}{x^3 - 1} = 0
\]
Tämä on pelkkä yhtälö, ei DY, koska mukana ei ole yhtään derivaattaa.

\section*{2.}

\begin{equation}\label{ex2eq}
  xy' = 2y
\end{equation}

\subsection*{(a)}

Sijoitetaan $y(x) = x^2$ ja $y'(x) = 2x$ yhtälöön \ref{ex2eq}.
\begin{align*}
  x*2x &= 2x^2 \\
  2x^2 &= 2x^2
\end{align*}
Tämä on totta.

\subsection*{(b)}

Kokeillaan taas sijoittamalla $y(x) = -3x^2$ ja $y'(x) = -6x$.
\begin{align*}
  x(-6x) &= 2(-3x^2) \\
  -6x^2 &= -6x^2
\end{align*}
Tämä on totta, joten $y(x) = -3x^2$ on yhtälön \ref{ex2eq} ratkaisu.

\subsection*{(c)}

Sijoitetaan $y(x) = Cx^2$ ja $y'(x) = 2Cx$.
\begin{align*}
  x * 2Cx &= 2(Cx^2) \\
  2Cx^2 &= 2Cx^2 \qquad \qedsymbol
\end{align*}

\subsection*{(d)}

Sijoitetaan $y(x) = Cx^2$ yhtälöön $y(2) = -10$.
\begin{align*}
  y(2) &= -10 \\
  C * 2^2 &= -10 \\
  C &= -\frac{10}{4} = -\frac{5}{2}
\end{align*}
Siis alkuehdon toteuttaa $y(x) = -\frac{5}{2}x^2$.

\section*{3.}

\subsection*{(a)}

\begin{align*}
  y' &= \sin 2x + 3 \\
  y &= \int \sin (2x + 3) \,dx \\
    &= -\frac{1}{2} \cos(2x + 3) + C, \quad C \in \mathbb{R} \\
\end{align*}

\subsection*{(b)}

\begin{align*}
  y'' &= e^{3x+2} \\
  y &= \iint e^{3x+2} \,dx\,dx \\
    &= \frac{1}{3} \int (e^{3x+2} + C) \,dx \quad C \in \mathbb{R} \\
    &= \frac{1}{9} e^{3x+2} + Cx + D \quad C, D \in \mathbb{R} \\
\end{align*}

\subsection*{(c)}

\begin{align*}
  y''' &= ax + b \\
  y &= \iiint (ax + b) \,dx\,dx\,dx \\
    &= \iint (\frac{a}{2} x^2 + bx + C) \,dx\,dx, \quad C \in \mathbb{R} \\
    &= \int (\frac{a}{6} x^3 + \frac{b}{2} x^2 + Cx + D) \,dx, \quad C,D \in \mathbb{R} \\
    &= \frac{a}{24} x^4 + \frac{b}{6} x^3 + \frac{C}{2} x^2 + Dx + E, \quad C,D,E \in \mathbb{R} \\
\end{align*}

\section*{4.}

\subsection*{(a)}
\[
  \begin{cases}
    y' = x - 2 \\
    y(0) = 3
  \end{cases}
\]
Ratkaistaan ensin DY.
\begin{align*}
  y' &= x - 2 \\
  y &= \int (x - 2)\,dx \\
    &= \frac{1}{2} x^2 - 2x + C, \quad C \in \mathbb{R} \\
\end{align*}
Sijoitetaan $y(0) = 3$.
\begin{align*}
  \frac{1}{2} 0^2 - 2 * 0 + C &= 3 \\
  C &= 3 \\
\end{align*}
Vastaus: $y = \frac{1}{2} x^2 - 2x + 3$.

\subsection*{(b)}
\[
\begin{cases}
  y' = 3\sqrt{x} \\
  y(4) = 1
\end{cases}
\]
Ratkaistaan DY.
\begin{align*}
  y' &= 3\sqrt{x} \\
  y &= \int 3\sqrt{x} \,dx \\
    &= 2x^{\frac{3}{2}} + C, \quad C \in \mathbb{R} \\
\end{align*}
Sijoitetaan $y(4) = 1$.
\begin{align*}
  2*4^{\frac{3}{2}} + C &= 1 \\
  16 + C &= 1 \\
  C &= -15 \\
\end{align*}
Vastaus: $y = 2x^{\frac{3}{2}} - 15$.

\subsection*{(c)}
\[
  \begin{cases}
    y' = \sin 2x \\
    y(\frac{\pi}{6}) = 1 \\
  \end{cases}
\]
Ratkaistaan DY.
\begin{align*}
  y' &= \sin 2x \\
  y &= \int \sin 2x \,dx \\
    &= -\frac{1}{2} \cos 2x + C, \quad C \in \mathbb{R} \\
\end{align*}
Sijoitetaan $y(\frac{\pi}{6}) = 1$.
\begin{align*}
  -\frac{1}{2} \cos(2 * \frac{\pi}{6}) + C &= 1 \\
  C &= -2 - \cos \frac{\pi}{3} \\
    &= -2 - \frac{1}{2} \\
    &= -\frac{5}{2} \\
\end{align*}
Vastaus: $y = -\frac{1}{2} \cos 2x - \frac{5}{2}$.

\section*{5.}

\subsection*{A}
\[
  y' - 2xy = \sin x
\]
Ei ole separoituva. On lineaarinen, ei homogeeninen. \\
Luontevimmalta minulle tuntuu käyttää derivaattoihin Leibnizin notaatiota
derivaattojen kanssa ja mahdollisesti merkitä muuttujat sulkeisiin funktion
yhteyteen, jos haluaa olla mahdollisimman yksikäsitteinen.
\[
  \frac{dy}{dx} - 2xy(x) = \sin x
\]

\subsection*{B}
\[
  \frac{df}{dx} = f^2 + x^2
\]
Ei ole separoituva. Ei myöskään lineaarinen.
\[
  \frac{df}{dx} = (f(x))^2 + x^2
\]

\subsection*{C}
\[
  y' + xy^2 = 0
\]
On separoituva, ei lineaarinen. Normaalimuoto:
\[
  \frac{dy}{dx} = -x(y(x))^2
\]
Separoitu muoto:
\[
  \frac{1}{(y(x))^2} \frac{dy}{dx} = -x
\]

\subsection*{D}
\[
  xu' = u + 1
\]

\begin{align*}
  \frac{du}{dx} &= \frac{u(x) + 1}{x} \\
                &= \frac{u(x)}{x} + \frac{1}{x}
\end{align*}
Ei ole separoituva. On lineaarinen, ei homogeeninen.

\subsection*{E}
\[
  v' = 2tv + t^2
\]
Ei ole separoituva. On lineaarinen, ei homogeeninen.
\[
  \frac{dv}{dt} = 2tv(t) + t^2
\]

\subsection*{F}
\[
  y' - \frac{x^2}{x^3 - 1} = 0
\]
On separoituva ($g(y) = 1$) ja lineaarinen, mutta ei homogeeninen.
Normaalimuoto:
\[
  \frac{dy}{dx} = \frac{x^2}{x^3 - 1}
\]
Tämä on myös separoitu muoto.

\section*{6.}

\subsection*{(a)}

\[
  \frac{dy}{dx} = 4xy
\]

$y = 0$ toteuttaa yhtälön. Jos $y \neq 0$:

\begin{align*}
  \frac{dy}{dx} &= 4xy \\
  \frac{1}{y} \frac{dy}{dx} &= 4x \\
  \int \frac{1}{y} \frac{dy}{dx} \, dx &= \int 4x \, dx \\
  \int \frac{1}{y} \, dy &= \int 4x \, dx \\
  \ln |y| &= 2x^2 + C, \quad C \in \mathbb{R} \\
  |y| &= e^{2x^2 + C} \\
  y &= \pm e^C e^{2x^2} \\
  y &= \tilde{C} e^{2x^2}, \quad \tilde{C} \in \mathbb{R} \setminus \{0\} \\
\end{align*}

Koska myös $y = 0$ toteuttaa yhtälön, niin yleinen ratkaisu on
\[
  y = Ce^{2x^2}, \quad C \in \mathbb{R}.
\]

\subsection*{(b)}
\[
  \frac{dy}{dx} = \frac{y}{2x}
\]
$y = 0$ toteuttaa yhtälön. Jos $y \neq 0$:

\begin{align*}
  \frac{dy}{dx} &= \frac{y}{2x} \\
  \frac{1}{y} \frac{dy}{dx} &= \frac{1}{2x} \\
  \int \frac{1}{y} \, dy &= \int \frac{1}{2x} \, dx \\
  \ln |y| &= \frac{1}{2} \ln |2x| + C_1, \quad C_1 \in \mathbb{R} \\
  |y| &= e^{\frac{1}{2} \ln |2x| + C_1} \\
      &= (e^{\ln |2x| + 2C_1})^{\frac{1}{2}} \\
      &= \sqrt{e^{2C_1} |2x|} \\
      &= \sqrt{2e^{C_2} |x|}, \quad C_2 \in \mathbb{R} \\
      &= C_3 \sqrt{|x|}, \quad C_3 > 0 \\
  y &= \pm C_3 \sqrt{|x|} \\
    &= C_4 \sqrt{|x|}, \quad C_4 \in \mathbb{R} \setminus \{0\} \\
\end{align*}

Koska myös $y = 0$ toteuttaa yhtälön, niin yleinen ratkaisu on
\[
  y = C \sqrt{|x|}, \quad C \in \mathbb{R}.
\]

\subsection*{(c)}
\[
  \frac{dy}{dx} = 1 + y^2
\]
Tällä kertaa $y = 0$ ei ole ratkaisu.

\begin{align*}
  \frac{dy}{dx} &= 1 + y^2 \\
  \frac{1}{1 + y^2} \frac{dy}{dx} &= 1 \\
  \int \frac{1}{1 + y^2} \, dy &= \int 1 \, dx \\
  \arctan y &= x + C, \quad C \in \mathbb{R} \\
  y &= \tan(x + C)
\end{align*}

Tämä ei enää sievene, eikä nolla ole erikoistapaus, joten tämä on ratkaisu.

\section*{7.}

\subsection*{(a)}
\[
  \begin{cases}
    \frac{dy}{dx} + 10y = 1 \\
    y(\frac{1}{10}) = \frac{2}{10} \\
  \end{cases}
\]
Ratkaistaan DY.
\begin{align*}
  \frac{dy}{dx} + 10y &= 1 \\
  \frac{dy}{dx} &= -10y + 1 \\
  \frac{1}{-10y + 1} \frac{dy}{dx} &= 1 \\
  \int \frac{1}{-10y + 1} \,dy &= \int 1 \,dx \\
  - \frac{1}{10} \ln(|-10y + 1|) &= x + C_1, \quad C_1 \in \mathbb{R} \\
  \ln(|-10y + 1|) &= -10x + C_2, \quad C_2 \in \mathbb{R} \\
  |-10y + 1| &= e^{-10x + C_2} \\
             &= e^C_2 e^{-10x} \\
             &= C_3 e^{-10x}, \quad C_3 > 0 \\
  -10y + 1 &= \pm C_3 e^{-10x} \\
           &= C_4 e^{-10x}, \quad C_4 \in \mathbb{R} \setminus \{0\} \\
  y &= -\frac{C_4 e^{-10x} - 1}{10} \\
    &= C_5 e^{-10x} + \frac{1}{10}, \quad C_5 \in \mathbb{R} \setminus\{0\} \\
\end{align*}
Tämä on totta, jos $-10y + 1 \neq 0$.
Jos $-10y + 1 = 0 \iff y = \frac{1}{10}$, niin
$\frac{dy}{dx} = 0$ ja yhtälö ei toteudu.
Siis yleinen ratkaisu on
\[
  y = C e^{-10x} + \frac{1}{10}, \quad C \in \mathbb{R} \setminus\{0\}.
\]
Sijoitetaan $y(\frac{1}{10}) = \frac{2}{10}$.

\begin{align*}
  C e^{-10 * \frac{1}{10}} + \frac{1}{10} &= \frac{2}{10} \\
  \frac{C}{e} &= \frac{1}{10} \\
  C &= \frac{e}{10} \\
\end{align*}

Vastaus:
\begin{align*}
  y &= \frac{e}{10} e^{-10x} + \frac{1}{10} \\
    &= \frac{1}{10} (e^{-10x + 1} + 1) \\
\end{align*}

\subsection*{(b)}
\[
  \begin{cases}
    \frac{dy}{dx} = (x + 4y)^2 \\
    y(0) = 0 \\
  \end{cases}
\]
Kokeillaan muuttujanvaihtoa $u = x + 4y$.
Tällöin $y = \frac{1}{4}(u - x)$ ja $\frac{dy}{dx} = \frac{1}{4}(\frac{du}{dx} - 1)$.

\begin{align*}
  \frac{dy}{dx} &= (x + 4y)^2 \\
  \frac{1}{4}(\frac{du}{dx} - 1) &= u^2 \\
  \frac{du}{dx} &= 4u^2 + 1 \\
  \frac{1}{4u^2 + 1} \frac{du}{dx} &= 1 \\
  \int \frac{1}{4u^2 + 1} \,du &= \int 1 \,dx \\
  \frac{1}{2} \arctan 2u &= x + C_1, \quad C_1 \in \mathbb{R} \\
  \arctan 2u &= 2x + C_2, \quad C_2 \in \mathbb{R} \\
  2u &= \tan(2x + C_2) \\
  u &= \frac{1}{2} \tan(2x + C_2) \\
\end{align*}
Tämä on totta, jos $4u^2 + 1 \neq 0$.
Jos $4u^2 + 1 = 0 \iff u^2 = -\frac{1}{4} \iff u = \pm \frac{1}{2}$,
niin $\frac{du}{dx} = 0$ ja yhtälö ei toteudu.
Siis yleinen ratkaisu on
\[
  u = \frac{1}{2} \tan(2x + C), \quad C \in \mathbb{R}.
\]
Sijoittamalla $u = x + 4y$ saadaan
\begin{align*}
  x + 4y &= \frac{1}{2} \tan(2x + C) \\
  y &= \frac{1}{8} \tan(2x + C) - \frac{1}{4}x. \\
\end{align*}

Sijoitetaan $y(0) = 0$.
\begin{align*}
  \frac{1}{8} \tan(2*0 + C) - \frac{1}{4}*0 &= 0 \\
  \tan(C) &= 0 \\
  C &= 0 \\
\end{align*}

Vastaus:
\[
  y = \frac{1}{8} \tan(2x) - \frac{1}{4}x
\]

\section*{8.}

\begin{align*}
  xy &= C, \quad C \in \mathbb{R} \\
  \iff \quad y &= \frac{C}{x} \\
\end{align*}
Käyrien kulmakertoimet derivoimalla:
\[
  \frac{dy}{dx} = -\frac{C}{x^2}
\]
Sijoitetaan $C = xy$:
\[
  \frac{dy}{dx} = -\frac{y}{x}
\]
Kohtisuorien leikkaajien DY on tämän käänteisluvun vastaluku,
\[
  \frac{dy}{dx} = \frac{x}{y}.
\]
Ratkaistaan tämä:
\begin{align*}
  \frac{dy}{dx} &= \frac{x}{y} \\
  y\frac{dy}{dx} &= x \\
  \int y \,dy &= \int x \,dx \\
  \frac{1}{2} y^2 &= \frac{1}{2} x^2 + C_1, \quad C_1 \in \mathbb{R} \\
  y^2 &= x^2 + C_2, \quad C_2 \in \mathbb{R} \\
  y^2 - x^2 &= C, \quad C \in \mathbb{R} \\
  \frac{y^2}{C} - \frac{x^2}{C} &= 1 \\
\end{align*}

Tämä on hyperbeliparvi.

\section*{9.}

\subsection*{(a)}

Tämä on alkuarvotehtävä
\[
  \begin{cases}
    \frac{dy}{dt} = -0,015y \\
    y(2020) = 8500. \\
  \end{cases}
\]
Ratkaistaan DY.
\begin{align*}
  \frac{dy}{dt} &= -0,015y \\
  \frac{1}{y} \frac{dy}{dt} &= -0,015 \\
  \int \frac{1}{y} \,dy &= -\int 0,015 \,dx \\
  \ln |y| &= -0,015x + C_1, \quad C_1 \in \mathbb{R} \\
  |y| &= e^{-0,015x + C_1} \\
      &= e^{C_1} e^{-0,015x} \\
      &= C_2 e^{-0,015x}, \quad C_2 > 0 \\
  y &= \pm C_2 e^{-0,015x} \\
    &= C_3 e^{-0,015x}, \quad C_3 \in \mathbb{R} \setminus \{0\} \\
\end{align*}

Myös $y = 0$ toteuttaa yhtälön, joten yleinen ratkaisu on
\[
  y = Ce^{-0,015x}, \quad C \in \mathbb{R}.
\]

Sijoitetaan $y(2020) = 8500$.
\begin{align*}
  Ce^{-0,015 * 2020} &= 8500 \\
  C &= 8500e^{30,3} \\
\end{align*}

Alkuarvotehtävän ratkaisu on siis
\[
  y = 8500e^{30,3} e^{-0,015x}
    = 8500e^{30,3 - 0,015x}.
\]

Tästä saadaan vuoden 2030 väkilukuarvioksi
\[
  y(2030) = 8500e^{30,3 - 0,015 * 2030} \approx 7316.
\]

\subsection*{(b)}

Tämän mallin mukaan väkiluku vuonna 2030 olisi
\[
  8500 * 0,985^{2030 - 2020} \approx 7308.
\]
Tämä on lähes mutta ei täsmälleen sama tulos kuin DY-mallilla saatu.
Samankaltaisuus selittyy sillä, että $e^{-0,15} \approx 0,985^{10}$.
Samoin kaikilla $x$ on $e^{30,3 - 0,015x} \approx 0,985^{x - 2020}$.
Tämän voi todeta logaritmin avulla:
\[
  \log_{0,985} e^{30,3 - 0,015x}
    = \frac{\ln e^{30,3 - 0,015x}}{\ln 0,985}
    \approx \frac{30,3 - 0,015x}{-0,015}
    = x - 2020
\]
Tulosten pieni ero selittynee sillä, että nämä eivät kuitenkaan ole
täsmälleen yhtä suuret.

\section*{10.}

\subsection*{(a)}

$\frac{dx}{dt} = 0$, kun $a - bx = 0 \iff x = \frac{a}{b}$.
Siis $x$:n arvo ei muutu enää arvon $\frac{a}{b}$ saavuttamisen
jälkeen. Lisäksi $\frac{dx}{dt} > 0$, kun $x < \frac{a}{b}$
ja $\frac{dx}{dt} < 0$, kun $x > \frac{a}{b}$, joten x saavuttaa aina
ko. arvon. Siis kaikilla alkuarvoilla ja sallituilla $a$:n ja $b$:n arvoilla
pätee
\[
  \lim_{t \to \infty} x(t) = \frac{a}{b}.
\]

\subsection*{(b)}

Ratkaistaan DY.
\begin{align*}
  \frac{dx}{dt} &= a - bx \\
  \frac{1}{a - bx} \frac{dx}{dt} &= 1 \\
  \int \frac{1}{a - bx} \,dx &= \int 1 \,dt \\
  -\frac{1}{b} \ln|a - bx| &= t + C_1, \quad C_1 \in \mathbb{R} \\
  \ln|a - bx| &= -bt - C_1t \\
  |a - bx| &= e^{-bt} e^{C_1t} \\
           &= C_2 e^{-bt}, \quad C_2 > 0 \\
  a - bx &= \pm C_2 e^{-bt} \\
         &= C_3 e^{-bt}, \quad C_3 \in \mathbb{R} \setminus \{0\} \\
  x &= -\frac{1}{b}(C_3 e^{-bt} - a) \\
    &= C_4e^{-bt} + \frac{a}{b}, \quad C_4 \in \mathbb{R} \setminus \{0\} \\
\end{align*}
Tämä pätee, jos $a - bx \neq 0$.
Jos $a - bx = 0 \iff x = \frac{a}{b}$, niin yhtälö toteutuu.
Yleinen ratkaisu on siis
\[
  x = Ce^{-bt} + \frac{a}{b}, \quad C \in \mathbb{R}.
\]
Sijoitetaan $x(0) = 0$.
\begin{align*}
  Ce^{-b*0} + \frac{a}{b} &= 0 \\
  C &= -\frac{a}{b} \\
\end{align*}
Saadaan funktio
\[
  x(t) = -\frac{a}{b}e^{-bt} + \frac{a}{b}.
\]

\subsection*{(c)}

Sijoitetaan b-kohdan ratkaisuun $x = \frac{a}{2b}$
ja ratkaistaan $t$.

\begin{align*}
  -\frac{a}{b}e^{-bt} + \frac{a}{b} &= \frac{a}{2b} \\
  -\frac{a}{b}(e^{-bt} + 1) &= \frac{a}{2b} \\
  e^{-bt} + 1 &= -\frac{1}{b} \\
  e^{-bt} &= -\frac{1}{b} - 1 \\
  -bt &= \ln(-\frac{1}{b} - 1) \\
  t &= -\frac{1}{b}\ln(-\frac{1}{b} - 1) \\
\end{align*}

\end{document}

%! TEX program = xelatex

\documentclass{article}
\usepackage[a4paper, margin=3cm]{geometry}
\setlength{\parindent}{0pt}
\setlength{\parskip}{1em}
%\usepackage{fontspec}
%\setmainfont{Lato}

\usepackage{amsmath,amssymb,amsthm}
%\usepackage{graphicx}
%\usepackage{pgfplots}
%\pgfplotsset{compat=1.16}

\title{MATA114 Harjoitus 5}
\author{Mikael Myyrä}
\date{}

\begin{document}
\maketitle

\section*{1.}

\subsection*{(b)}

\[
  x^2\frac{d^2 y}{d x^2} - 3x\frac{dy}{dx} + 4y = 0, \quad x > 0
\]

Sijoitetaan $y = x^2$, $\frac{dy}{dx} = 2x$, $\frac{d^2 y}{d x^2} = 2$:
\begin{align*}
  x^2*2 - 3x*2x + 4x^2 &= 0 \\
  2x^2 - 6x^2 + 4x^2 &= 0 \\
  0 &= 0 \\
\end{align*}
Yhtälö toteutuu, joten $y = x^2$ on ratkaisu.

Etsitään toinen kantaratkaisu yritteellä $y_2 = u(x)x^2$,
jolle $\frac{dy_2}{dx} = \frac{du}{dx}x^2 + 2xu$ ja
$\frac{d^2 y_2}{d x^2} = \frac{d^2 u}{d x^2}x^2 + 4x\frac{du}{dx} + 2u$.
\begin{align*}
  x^2(\frac{d^2 u}{d x^2}x^2 + 4x\frac{du}{dx} + 2u)
  - 3x(\frac{du}{dx}x^2 + 2ux) + 4ux^2 &= 0 \\
  x^4\frac{d^2 u}{d x^2} + x^3\frac{du}{dx} &= 0 \\
\end{align*}
Merkitään $v = \frac{du}{dx}$ ja ratkaistaan:
\begin{align*}
  x^4\frac{dv}{dx} + x^3v &= 0 \\
  \frac{dv}{dx} &= -\frac{1}{x}v \\
  \frac{1}{v}\frac{dv}{dx} &= -\frac{1}{x} \\
  \int \frac{1}{v}\,dv &= -\int \frac{1}{x}\,dx \\
  \ln |v| &= -\ln |x| \quad \text{(valitaan C = 0)} \\
  v &= \pm \frac{1}{|x|} \\
\end{align*}
Valitaan mahdollisista yksittäisratkaisuista
$v = \frac{1}{x}$.
Nyt $u = \int v\,dx = \int \frac{1}{x}\,dx = \ln |x| = \ln x$, koska $x > 0$,
ja $y_2 = ux^2 = x^2\ln x$.

Tästä ja annetusta ratkaisusta saadaan lineaarikombinaationa yleinen ratkaisu
\[
  y = C_1x^2 + C_2x^2\ln x, \quad C_1,C_2 \in \mathbb{R}.
\]

\section*{2.}

\subsection*{(I)}

\[
  \frac{d^2 y}{d x^2} + \frac{dy}{dx} - 2y = e^{-x}
\]

Ratkaistaan vastaava homogeeninen yhtälö karakteristisen yhtälön avulla:
\begin{align*}
  r^2 + r - 2 &= 0 \\
  r &= \frac{-1 \pm \sqrt{1 + 8}}{2} \\
    &= \begin{cases} -2 \\ 1 \end{cases} \\
\end{align*}
Saadaan ratkaisu
\[
  y = C_1e^{-2x} + C_2e^{x}, \quad C_1,C_2 \in \mathbb{R}.
\]

\subsubsection*{(a)}

Etsitään epähomogeenisen yhtälön yksittäisratkaisu sijoittamalla yrite
$y_e = Ae^{-x}$.
\begin{align*}
  Ae^{-x} - Ae^{-x} - 2Ae^{-x} &= e^{-x} \\
  -2Ae^{-x} &= e^{-x} \\
  A &= -\frac{1}{2} \\
\end{align*}

Saadaan ratkaisu
\[
  y = C_1e^{-2x} + C_2e^{x} - \frac{1}{2}e^{-x}, \quad C_1,C_2 \in \mathbb{R}.
\]

\subsubsection*{(b)}

Sijoitetaan homogeenisen yhtälön ratkaisuun $C_1 = C_1(x)$ ja $C_2 = C_2(x)$.
Saadaan
\begin{align*}
  y &= C_1(x)e^{-2x} + C_2(x)e^{x}, \\
  \frac{dy}{dx} &= (\frac{dC_1}{dx} - 2C_1)e^{-2x} + (\frac{dC_2}{dx} + C_2)e^x \\
  \noalign{\text{Lisäoletus: $\frac{dC_1}{dx}e^{-2x} + \frac{dC_2}{dx}e^x = 0$}} \\
  \implies \frac{dy}{dx} &= -2C_1e^{-2x} + C_2e^x \\
  \text{ja} \\
  \frac{d^2 y}{d x^2} &= (4C_1 - 2\frac{dC_1}{dx})e^{-2x} + (C_2 + \frac{dC_2}{dx})e^x \\
\end{align*}
Sijoitetaan nämä yhtälöön:
\begin{align*}
  e^{-x} &= \frac{d^2 y}{d x^2} + \frac{dy}{dx} - 2y \\
         &= (4C_1 - 2\frac{dC_1}{dx})e^{-2x} + (C_2 + \frac{dC_2}{dx})e^x
          - 2C_1e^{-2x} + C_2e^x
          - 2(C_1e^{-2x} + C_2e^{x}) \\
         &= -2\frac{dC_1}{dx}e^{-2x} + \frac{dC_2}{dx}e^x \\
         &= -2\frac{dC_1}{dx}e^{-x}e^{-x} + \frac{dC_2}{dx}e^x \\
\end{align*}
Tästä saadaan $\frac{dC_2}{dx} = 0 \implies C_2 = 0$ (valitaan integroimisvakio nolla)
ja
\begin{align*}
  -2\frac{dC_1}{dx}e^{-x} &= 1 \\
  \frac{dC_1}{dx} &= -\frac{1}{2}e^x \\
  C_1 &= -\frac{1}{2}e^x \\
\end{align*}
Saadaan yksittäisratkaisu
\[
  y_e = C_1e^{-2x} = -\frac{1}{2}e^xe{-2x} = -\frac{1}{2}e^{-x}
\]
ja yleinen ratkaisu
\[
  y = C_1e^{-2x} + C_2e^{x} - \frac{1}{2}e^{-x}, \quad C_1,C_2 \in \mathbb{R}.
\]

\section*{3.}

\subsection*{(a)}

\[
  x^2\frac{d^2 y}{d x^2} - (2x + x^2)\frac{dy}{dx} + (2 + x)y = 0
\]
Sijoitetaan $y = x$:
\[
  x^2*0 - (2x + x^2)*1 + (2 + x)x = -2x + 2x - x^2 + x^2 = 0
\]
Yhtälö toteutuu.

Käytetään toisen ratkaisun löytämiseen kertaluvun pudotusta.
Yrite $y_2 = u(x)x$, $\frac{dy_2}{dx} = \frac{du}{dx}x + u$,
$\frac{d^2 y_2}{d x^2} = \frac{d^2 u}{d x^2}x + 2\frac{du}{dx}$.

\begin{align*}
  x^2(\frac{d^2 u}{d x^2}x + 2\frac{du}{dx})
    - (2x + x^2)(\frac{du}{dx}x + u)
    + (2 + x)ux &= 0 \\
  x^3\frac{d^2 u}{d x^2} + (2x^2 - 2x^2 - x^3)\frac{du}{dx} + (-2x - x^2 + 2x + x^2)u &= 0 \\
  x^3\frac{d^2 u}{d x^2} - x^3\frac{du}{dx} &= 0 \\
\end{align*}
Merkitään $v = \frac{du}{dx}$.
\begin{align*}
  x^3\frac{dv}{dx} - x^3v &= 0 \\
  \frac{dv}{dx} &= v \\
  \frac{1}{v}\frac{dv}{dx} &= 1 \\
  \int \frac{1}{v}\,dv &= \int 1\,dx \\
  \ln |v| &= x \quad \text{(valitaan $C=0$)} \\
  v &= \pm e^{x}
\end{align*}
Valitaan näistä $v = e^x$, jolloin saadaan
$u = \int v\,dx = e^x$ ja $y_2 = ux = xe^x$.
Siis yleinen ratkaisu on
\[
  y = C_1x + C_2xe^x, \quad C_1,C_2 \in \mathbb{R}.
\]

\section*{4.}

\subsection*{(a)}

\[
  \frac{d^2 y}{d x^2} - 2\frac{dy}{dx} + y = \frac{e^x}{x}
\]

Ratkaistaan vastaava homogeeninen yhtälö karakteristisen yhtälön avulla:
\begin{align*}
  r^2 - 2r + 1 &= 0 \\
  r &= \frac{2 \pm \sqrt{4 - 4}}{2} \\
    &= 1 \\
\end{align*}
Saadaan ratkaisu
\[
  y = C_1e^x + C_2xe^x, \quad C_1,C_2 \in \mathbb{R}.
\]

Käytetään epähomogeenisen yhtälön ratkaisun etsimiseen vakioiden variointia,
$C_1 = C_1(x)$ ja $C_2 = C_2(x)$, jolloin
\begin{align*}
  \frac{dy}{dx} &= (C_1 + \frac{dC_1}{dx})e^x + C_2(x + 1)e^x + \frac{dC_2}{dx}xe^x \\
                &= (C_1 + \frac{dC_1}{dx} + C_2x + C_2 + \frac{dC_2}{dx}x)e^x \\
  \noalign{\text{Lisäoletus: $\frac{dC_1}{dx} + \frac{dC_2}{dx}x = 0$}} \\
                &= (C_1 + C_2x + C_2)e^x \\
\end{align*}
ja
\begin{align*}
  \frac{d^2 y}{d x^2} &= ((C_1 + C_2x + C_2) + (\frac{dC_1}{dx} + \frac{dC_2}{dx}x + C_2 + \frac{dC_2}{dx}))e^x \\
                      &= (C_1 + C_2(2 + x) + \frac{dC_1}{dx} + 2\frac{dC_2}{dx})e^x \\
\end{align*}
Sijoitetaan nämä yhtälöön:
\begin{align*}
  (C_1 + C_2(2 + x) + \frac{dC_1}{dx} + 2\frac{dC_2}{dx})e^x
  - 2(C_1 + C_2x + C_2)e^x
  + C_1e^x + C_2xe^x &= \frac{e^x}{x} \\
  (\frac{dC_1}{dx} + 2\frac{dC_2}{dx})e^x &= \frac{1}{x}e^x \\
  \frac{dC_1}{dx} + 2\frac{dC_2}{dx} &= \frac{1}{x} \\
\end{align*}
Lisäoletuksesta saadaan lisäksi $\frac{dC_1}{dx} + \frac{dC_2}{dx}x = 0 \iff \frac{dC_1}{dx} = -x\frac{dC_2}{dx}$.
Sijoitetaan:
\begin{align*}
  -x\frac{dC_2}{dx} + 2\frac{dC_2}{dx} &= \frac{1}{x} \\
  \frac{dC_2}{dx} &= \frac{1}{x(2 - x)} \\
\end{align*}
ja
\begin{align*}
  \frac{dC_1}{dx} &= -x\frac{dC_2}{dx} \\
                  &= -x\frac{1}{x(2 - x)} \\
                  &= \frac{1}{x - 2} \\
\end{align*}
Integroimalla saadaan halutut funktiot:
\begin{align*}
  C_1 &= \int \frac{1}{x - 2}\,dx \\
      &= \ln |x - 2| \\
\end{align*}
ja
\begin{align*}
  C_2 &= \int \frac{1}{x(2 - x)}\,dx \\
      &= \int \frac{1}{2}(\frac{1}{x} + \frac{1}{2-x})\,dx \\
      &= \frac{1}{2}(\ln |x| - \ln |2-x|) \\
\end{align*}

Epähomogeenisen yhtälön yksittäisratkaisu saadaan sijoittamalla nämä
homogeenisen yhtälön rat\-kai\-suun:
\begin{align*}
  y_e &= C_1e^x + C_2xe^x \\
      &= \ln|x-2|e^x + \frac{1}{2}(\ln |x| - \ln |2-x|)xe^x \\
\end{align*}

Yleinen ratkaisu on
\[
  y = C_1e^x + C_2xe^x
    + \ln|x-2|e^x + \frac{1}{2}(\ln |x| - \ln |2-x|)xe^x, \quad C_1,C_2 \in \mathbb{R}.
\]

\section*{5.}

\subsection*{(a)}

\[
  y''' - 4y'' + 3y' = 0
\]
Karakteristinen yhtälö:
\begin{align*}
  r^3 - 4r^2 + 3r &= 0 \\
  r(r^2 - 4r + 3) &= 0 \\
  r = 0 \vee r &= \frac{4 \pm \sqrt{16 - 12}}{2} \\
               &= \begin{cases} 1 \\ 3 \end{cases} \\
\end{align*}
Kaikki juuret ovat reaalisia ja yksinkertaisia, joten saadaan ratkaisuksi
\[
  y = C_1 + C_2e^x + C_3e^{3x}, \quad C_1,C_2,C_3 \in \mathbb{R}.
\]

\subsection*{(b)}

\[
  y^{(4)} - 2y'' + y = 0
\]
Karakteristinen yhtälö:
\begin{align*}
  r^4 - 2r^2 + 1 &= 0 \\
  r^2 &= \frac{2 \pm \sqrt{4 - 4}}{2} \\
      &= 1 \\
  r &= \pm \sqrt{1} \\
    &= \pm 1 \\
\end{align*}
Tässä on kaksi kaksinkertaista juurta. Yhtälön ratkaisu on
\[
  y = C_1e^x + C_2xe^x + C_3e^{-x} + C_4xe^{-x}, \quad C_1,C_2,C_3,C_4 \in \mathbb{R}.
\]

\section*{6.}

\[
  y''' - 2y' - 4y = 0
\]
Sijoitetaan $y = e^{2x}$:
\begin{align*}
  8e^{2x} - 4e^{2x} - 4e^{2x} &= 0 \\
  0 &= 0 \qquad \qedsymbol \\
\end{align*}
Tästä voidaan päätellä, että $r = 2$ on yksi karakteristisen yhtälön ratkaisu,
mikä auttaa karakteristisen yhtälön tekijöihin jakamisessa:
\begin{align*}
  r^3 - 2r - 4 &= 0 \\
  (r-2)(r^2 + 2r + 2) &= 0 \\
  r = 2 \vee r &= \frac{-2 \pm \sqrt{4 - 8}}{2} \\
               &= -1 \pm 2i \\
\end{align*}
Saadaan yhtälön ratkaisu
\[
  y = C_1e^{2x} + C_2e^{-x}\cos(2x) + C_3e^{-x}\sin(2x), \quad C_1,C_2,C_3 \in \mathbb{R}.
\]

\section*{7.}

\[
  (r^2 - r - 2)^2(r^2 - 4)^2 = 0
\]
Ratkaistaan yhtälön tekijöiden juuret erikseen.
\begin{align*}
  r^2 - r - 2 &= 0 \\
  r &= \frac{1 \pm \sqrt{1 + 8}}{2} \\
    &= \begin{cases} 2 \\ -1 \end{cases}
\end{align*}
Koska alkuperäisessä yhtälössä tämä termi on neliönä, niin nämä ovat sen
kaksinkertaisia ratkai\-suja.
\begin{align*}
  r^2 - 4 &= 0 \\
  r &= \pm 2 \\
\end{align*}
Samoin nämä ovat kaksinkertaisia ratkaisuja. Kaikkiaan saadaan
nelinkertainen juuri 2 ja kaksin\-kertaiset juuret -1 ja -2.
Yhtälön ratkaisu on siis
\[
  y = C_1e^{2x} + C_2xe^{2x} + C_3x^2e^{2x} + C_4x^3e^{2x}
    + C_5e^{-x} + C_6xe^{-x} + C_7e^{-2x} + C_8xe^{-2x},
    \quad C_i \in \mathbb{R}.
\]

\section*{8.}

\subsection*{(a)}

\[
  x^2\frac{d^2 y}{d x^2} - x\frac{dy}{dx} + y = 0
\]
Karakteristinen yhtälö:
\begin{align*}
  r(r - 1) - r + 1 &= 0 \\
  r^2 - 2r + 1 &= 0 \\
  r &= \frac{2 \pm \sqrt{4 - 4}}{2} \\
    &= 1 \\
\end{align*}
Yleinen ratkaisu on
\[
  y = C_1|x| + C_2|x|\ln|x|, \quad C_1,C_2 \in \mathbb{R},
\]
kun $x \neq 0$.

\subsection*{(b)}

\[
  x^2\frac{d^2 y}{d x^2} - x\frac{dy}{dx} - 3y = 0
\]
Karakteristinen yhtälö:
\begin{align*}
  r(r-1) - r - 3 &= 0 \\
  r^2 - 2r - 3 &= 0 \\
  r &= \frac{2 \pm \sqrt{4 + 12}}{2} \\
    &= \begin{cases} 3 \\ -1 \end{cases} \\
\end{align*}
Yhtälön ratkaisu on
\[
  y = C_1|x|^3 + C_2|x|^{-1}, \quad C_1, C_2 \in \mathbb{R}.
\]

\subsection*{(c)}

\[
  x^2\frac{d^2 y}{d x^2} - x\frac{dy}{dx} + 5y = 0
\]
Karakteristinen yhtälö:
\begin{align*}
  r(r - 1) - r + 5 &= 0 \\
  r^2 - 2r + 5 &= 0 \\
  r &= \frac{2 \pm \sqrt{4 - 20}}{2} \\
    &= 1 \pm 2i \\
\end{align*}
Yhtälön ratkaisu on
\[
  y = C_1|x|\sin(2\ln |x|) + C_2|x|\cos(2\ln |x|), \quad C_1,C_2 \in \mathbb{R}.
\]

\section*{9.}

\subsection*{(a)}

\[
  x^2\frac{d^2 y}{d x^2} + x\frac{dy}{dx} - y = x^2, \quad x > 0
\]
Ratkaistaan vastaava homogeeninen yhtälö karakteristisen yhtälön avulla:
\begin{align*}
  r(r-1) + r - 1 &= 0 \\
  r^2 - 1 &= 0 \\
  r &= \pm 1 \\
\end{align*}
Saadaan ratkaisu
\[
  y_h = C_1x + C_2x^{-1}, \quad C_1,C_2 \in \mathbb{R}.
\]
(ei itseisarvoja, koska on määrätty $x > 0$.)

Etsitään epähomogeenisen yhtälön yksittäisratkaisua yritteellä $y_e = Ax^2$.
\begin{align*}
  x^2(2A) + x(2Ax) - Ax^2 &= x^2 \\
  3Ax^2 &= x^2 \\
  A &= \frac{1}{3} \\
\end{align*}
Saadaan yleiseksi ratkaisuksi
\[
  y_h = C_1x + C_2x^{-1} + \frac{1}{3}x^2, \quad C_1,C_2 \in \mathbb{R}.
\]

\subsection*{(b)}

\[
  x^2\frac{d^2 y}{d x^2} + x\frac{dy}{dx} - y = x, \quad x > 0
\]
Vastaava homogeeninen yhtälö ratkaistiin (a)-kohdassa.
Epähomogeenisen yhtälön yksittäis\-ratkai\-sun löytymiseksi täytyy päästä eroon
toisen derivaatan kertoimena olevasta $x^2$:sta, mikä onnistuu ainakin yritteellä,
jolle $\frac{d^2 y_e}{d x^2} = 0$. Kokeillaan $y_e = Ax$:
\begin{align*}
  x^2*0 + x*A - Ax &= x \\
  0 &= x \\
\end{align*}
Ei toiminut. Muunlaisen yritteen keksiminen on vaikeaa, joten käytetään
vakioiden variointia:
\begin{align*}
  y_e &= C_1(x)x + C_2(x)x^{-1} \\
  \frac{dy_e}{dx} &= \frac{dC_1}{dx}x + C_1 + \frac{dC_2}{dx}x^{-1} - C_2x^{-2} \\
  \noalign{\text{Lisäoletus $\frac{dC_1}{dx}x + \frac{dC_2}{dx}x^{-1} = 0$:}} \\
                  &= C_1 - C_2x^{-2} \\
  \frac{d^2 y_e}{d x^2} &= \frac{dC_1}{dx} - (\frac{dC_2}{dx}x^{-2} - 2C_2x^{-3}) \\
\end{align*}
Sijoitetaan yhtälöön:
\begin{align*}
  x^2(\frac{dC_1}{dx} - \frac{dC_2}{dx}x^{-2} + 2C_2x^{-3})
  + x(C_1 - C_2x^{-2}) - (C_1x + C_2x^{-1}) &= x \\
  x^2\frac{dC_1}{dx} - \frac{dC_2}{dx} &= x \\
  \frac{dC_2}{dx} &= x^2\frac{dC_1}{dx} - x \\
\end{align*}
Sijoitetaan lisäoletukseen:
\begin{align*}
  \frac{dC_1}{dx}x + (x^2\frac{dC_1}{dx} - x)x^{-1} &= 0 \\
  2x\frac{dC_1}{dx} - 1 &= 0 \\
  \frac{dC_1}{dx} &= \frac{1}{2x} \\
\end{align*}
Sijoitetaan takaisin edelliseen yhtälöön:
\begin{align*}
  \frac{dC_2}{dx} &= x^2\frac{1}{2x} - x \\
                  &= \frac{x}{2} - x \\
                  &= -\frac{1}{2}x \\
\end{align*}
Integroidaan (valitaan integroimisvakio nolla):
\begin{align*}
  C_1 &= \int \frac{1}{2x}\,dx \\
      &= \frac{1}{2}\ln x \quad (x > 0) \\
\end{align*}
\begin{align*}
  C_2 &= -\int \frac{1}{2}x\,dx \\
      &= -x^2 \\
\end{align*}
Saadaan yhtälön yleinen ratkaisu
\begin{align*}
  y &= C_1x + C_2x^{-1} + \frac{1}{2}x\ln x - x \\
    &= C_1x + C_2x^{-1} + \frac{1}{2}x\ln x, \quad C_1, C_2 \in \mathbb{R}.
\end{align*}

\section*{10.}

\subsection*{(a)}

\[
  x^2\frac{d^2 y}{d x^2} + x\frac{dy}{dx} + (x^2 - \frac{1}{4})y = 0
\]
Sijoitetaan $y = x^{-\frac{1}{2}}$:
\begin{align*}
  0 &= x^2(\frac{3}{4}x^{-\frac{5}{2}}\cos x + x^{-\frac{3}{2}}\sin x
   - x^{-\frac{1}{2}}\cos x) \\
    &\qquad+ x(-\frac{1}{2}x^{-\frac{3}{2}}\cos x - x^{-\frac{1}{2}}\sin x) \\
    &\qquad+ (x^2 - \frac{1}{4})x^{-\frac{1}{2}}\cos x \\
    &= (\frac{3}{4}x^{-\frac{1}{2}} - x^{\frac{3}{2}} - \frac{1}{2}x^{-\frac{1}{2}}
    + x^{\frac{3}{2}} - \frac{1}{4}x^{-\frac{1}{2}})\cos x \\
    &\qquad + (\frac{1}{2}x^{\frac{3}{2}} + \frac{1}{2}x^{\frac{3}{2}} - x^{\frac{3}{2}})\sin x \\
    &= 0\cos x + 0\sin x = 0 \qquad \qedsymbol
\end{align*}

Yleinen ratkaisu kertaluvun pudotuksella, yrite $u(x)x^{-\frac{1}{2}}\cos x$:
\begin{align*}
  0 &= x^2(u(\frac{3}{4}x^{-\frac{5}{2}}\cos x + x^{-\frac{3}{2}}\sin x
  - x^{-\frac{1}{2}}\cos x)
  + \frac{du}{dx}(-\frac{1}{2}x^{-\frac{3}{2}}\cos x - x^{-\frac{1}{2}}\sin x) \\
    &\qquad\qquad+ \frac{du}{dx}(\frac{1}{2}x^{-\frac{3}{2}}\sin x - x^{-\frac{1}{2}}\cos x)
    + \frac{d^2 u}{d x^2}(x^{-\frac{1}{2}}\cos x)
  ) \\
    &\qquad+ x(u(-\frac{1}{2}x^{-\frac{3}{2}}\cos x - x^{-\frac{1}{2}}\sin x)
  + \frac{du}{dx} x^{-\frac{1}{2}} \cos x) \\
    &\qquad+ (x^2 - \frac{1}{4})ux^{-\frac{1}{2}}\cos x \\
    &= (\frac{3}{4}x^{-\frac{1}{2}}u - x^{\frac{3}{2}}u - \frac{1}{2}x^{\frac{1}{2}}\frac{du}{dx}
    - x^{\frac{3}{2}}\frac{du}{dx} + x^{\frac{3}{2}}\frac{d^2 u}{d x^2}
    - \frac{1}{2}x^{-\frac{1}{2}}u + x^{\frac{1}{2}}\frac{du}{dx}
    + x^{\frac{3}{2}}u - \frac{1}{4}x^{-\frac{1}{2}}u) \cos x \\
    &\qquad+ (x^{\frac{1}{2}}u - x^{\frac{3}{2}}\frac{du}{dx}
    + \frac{1}{2}x^{\frac{1}{2}}\frac{du}{dx} - x^{\frac{1}{2}}u) \sin x \\
    &= (-x^{\frac{3}{2}}\frac{du}{dx} + x^{\frac{3}{2}}\frac{d^2 u}{d x^2}) \cos x
    + (\frac{1}{2}x^{\frac{1}{2}} - x^{\frac{3}{2}})\frac{du}{dx}\sin x \\
\end{align*}
Nyt on 13 tunnin työpäivän jäljiltä pää niin sekaisin, että en pääse tästä enää eteenpäin.
Jätän kesken.

\end{document}

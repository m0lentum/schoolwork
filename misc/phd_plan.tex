%! TEX program = xelatex

\documentclass{article}
\usepackage[a4paper, margin=3cm]{geometry}
\setlength{\parindent}{0pt}
\setlength{\parskip}{1em}
\usepackage{titlesec}
\titlespacing\section{0pt}{16pt plus 4pt minus 2pt}{2pt plus 2pt minus 2pt}
\usepackage{fontspec}
\setmainfont{Lato}
\usepackage[english]{babel}

\usepackage{amsmath,amssymb,amsthm}
\usepackage{csquotes}

%
% begin document
%

\usepackage[authordate,noibid,backend=biber]{biblatex-chicago}
\addbibresource{phd_plan.bib}


\title{Doctoral research plan: \\
Discrete exterior calculus for Maxwell-Vlasov plasma flow}
\author{Mikael Myyrä \\
Faculty of Information Technology \\
University of Jyväskylä}
\date{\number\day.\number\month.\number\year}

\begin{document}
\maketitle

\section*{Background}

Discrete exterior calculus (DEC) is a method of discretizing
differential equations expressed in the exterior calculus of differential forms.
It was introduced by \textcite{desbrun_discrete_2005},
incorporating and generalizing ideas originating
in the finite-difference time domain (FDTD)
method of \textcite{yee_numerical_1966} and the work of Bossavit
\parencite*{bossavit_geometry_1998,bossavit_computational_1999,
bossavit_computational_2000,bossavit_generalized_2001,bossavit_discretization_2005}.
It has been applied in a variety of simulation domains,
including electromagnetics \parencite{rabina_efficient_2015,monkola_discrete_2022},
Darcy flow \parencite{hirani_numerical_2015},
quantum mechanics \parencite{rabina_three-dimensional_2018,kivioja_evolution_2023},
and fluid dynamics \parencite{mohamed_discrete_2016,nitschke_discrete_2017}.
I have personally applied DEC to acoustic scattering problems
in my master's thesis \parencite{myyra_discrete_2023} 
in combination with an optimization method to obtain time-harmonic solutions,
and the computational science working group at the JYU Faculty of Information Technology
has an ongoing history of research on the topic,
including the aforementioned \textcite{rabina_efficient_2015}, \textcite{rabina_three-dimensional_2018},
and \textcite{monkola_discrete_2022},
as well as theoretical work developing a DEC-based framework
for a general class of wave propagation problems \parencite{rabina_generalized_2018}.

DEC has a number of advantageous properties.
It works on unstructured computation meshes,
which enables accurate approximations of complex or irregular geometry,
solving a problem that methods reliant on structured meshes
such as classical finite difference methods struggle with.
The geometric flexibility of DEC also makes it possible to run simulations
in curved manifold domains such as relativistic spacetime.
Another benefit of DEC is its computational efficiency
arising from the diagonal Hodge operator,
which produces time-stepping formulas where only a diagonal matrix needs to be inverted.
In contrast, finite element methods
often require the inversion of a general matrix, which is expensive
and limits the achievable scale.

Despite its potential benefits,
DEC is not yet widely known.
Our research demonstrating new practical applications
will provide valuable information on its real-world performance.

Our main focus in this research is on plasma flows described by the Maxwell-Vlasov equations,
which appear in e.g. space weather (solar winds and northern lights)
and nuclear fusion reactors.
Like most fluid flows, this is a challenging phenomenon to model.
Moving particles carry electromagnetic fields
which in turn affect the future movement of particles,
creating a complex coupling.
\textcite{palmroth_vlasov_2018} cover a variety of approaches to deal with this problem
in the context of existing (non-DEC) solvers for the Maxwell-Vlasov equations,
many of which are quite computationally expensive.
Additional approaches to the problem have been developed specifically for DEC,
such as the one employed by \textcite{mohamed_discrete_2016}
in the context of the Navier-Stokes equations,
which may prove useful for our applications.

\section*{Objectives and methods}

We seek to develop two- and three-dimensional DEC-based discretizations
for the Maxwell-Vlasov equations
and examine their performance characteristics such as accuracy,
stability, and computational cost.
Questions such as the following will be investigated:
\begin{itemize}
  \item In a simple scenario with a known solution,
  what is the error of the discrete simulation?
  \item What is the largest stable timestep for the simulation?
  \item Does the simulation remain stable when the complexity of geometry increases?
  If not, why?
  \item How do the answers to preceding questions change
  when the density of mesh elements changes?
\end{itemize}

We will answer these questions by deriving a discrete model from the continuous equations,
implementing computer simulations based on this model,
and measuring their outcomes.
We expect to see computational efficiency that is competitive with state-of-the-art methods
and accuracy and stability characteristics that depend on properties of the computation mesh.
As a possible additional research topic, methods to optimize the mesh
to minimize error may be investigated.

\section*{Results}

The result of this work will be a compilation dissertation
consisting of multiple journal articles.
The primary artifacts we intend to develop
are a new discrete mathematical model of the studied phenomenon
and a practical implementation thereof,
providing theoretical and technical references
for researchers and engineers working on similar problems.
We aim to create a two- and three-dimensional version of the model,
as well as a GPU implementation providing real-time performance at larger scales.
Each of these will yield its own journal article.

Additionally, an open-source software library implementing the operations of DEC
and visualization utilities
for the Rust programming language
has already been developed and will be refined as part of this research,
providing a reusable foundation for future projects.
The source code of this library is available
in a public GitHub repository \parencite{myyra_m0lentumdexterior_2024},
licensed under a dual MIT/Apache 2.0 license following the Rust convention.

We have also planned two additional articles;
firstly, about DEC and exact controllability in time-harmonic acoustics
based on research already done for my master's thesis \parencite{myyra_discrete_2023},
and secondly, an engineering-focused article about the technical developments
made in the aforementioned software library.

\section*{Timetable}

I will begin my doctoral studies with a period of study and development
to build the necessary skills to derive our mathematical models
and the foundational software tools to implement our simulations.
After this we'll build our model and simulations,
starting with a two-dimensional case,
then moving on to three dimensions
and ultimately to a GPU implementation,
publishing an article at each stage.
The other two articles about acoustics and the technical implementation
will be interspersed with these steps,
with the acoustics being first
and the implementation later after the technology has been developed further
and tested in practice with the Maxwell-Vlasov simulations.

The exact timing of these steps depends on funding.
It is presently unclear whether I will be able to work on this research full-time.
However, since this is independent work with no schedule constraints,
this is not a problem.
We will progress in the described order with the pace determined by available work hours.

\section*{Resources and collaboration}

DEC is an active topic of research with the computational science working group at JYU.
Sanna Mönkölä, Tuomo Rossi, and Tytti Saksa
have previously advised me in my master's thesis on the topic
and I look forward to continuing to work with them.

\printbibliography

\end{document}

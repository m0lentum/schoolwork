%! TEX program = xelatex

\documentclass{article}
\usepackage[a4paper, margin=3cm]{geometry}
\setlength{\parindent}{0pt}
\setlength{\parskip}{1em}
\usepackage{titlesec}
\titlespacing\section{0pt}{16pt plus 4pt minus 2pt}{2pt plus 2pt minus 2pt}
\usepackage{fontspec}
\setmainfont{Lato}
\usepackage[english]{babel}

\usepackage{amsmath,amssymb,amsthm}
\usepackage{csquotes}

%
% begin document
%

\usepackage[authordate,noibid,backend=biber]{biblatex-chicago}
\addbibresource{phd_plan.bib}


\title{Doctoral research plan: \\
Discrete exterior calculus models for fluid flow phenomena}
\author{Mikael Myyrä \\
Faculty of Information Technology \\
University of Jyväskylä}
\date{\number\day.\number\month.\number\year}

\begin{document}
\maketitle

\section*{Background}

Discrete exterior calculus (DEC) is a method of discretizing
differential equations expressed in the exterior calculus of differential forms.
It was introduced by \textcite{desbrun_discrete_2005},
incorporating and generalizing ideas originating
in the finite-difference time domain (FDTD)
method of \textcite{yee_numerical_1966} and the work of Bossavit
\parencite*{bossavit_geometry_1998,bossavit_computational_1999,
bossavit_computational_2000,bossavit_generalized_2001,bossavit_discretization_2005}.
It has been applied in a variety of simulation domains,
including electromagnetics \parencite{rabina_efficient_2015,monkola_discrete_2022},
Darcy flow \parencite{hirani_numerical_2015},
quantum mechanics \parencite{rabina_three-dimensional_2018,kivioja_evolution_2023},
and fluid dynamics \parencite{mohamed_discrete_2016,nitschke_discrete_2017}.
I have previously applied DEC to acoustic scattering problems
in my master's thesis \parencite{myyra_discrete_2023} 
in combination with an optimization method to obtain time-harmonic solutions.

This method has a variety of potential applications in engineering.
DEC generates computationally efficient and easily parallelizable discretizations,
which can enable increasingly complex experimental scenarios
and industrial designs to be simulated interactively.
DEC also introduces new possibilities in spatial domains
with its ability to extend to curved manifolds.
The method is not yet widely known,
and our research demonstrating new practical applications
will provide valuable information on its real-world performance.

\section*{Objectives and methods}

We seek to develop DEC-based discretizations for mathematical models of fluid flow,
including the Navier-Stokes equations for incompressible fluids,
the Maxwell-Vlasov equations for plasma flow,
and the Boltzmann equation for compressible fluids,
and examine their performance characteristics such as accuracy,
stability, and computational cost.
Questions such as the following will be investigated for each model:
\begin{itemize}
  \item In a simple scenario with a known solution,
  what is the error of the discrete simulation?
  \item What is the largest stable timestep for the simulation?
  \item Does the simulation remain stable when the complexity of geometry increases?
  If not, why?
  \item How do the answers to preceding questions change
  when the density of mesh elements changes?
\end{itemize}

We will answer these questions by deriving discrete models from the continuous equations,
implementing computer simulations based on these models,
and measuring their outcomes.
We expect to see computational efficiency that is competitive with state-of-the-art methods
and accuracy and stability characteristics that depend on properties of the computation mesh.
As a possible additional research topic, methods to optimize the mesh
to minimize error may be investigated.

\section*{Results}

This thesis will be a compilation dissertation exploring multiple appliations of DEC.
The results we intend to develop are new discrete mathematical models of the studied phenomena
and practical implementations thereof,
providing theoretical and technical references
for researchers and engineers working on similar problems.
Additionally, an open-source software library implementing the operations of DEC
for the Rust programming language
has already been developed and will be refined as part of this research,
providing a reusable foundation for future projects.
The source code of this library is available
in a public GitHub repository \parencite{myyra_m0lentumdexterior_2024},
licensed under a dual MIT/Apache 2.0 license following the Rust convention.

\section*{Timetable}

We will begin by publishing an article on DEC and exact controllability
in time-harmonic acoustics, research already done for my master's thesis
\parencite{myyra_discrete_2023}.
Following this will be a period of study and development
to build the necessary skills to derive and analyze our mathematical models
and the foundational software tools to implement our simulations.
After this we'll build our models and simulations,
publishing an article for each.
There is no requirement for the order in which these are done,
since each only depends on the foundational work in the previous phase.
Our initial idea is to begin with the Navier-Stokes equations,
follow with the Maxwell-Vlasov equations and end with the Boltzmann equation.

The exact timing of these steps depends on funding.
It is presently unclear whether I will be able to work on this research full-time.
However, since this is independent work with no schedule constraints,
this is not a problem.
We will progress in the described order with the pace determined by available work hours.

\section*{Resources and collaboration}

DEC is an active topic of research with the scientific computing working group at JYU.
Sanna Mönkölä, Tuomo Rossi, and Tytti Saksa
have previously advised me in my master's thesis on the topic
and I look forward to continuing to work with them.

\printbibliography

\end{document}

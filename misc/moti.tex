%! TEX program = xelatex

\documentclass{article}
\usepackage[a4paper, margin=3cm]{geometry}
\setlength{\parindent}{0pt}
\setlength{\parskip}{1em}
\usepackage{titlesec}
\titlespacing\section{0pt}{16pt plus 4pt minus 2pt}{2pt plus 2pt minus 2pt}
\usepackage{fontspec}
\setmainfont{Lato}
\usepackage[finnish]{babel}

\usepackage{amsmath,amssymb,amsthm}

\title{Motivaatiokirje - tohtorikoulutuspilotti}
\author{Mikael Myyrä}
\date{\number\day.\number\month.\number\year}

\begin{document}
\maketitle

Tervehdys. Olen juuri valmistumassa oleva
jyväskyläläinen laskennallisen tieteen maisteriopiskelija,
jonka uteliaisuus on tähänastisten opintojen aikana vain kasvanut.
Maisterin tutkinto on ollut antoisa pintaraapaisu monisyiseen ja haastavaan tieteenalaan,
mutta luonteeni ei salli lopettaa pintaraapaisuun.
Haluan syventyä asiaan kunnolla.
Tohtorikoulutuspilotin rahoitus olisi mahdollisuus irrottautua päivätyöstäni
ja käyttää syventymiseen sen ansaitsema aika.
Olen myös mahdollisesti kiinnostunut jatkamaan tutkijan työtä
tohtoriopintojen jälkeen.

Olen ammatti- ja harrastustaustani sekä tietotekniikan kandidaattiopintojen ansiosta
kokenut ja taitava ohjelmoija.
Siksi olen maisteriopinnoissani keskittynyt toisen intohimon kohteeni, matematiikan oppimiseen.
Tein gradututkielmani IT-tiedekunnan laskennallisen tieteen tutkimusryhmän 
(Tuomo Rossi, Sanna Mönkölä ja Tytti Saksa) ohjaamana,
ja olemme suunnitelleet yhteistyön jatkamista tohtoriopinnoissani.
Samoin jatkuu keskittyminen matemaattisten taitojeni vahvistamiseen.
Uskon, että työyhteisö hyötyy kokemuksestani
laadukkaiden ja helposti ylläpidettävien ohjelmistojen rakentamisessa,
ja minä puolestani pääsen oppimaan tutkimusryhmän
matemaattisesta ja tieteellisestä asiantuntemuksesta.

Ohjelmointitaitojeni lisäksi olen poikkeuksellisen nopea oppija ja tehokas työntekijä.
Minut palkittiin kandidaattiopintojeni aikana stipendillä 75 opintopisteen suorittamisesta
yhden lukuvuoden aikana, ja työnantajani sekä graduohjaajani ovat usein
hämmästelleet nopeutta, jolla hoidan tehtäväni.
Siksi olen varma, että pystyn tekemään tutkintoni valmiiksi kolmen vuoden rahoitusajassa.

Olen valmistautunut tohtoriopintoihini suunnittelun lisäksi myös ohjelmistokehityksen muodossa.
Olen rakentanut yleiskäyttöisen alustan, jonka tarjoamista osista
tutkimuksessa tuotettavat simulaatiot voidaan koota luotettavasti
tarvitsematta keskittyä pieniin yksityiskohtiin.
Tämä nopeuttaa osaltaan tutkimuksen etenemistä
ja on myös itsessään mahdollinen tieteellisen julkaisun aihe.
Olen käyttänyt tähän paljon vapaa-aikaani
huolimatta siitä, että tohtoriopinto-oikeus ei ole vielä varma asia,
koska nautin tekemisestä
ja luotan tämän olevan vahva konkreettinen osoitus motivaatiostani tähän tehtävään.

Nykyinen työnantajani on Jyväskylän yliopiston digipalvelut,
joten minulla on kokemusta yliopistoyhteisöstä myös taustapalvelujen näkökulmasta,
ja osaan arvostaa opetuksen ja tutkimuksen ylläpitämiseen vaadittavaa työtä.
Olen viihtynyt tässä organisaatiossa erinomaisen hyvin
sekä opiskelijana että työntekijänä,
ja tohtoriopinnot antavat jälleen uuden innostavan näkökulman.

Kiitän tilaisuudesta ja osallistun mielelläni tarvittaessa haastatteluun.

Mikael

\end{document}

%! TEX program = xelatex

\documentclass{article}
\usepackage[a4paper, margin=3cm]{geometry}
\setlength{\parindent}{0pt}
\setlength{\parskip}{1em}
\usepackage{titlesec}
\titlespacing\section{0pt}{16pt plus 4pt minus 2pt}{2pt plus 2pt minus 2pt}
\usepackage{fontspec}
\setmainfont{Lato}
\usepackage[english]{babel}
\usepackage{hyperref} % links

\usepackage{amsmath,amssymb,amsthm}

% malli: https://tenk.fi/sites/default/files/2021-06/TENK_Tutkijan_ansioluettelomalli_2020.pdf

\title{Ansioluettelo}
\author{Mikael Benjamin Myyrä}
\date{\number\day.\number\month.\number\year}

\begin{document}
\maketitle

\section*{Tutkinnot ja oppiarvot}

\begin{itemize}
  \item Luonnontieteiden maisteri, tietotekniikka.
    Jyväskylän yliopisto, informaatioteknologian tiedekunta,
    teknis-matemaattisen mallintamisen opintosuunta.
    Kirjoittamishetkellä kesken, valmistuminen kesällä 2024.
  \item Luonnontieteiden kandidaatti, tietotekniikka.
    Jyväskylän yliopisto, informaatioteknologian tiedekunta.
    2019.
\end{itemize}

\section*{Kielitaito}

Äidinkielenä suomi. Lisäksi erinomainen englannin kielen taito,
joka on arvioitu pro gradu -tutkielman kypsyysnäytteessä.

\section*{Nykyinen työ}

Sovellussuunnittelija, Jyväskylän yliopiston digipalvelut, alkaen 2018.
Työnkuvana pääasiallisesti web-sovellusten ohjelmointi.
Samalla osa-aikaisesti opiskelijana Jyväskylän yliopistossa.

\section*{Tutkimuksen tuotokset}

Julkaisuja toistaiseksi vain opinnäytetyöt.
\begin{itemize}
  \item Pro gradu -tutkielma:
    Discrete exterior calculus and exact controllability
    for time-harmonic acoustic wave simulation. (2023).
    \url{https://jyx.jyu.fi/handle/123456789/89358}
  \item Kandidaatintutkielma:
    Reaaliaikaiset nestesimulaatiot. (2019).
    \url{https://jyx.jyu.fi/handle/123456789/64632}
\end{itemize}

Lisäksi avoimen lähdekoodin ohjelmistoja:
\begin{itemize}
  \item dexterior: Discrete Exterior Calculus toolkit for Rust.
    \url{https://github.com/m0lentum/dexterior}
\end{itemize}

\section*{Palkinnot ja huomionosoitukset}

2018 Jyväskylän yliopiston informaatioteknologian tiedekunnan myöntämä stipendi
yli 75 opintopisteen suorittamisesta lukuvuoden aikana.

\end{document}

%! TEX program = xelatex

\documentclass{article}
\usepackage[a4paper, margin=3cm]{geometry}
\setlength{\parindent}{0pt}
\setlength{\parskip}{1em}
\usepackage{titlesec}
\titlespacing\section{0pt}{16pt plus 4pt minus 2pt}{2pt plus 2pt minus 2pt}
\usepackage{fontspec}
\setmainfont{Lato}
\usepackage[english]{babel}

\usepackage{amsmath,amssymb,amsthm}

\title{Jatko-opintosuunnitelma}
\author{Mikael Myyrä}
\date{\number\day.\number\month.\number\year}

\begin{document}
\maketitle

En tarvitse täydennystä ohjelmointitaitoihini,
joten voin keskittyä jatko-opinnoissani matematiikkaan,
jossa esitietoni ovat heikommat.

% ks. https://opinto-opas.jyu.fi/2024/fi/moduuli/matsyveri/

Tutkimukseni aihepiiriin sopivia opintojaksoja,
joita en ole vielä suorittanut maisteriopinnoissani:

\begin{itemize}
  \item MATA255 Vektorianalyysi 1 (4op) % https://opinto-opas.jyu.fi/2024/fi/opintojakso/mata255/
  \item MATS213 Metriset avaruudet (5op)
  \item MATS214 Topologia (4op)
  \item MATS215 Algebrallinen topologia (9op) % https://opinto-opas.jyu.fi/2024/fi/opintojakso/mats215/
  \item MATS2310 Osittaisdifferentiaaliyhtälöt 1A (5op)
  \item MATS2320 Osittaisdifferentiaaliyhtälöt 1B (5op)
\end{itemize}

Muita, vähemmän ydinasioita mutta kuitenkin relevantteja jaksoja:

\begin{itemize}
  \item MAST121 Kompleksianalyysi 1 (5op)
    (esitieto MATS215 Algebralliselle topologialle)
  \item MATS2210 Hilbert-avaruudet (5op)
  \item MATS2220 Banach-avaruudet (5op)
  \item MATS235 Sobolev-avaruudet (5op)
  \item MATA221 Algebra 1: Ryhmät (5op)
  \item MATA221 Algebra 1: Renkaat ja kunnat (5op)
\end{itemize}

Aikataulun suunnittelu ei ole vielä mahdollista,
koska opetusaikataulua ei ole vielä julkaistu,
eikä ole selvää, teenkö opintoja osa-aikaisesti vai kokopäiväisesti.
Valikoin tässä mainituista opintojaksoista aikatauluuni sopivia
ja opiskelen muiden asioita tarpeellisin osin omatoimisesti tutkimuksen yhteydessä.

\end{document}
